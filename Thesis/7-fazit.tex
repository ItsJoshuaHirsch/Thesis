\chapter{Diskussion}
\label{fazit}
\minitoc\pagebreak

\section{Fazit}
Ziel dieser Arbeit war die Bearbeitung der zugrundeliegenden Fragestellungen (\ref{fragen}).
Demnach sollten Anforderungen respektive Fragen der Nutzergruppen gesammelt und auf benötigte Daten analysiert werden.
Aus diesen Erkenntnissen resultiert die Modellierung eines geeigneten Datenmodells und auf Basis dieses Modells sollten entsprechend aussagekräftige Dashboards entwickelt werden.

Es wurden diverse Interviews mit vielen repräsentativen Personen geführt, um ein Bild der im Raum stehenden Fragen zu erhalten.
Dabei lag der Schwerpunkt auf interne Mitarbeiter, welche aufgrund des häufigen Hintergrundes im Rettungsdienst oder Kontakt zu diesem durchaus eine gefestigte Gruppe mit einer repräsentativen Meinung darstellen.
Es konnte auch ein Treffen zu dieser Thematik mit einem tatsächlichen Kunden der Nutzergruppe \glqq Ärztlicher Leiter Rettungsdienst\grqq{} arrangiert werden, aus welchem die Erkenntnisse mit besonderer Gewichtung in die weitere Entwicklung eingeflossen sind.
Hierbei wurde anfangs die Methodik der Einzelinterviews angewandt, welche sich retrospektiv als positiv erwiesen hat, da die unvoreingenommene Meinung der Einzelpersonen dadurch abgefragt werden konnte.
So entstand ein großes Sammelwerk an Anforderungen, welche in dieser Arbeit die Form von Fragestellungen haben (Auszug im Anhang \ref{att:anforderung}).
Diese wurden in zyklisch-iterativen Vorgängen später weiter analysiert und spezifiziert, so wie es die Prozesse des Requirements Engineering vorsehen. 
%\cite{ Bergsmann.2018, Leffingwell.2011, Patig., Pohl.2011}. 
Dabei wurden auch Gruppendiskussionen angesetzt, um ein kollaboratives Evaluieren und eine Auseinandersetzung mit anderen Meinungen zu ermöglichen.

Parallel wurden erste konzeptionelle Prototypen von Diagrammen entwickelt, um herauszufinden, welche Fragen bereits beantwortet werden können.
Auch erste Rückschlüsse für ein Datenmodell konnten hierbei extrahiert werden.
Dabei wurden verschiedene Modelle anhand von methodischen Schemata und Architekturen aus Literatur.
% (\cite{Bauer.2004, Gabriel.2011, Gitelman.2013, Kimball.2013, Mucksch.2000}) herangezogen und mit den vorliegenden Daten untersucht.
Diese Prototypen wurden wiederum mit den entsprechend befähigten Mitarbeitern der Firma evaluiert, sodass auch hier etwaige Kritik berücksichtigt und schnell darauf reagiert werden kann.

Anhand der Ergebnisse dieser Untersuchungen konnten tatsächliche Anforderungen an die Software \gls{ANALYSE} gestellt werden, ob neue Daten aus den Einsätzen zur Verfügung gestellt werden müssen.
Auch die Spezifikation, in welchem Format die Daten benötigt werden, 
%(Anhang \ref{att:stories}), 
sowie das bestmögliche Export-Format (\ref{sub:export}) waren Bestandteil dieser Arbeit. 

Da all diese Prozesse mehr oder minder parallel abliefen, ist eine strikte Trennung nicht möglich.
Es wurden stets neue Fragen erhoben, bestehende analysiert und spezifiziert und währenddessen wurde an dem Datenmodell, Prototyp oder der letztendlichen Umsetzung der Qlik-App gearbeitet.
Dabei wurden auch immerzu diverse Richtlinien, Usability-Erkenntnisse und Grundsätze zur Visualisierung von Daten und Gestaltung von Dashboards 
%(\cite{Bassler.2010, Card.2007, Fischer.2014, FischerStabel.2018, Hichert.2017, Kertzel.2018, Mccormick.1987, Schumann.2000, Ware.2009}) 
berücksichtigt, um möglichst konfliktfreie und ansprechende Auswertungen für die Anwender zu ermöglichen.
Die endgültigen, in dieser Arbeit entstandenen Dashboards können dem Anhang \ref{att:dashboards} entnommen werden.


%\section{Vorstellung der Ergebnisse}

%\section{Rückschlüsse für weitere Entwicklung}
%
%\subsection{Rechtliche Aspekte?}
%\label{sub:recht}
%
%%anonymisierung, datenschutz 
%
%%\subsubsection{Datenschutz}
%%\subsubsection{Anonymisierung}
%
%%als sub hier rein?
%\subsection{Sonstige Aspekte?}
%
%\subsubsection{Incremental Load?}
%\label{sub:incremental}
%\subsubsection{Internationalisierung}
%\subsubsection{Auslieferungsprozess?}
%
%
%% Aspekte vor Erstellung??
%
%
%%\section{Aufgetretene Probleme?}
%
%%\section{Erfüllte Anforderungen}
%%\subsection{Nutzeranforderungen}
%%\subsection{Theoriebasierte Anforderungen}

\section{Ausblick}
Die Arbeit hat einen Grundstein gelegt, um aus den zuvor nicht genutzten Daten einen Mehrwert zu erhalten.
Ein Export von relevanten Daten in einem geeigneten Format und die erste Überführung in ein effizientes Datenmodell waren die ersten Schritte in der Richtung \glqq Big Data Analytics\grqq {}.

Da Big Data laut Literatur \cite{Fischer.2014, Hausler.2018} in der Regel unstrukturierte Daten sind, und die für diese Dashboards zugrundeliegende Daten größtenteils strukturiert sind, ist es zum jetzigen Zeitpunkt kein handfestes \glqq Big Data Analytics\grqq{} (siehe  S. \pageref{fig:interrelations}, Abbildung \ref{fig:interrelations}).
Dennoch ist es ein erster Schritt in die entsprechende Richtung und kann in Zukunft weiter ausgeführt werden.

Es ist ein Reporting nach \gls{BI}-Praktiken entstanden, welches Daten aus unmittelbarer Vergangenheit\footnote{Nach Upload einer Mission auf den \gls{ANALYSE}-Server ist eine Aktualisierung der Visualisierungen binnen 5-10 Sekunden möglich (Bei einem Datenbestand von $\approx$ 10.000 Einsätzen; ungefähr linear skalierend)} in den erarbeiteten Dashboards anzeigt.
Bei dem späteren produktiven Einsatz bei Kunden wird es auch die professionelle Variante geben mit der Möglichkeit des Self-Service-Reporting.
Hierbei können die Anwender, statt der vorgefertigten Dashboards, eigene Diagramme und Dashboards auf Basis des hier erarbeiteten Datenmodells und \gls{DWH} zusammenstellen.

Bis zu diesem produktiven Einsatz beim Kunden sind noch ein paar Schritte seitens der Entwicklung notwendig.
Darunter fällt beispielsweise die Internationalisierung der Dashboards, sodass ein einfaches Wechseln der Sprache ermöglicht wird.
Ebenso ein \glqq incremental load\grqq{} ist eine sinnvolle Erweiterung, da bei einem beispielsweise nächtlichem Laden der Daten in der Regel der größte Teil des Datensatzes (>99\%)\footnote{Bei einem Datenbestand von einem Jahr kommt demnach im Durchschnitt an einem Tag 
$\frac{1}{365} \cdot100 \approx$ 0,27\% neue Daten hinzu. Mit jedem Tag wird dieser Anteil geringer.}  unverändert bleibt.
%Es kommen in den meisten Fällen lediglich ein paar Einsätze hinzu, und die bereits vorhandenen Einsätze 
Des Weiteren sind diverse rechtliche Aspekte zu klären.
So sollten beispielsweise die Zugriffsrechte aufgrund von Datenschutz bei unterschiedlichen Dashboards verwaltet werden, da mehrere Nutzergruppen Zugang zu einem \gls{ANALYSE}-Server haben können.
Dies kann voraussichtlich, je nach zukünftigem Lizenzierungsmodell, in der \gls{QMC}-Oberfläche unternommen werden.

In Zukunft ist es denkbar, neue Technologien, Methodiken und Ansätze des Bereichs \glqq Data Analytics\grqq{} anzuwenden.
Spannend ist hierbei vor allem der Bereich \glqq Artificial Intelligence\grqq{} (AI) beziehungsweise künstliche Intelligenz (KI).
Mit Algorithmen wie \glqq Machine Learning\grqq{} oder \glqq Deep Learning\grqq{} sind gänzlich neue Einblicke, Erkenntnisse Trends oder gar Vorhersagen möglich.
Auch das weitere Erkennen und Herausfiltern von Testeinsätzen wäre eine sinnvolle Aufgabe hierfür.

Die Basis durch die nun exportierbaren Daten in einem adäquaten Format ist gelegt.
Die theoretischen Algorithmen und entsprechende Bibliotheken zum Programmieren werden immer ausgereifter und bilden so im Einklang mit den technischen und medizinischen Daten die besten Voraussetzungen, bisher unentdeckte Erkenntnisse zu gewinnen, die möglicherweise schon bald viele Leben retten könnten.


%\subsection{Mehrwert}
