

\title{ {\large  Exposé zur Bachelorthesis} \\
\vspace{1cm} 
{\huge \bfseries Big Data Analytics mit Medizintechnik im Rettungsdienst} \\
\vspace{1cm} {\small Betreut durch: } \\
{\large Prof. Jan-Torsten Milde} \\
{\large Christoph Graumann, M.Sc.} \\
\vspace{1cm} {\large Hochschule Fulda} \\
{\small Digitale Medien} \\
}

\author{ Joshua Hirsch }
\date{\today}

\documentclass[12pt]{article}
% \usepackage[latin1]{inputenc}
\usepackage{breakcites}
\usepackage{microtype}
\begin{document}
\maketitle \clearpage

\tableofcontents \clearpage

\section{Problemstellung}\label{problem}
Ein Einsatz im Rettungsdienst ist jedes Mal ein gänzlich neuer Fall. Es kommen
verschiedene Faktoren in unterschiedlicher Gewichtung hinzu und machen einen
solchen Einsatz einzigartig. Dennoch lassen sich Korrelationen feststellen,
welche in einigen Einsätzen zu gleichen oder ähnlichen Ereignissen führen.

Die heutzutage eingesetzten Geräte in diesen Bereichen besitzen viel Technik und
Möglichkeiten der Datensammlung und -haltung. Diese Daten müssen, wegen
rechtlicher Aspekte zumindest in Deutschland, permanent abgespeichert werden.
Jedoch liegen diese meistens im Anschluss der Persistierung ungenutzt auf einem
Datenträger oder Server. \\
Dabei verbergen sich in diesen Daten neue Erkenntnisse und unzählige Antworten
auf Fragen, welche sich die Leiter entsprechender Einrichtungen jährlich,
monatlich oder gar täglich stellen, und zum jetzigen Zeitpunkt keine adäquate,
schnelle und simple Möglichkeit zur Beantwortung jener zur Verfügung haben.

Mit dieser Arbeit sollen genau diese ungenutzten Daten so aufbereitet werden,
damit sie für eine grafische Auswertung sinnvoll sind und Rettungswachen,
Krankenhäusern oder anderen Einrichtungen einen deutlichen Mehrwert bieten.


% \subsection{Forschungsstand}\label{previous work} A much longer \LaTeXe{}
% example was written by Gil~\cite{Gil:02}.

\subsection{Erkenntnisinteresse}\label{erkenntnis}
Mit der grafischen Auswertung von vielen Einsätzen über längere Zeiträume mit
unterschiedlichen Geräten sollen bereits vermutete Fragestellungen be\-stätigt
oder widerlegt werden können.
Fragen wie "Wird tagsüber besser reanimiert als nachts? Wie steht dies im
Verhältnis zur Einsatzdichte der jeweiligen Schichten?" sind nur einige der
vielen Fragen, die derzeit im Raum stehen und nur schwerlich mit einer Antwort
ausgestattet werden können.
Auch sehr forschungsnahe Fragen, beispielsweise ob der Anstieg des Blutdrucks in
Kombination mit der Senkung der Sauerstoffsättigung zu einer Apnoe führt, sind
denkbar spannend.

Die Daten sollen entsprechend vorliegen, damit die zu erwartenden
Fragen in adäquater Form und Zeit beantwortet werden können.
Hierbei soll auch die Möglichkeit der Überprüfung gewährleistet werden, dass entsprechende
gesetzliche Richtlinien \cite{Maconochie.2015} oder lokale Vorgaben eingehalten
werden oder die Stärken und Schwächen von Menschen und Geräten identifiziert werden können. Auch
Prognosen für die Zukunft sollen möglich gemacht werden oder gar neue Forschungsfragen gefunden  
und im besten Fall gleich beantwortet werden können.

Eine gute Möglichkeit um diese Daten übersichtlich zu visualisieren sind
sogenannte "Dashboards", ein englischer Begriff welcher "Armaturenbrett"
bedeutet. Dieser hat sich in der digitalen Welt als Schlagwort etabliert, eine
Übersicht über viele verschiedene Informationen und Daten zu liefern, so wie es
ein Armaturenbrett im Auto vollbringt.

% \paragraph{Outline} The remainder

\section{Fragestellung}\label{fragen}
Aus der oben genannten Problemstellung und Erkenntnisinteresse ergeben sich
folgende Fragen:
\begin{description}
\item[Fragestellungen]~\par
\begin{itemize}
      \item Welche Fragen haben unsere Anwender, die wir mit unseren Daten beantworten können?    
      \item Wie müssen wir Dashboards gestalten, damit diese Fragen beantwortet werden?
      \begin{itemize}
        \item Wie müssen die Dashboards entworfen werden, damit sie medizinisch korrekt sind?  
      \end{itemize}
      \item Was müssen wir in unserem Datenmodell beachten, damit wir diese Dashboards erstellen können?
      \begin{itemize}
        \item Was muss bei der Schnittstelle beachtet werden?
      \end{itemize}
\end{itemize}
\end{description}

\subsection{Zielsetzung}\label{ziel}
Ziel ist es, so viele Fragestellungen wie möglich der entsprechenden Anwender,
wie z.B. Rettungswachenleiter, Ärztlicher Leiter,
Qualitätsmanagement\-beauftragte u.a. herauszufinden und zu konkretisieren.
Anschließend sollen die erhobenen Fragen bezüglich der benötigten Daten und
deren Format analysiert werden. Daraufhin folgt die Konzipierung von Dashboards,
welche so viele Fragestellungen wie möglich beantworten sollen. Erste Ent\-würfe
und letztendlich präsentierfähige Dashboards sollen das visuelle Ergebnis dieser
Arbeit werden. \\
Simultan werden geeignete Datenmodelle zur reibungslosen Darstellung sowie
Anforderungen an die Schnittstelle erarbeitet.

% \subsection{Theoriebezug}\label{previous work}
% A much longer \LaTeXe{} example was written by Gil~\cite{Gil:02}.


\section{Abgrenzung} \label{abgrenzung}
Im Rahmen der Bachelorarbeit ist es nicht notwendig, die entstandenen Dashboards
in die derzeit laufende Software zu implementieren. Handlungs\-empfehlungen für
die entsprechenden Entwickler sind hierbei ausreichend.

Des Weiteren ist keine produktive Anbindung an die entsprechende Schnitt\-stelle
vorgesehen. Etwaige Rückschlüsse oder das Testen von Technologien sind hierbei
zureichend.

\section{Methodik}\label{methodik}
Die Erhebung der Fragestellungen wird voraussichtlich komplett durch Interviews
stattfinden. Dabei werden vorerst die internen Mitarbeiter gefragt, welche
Fragestellungen sie für sinnvoll erachten. Dabei wird darauf geachtet, dass die
befragten Personen einen engen Bezug zum Rettungsdienst oder zu Kunden, bzw.
Mitarbeitern in dieser Branche haben. Das Unternehmen liefert hierbei eine Menge
in Frage kommender Stakeholder, da die Quote der ehemaligen und noch aktiven
Rettungsdienstmitarbeiter überdurchschnittlich hoch ist. 

Des Weiteren werden Interviews oder Gespräche mit Key-Opinion-Leaders dieser
Thematik angestrebt, um Einschätzungen und Erkenntnisse aus professioneller,
erster Hand zu gewinnen. \\
Im Anschluss der Erhebung werden die ermittelten Fragestellungen analysiert,
hierbei wird untersucht:
\begin{itemize}
      \item Welche Daten für jene Frage benötigt werden
      \item Liegen die benötigten Daten vor
      \begin{itemize}
        \item Wenn ja, sind die Daten in dem richtigen Format
      \end{itemize}
      \item Wie relevant, bzw. interessant die Frage allgemein ist
\end{itemize}
In Bezug darauf werden exemplarisch Dashboards für die Fragestellungen
entworfen, bei welchen die benötigten Daten im richtigen Format vorliegen.
Danach werden diese vorgestellt und von entsprechenden Mitarbeitern validiert
und evaluiert. Parallel dazu werden für die Fragestellungen, welche die Daten
nicht im richtigen Format vorliegen haben, entsprechende Datenmodelle oder
Datenformate erarbeitet und Hinweise an die Softwareabteilung bezüglich der
Schnittstelle kommuniziert. \\
Anschließend beginnt der gesamte Prozess mit den herausgefundenen
Validierungsergebnissen erneut. Dies passiert in einem zyklisch-iterativen
Verfahren, gemäß etablierter Prozesse des Requirements Engineering.
\cite{Pohl.2010}

Nach den ersten Zyklen, sobald die ersten Dashboards und Metriken intern
validiert wurden, werden diese entsprechend interessierten Kunden vorgestellt.
Die Erkenntnisse hierbei werden besonders in die Evaluierung und erneute
Spezifizierung einfließen, da die Erfahrungen und der Wissensstand dieser
Personen von enormer Wichtigkeit sind.

\subsection{Technologien}\label{tech}
Für das Erstellen der Dashboards wird ein Business-Intelligence-Werkzeug namens
"Qlik Sense" eingesetzt. Hierbei handelt es sich um eine leistungs\-starke
Software, welche auch z.B. das Laden der Daten oder Umstrukturieren der
geladenen Daten per Skript erlaubt. Im Unternehmen gibt es Lizenzen für die
Enterprise Variante, welche auch für den späteren produktiven Einsatz bei Kunden
zum Tragen kommen wird.

\sloppypar{
Wie in Kapitel \ref{abgrenzung} beschrieben werden sich die Entwickler mit der
Schnitt\-stelle beschäftigen, sodass die Daten vom entsprechenden Server im
geforderten Format vorliegen. Eine mögliche Evaluation meinerseits von
Technologien, welche für die Schnittstelle hilfreich sein könnten, ist optional.
}


% Hierbei wäre derzeit eine zu evaluierende Technologie beispielsweise die
% Programmiersprache "GO" für eine eventuelle Proxy-REST-Schnittstelle.
% Allerdings stünde die Beurteilung der Notwendigkeit hierbei voran.

\subsubsection{Daten}
Das Erarbeiten der benötigten Datenmodelle wird vorerst ohne produktive Daten
erfolgen. Hierbei ist das Ziel, eine grundlegende Struktur abzubilden, welche
eine breite und nutzbare Basis von Daten schaffen soll. Diese wird mithilfe von
visuellen Datenmodellen ebenfalls in der im Kapitel \ref{tech} genannten
Software dargestellt.

Die Daten der Geräte liegen größtenteils im eigenen "corpuls"-Format vor.
Hierbei finden sich auch sogenannte "Events", welche bestimmte Ereignisse im
Laufe einer Mission mit Parametern beschreiben, wie z.B. eine Blutdruckmessung
mit dem systolischen und diastolischen Druck als Parameter. Somit ist ein
einzelner Einsatz in der Auswertung als abgeschlossener atomarer Datensatz
anzusehen, wobei es genau genommen eine definierte Zeitspanne mit \textit{n}
Ereignissen und Messungen ist. 

Damit ein solches mehrdimensionales Datenpaket derzeit zur einfachen manuellen
Auswertung geeignet ist, besteht die Notwendigkeit einer gewissen internen
Vorverarbeitung dieser Daten. Dies wird zurzeit unter anderem mit sogenannten
"MissionMarker" abgebildet. Diese bilden Aggregationen von bestimmten Events,
sodass beispielsweise die Anzahl an Blutdruckmessungen zusammengefasst wird. Mit
diesen Aggregationen lassen sich bereits viele Auswertungen und Erkenntnisse
finden.

Jedoch ist für eine tiefere Analyse eine komplexere Datenstruktur notwendig,
sodass die eigentlich gebotene Mehrdimensionalität zur Verfügung stehen sollte.
Die konfliktfreie und sinnvolle Modellierung von mehreren solcher
mehrdimensionalen Daten ist eine der großen Herausforderungen der
Bachelorarbeit.


% \subsection{Material}\label{previous work}
% A much longer \LaTeXe{} example was written by Gil~\cite{Gil:02}.
\clearpage
\section{Gliederung}\label{gliederung}
Eine provisorische inhaltliche Gliederung der Thesis sieht derzeit wie folgt
aus: 
\begin{enumerate}
  \item Einleitung
  \begin{itemize}
      \item Vorstellung Firma; Produkte; corpuls.web; ANALYSE
      \item Business Intelligence (im Bereich Rettungsdienst), QLIK  
      \item Abgrenzung      
  \end{itemize}
  \item Stand der Technik
  \begin{itemize}
      \item corpuls.web ANALYSE
      \item Daten
      \item Qlik
  \end{itemize}        
   \item Vorgehensbeschreibung
   \begin{itemize}
      \item Festlegung der Nutzergruppen
   \end{itemize}
   \item iterative Durchführung
   \begin{itemize}
      \item Befragung/Ermittlung
      \item Wie sehen die Dashboards zur Beantwortung der Fragen aus
      \item Welches Format müssen die Daten haben
      \item Validierung/Evaluation
  \end{itemize}
    \item Zusammenfassung
    \begin{itemize}
      \item  Vorstellung Ergebnisse (Qlik-Apps, Dashboards, Datenmodelle)
      \item Rückschlüsse / Handlungsempfehlungen für Schnittstelle        
  \end{itemize}
    \item Fazit
    \begin{itemize}
      \item Probleme
      \item Erfüllte und nicht erfüllte Anforderungen
  \end{itemize}
\end{enumerate}

% \subsection{Zeitplan}\label{conclusions}
% 04.01.2019: Beginn der Bachelorarbeit \\


\bibliographystyle{apalike}
\bibliography{mybib}

\end{document}