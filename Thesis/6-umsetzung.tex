\chapter{Umsetzung} %Should be roughly 18pages
\label{kap:umsetzung}
\minitoc\pagebreak
% Erst JIRA Stories dann Datenmodell oder andersrum?
% Benötigtes Modell, dann stories, dann umsetzung
% Added new Data models like cpr, shocks, nibp!! as subsections!

\section{Technische Aspekte}
\label{sec:tech}
Zu Beginn der Umsetzung gilt es vorab diverse technische Aspekte zu analysieren und im Laufe der Entwicklung zu berücksichtigen.
Hierzu gehören beispielsweise die Schnittstelle und der daraus resultierende Export der Daten oder die Haltung der Daten in einer Datenbank.
Dies ist eine essentielle Basis, damit eine effiziente und angemessene Datengrundlage für die anschließend entwickelten Dashboards und Auswertungen zur Verfügung steht.

Dabei ist es bezüglich der Flexibilität und Unabhängigkeit von Vorteil , dass die Daten und der entsprechende Export nicht ausschließlich für die hier verwendete Technologie Qlik Sense abgestimmt und entwickelt werden soll.
Auch andere Software-Lösungen, Drittanbieterprogramme oder Business-Intelligence-Werkzeuge sollen die Daten und den neuen Export von \gls{ANALYSE} zur Auswertung verwenden können, damit eine gewisse Freiheit und keine absolute Abhängigkeit an einer Technologie entsteht.

\subsection{Schnittstelle \acrlong*{ANALYSE} und Qlik Sense} %StandDerTechnik??
\label{sub:schnittstelle}

\subsection{Export der Daten} %subsub?
\label{sub:export}

\subsubsection{JIRA-Stories/Anforderungen als User Stories}
\label{subsub:stories}

\subsubsection{Aktuelles Format}
\label{subsub:aktuellesformat}
json format cpr (siehe mails) (Fotos whiteboard als anhang?),
null values bei cpr trends
%\subsection{Incremental Load?}
\subsection{Datenbankhaltung?}
\subsubsection{Lasttests?}
%dummydaten?
%\subsection{JIRA-Stories/Anforderungen als User Stories}

%\section{Datenmodell?}

\section{Erstellung der Qlik-App(s)}
\label{sec:erstellung}
%subsUnterschiedliche Apps?
\subsection{ETL-Prozess}
\label{sub:etl}

\subsubsection{Ladeskripte}
\label{subsub:scripts}

\subsubsection{Demo- und Testeinsätze herausfiltern}
\label{subsub:testfilter}

\subsubsection{Weitere manuell hinzugefügte Felder (ReaStart, Calendar, Test...)}
\label{subsub:weitereFelder}


\subsection{Datenmodell (hier oder eigenes section?)}
\label{sub:datenmodell}

\subsection{Dimensionen}
\subsection{Kennzahlen}
\subsection{Verwendung von Erweiterungen?}
\subsection{Dashboards}
\subsubsection{Neue mögliche Visualisierungen durch Anforderungen Schocks, Nibp, Cpr}
test
\subsection{(Auszug) Umsetzung der Evaluierungsergebnisse}
\subsubsection{Zielgruppenunterschiedliche Startseiten}
\subsubsection{Lesezeichen?} %vlt so 10 Beispiel Lesezeichen
\subsubsection{Usability/Nutzerführung/Hilfetexte}
\subsubsection{Reduzierung Inhalt pro Arbeitsblatt}
\subsubsection{Weitere}
%subs testeinsätze filtern

\subsection{Einstellungen der Arbeitsblätter und Diagramme}
% (wie zB. Farben bei Auswahl beibehalten)
%CustomThemes?
%subs internationalisierung

% Aspekte vor Erstellung??
\section{Rechtliche Aspekte?}
\subsection{Datenschutz}
\subsection{Anonymisierung}


\section{Sonstige Aspekte?}
\subsection{Auslieferungsprozess?}
\subsection{Internationalisierung}
\subsection{Incremental Load?}


\section{Evaluierung der Ergebnisse?}
%subs Usability-Tests?
