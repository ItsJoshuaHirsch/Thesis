\chapter{Konzeption}
\label{konzept}

\minitoc\pagebreak
%\lipsum[1-60]

\section{Vorgehen bei der Konzeption}

%\bildrechts{iterative} {2cm} {it} {iter}

Gemäß der Methoden des Requirements Engineering \ref{require} wird anhand der erhobenen und ausgearbeiteten Fragestellungen bzw. Anforderungen aus \ref{4.4 fragestellungen} ein Konzept erarbeitet. 
Auch hierbei wurde eine iterative Herangehensweise gewählt, unter anderem um in einer kurzen Zeit grobe Fehler zu erkennen und viele Verbesserungen herauszufinden. 
Auch wird hierdurch gewährleistet, dass die exemplarischen und später produktiven Auswertungen und Visualisierungen nicht gänzlich in eine falsche Richtung gehen, sondern von Anfang an etwaige Missverständnisse aufgedeckt werden oder grundlegende Entscheidungen frühzeitig überdacht werden können.

Um von vornherein ein realistisches "`Look-and-Feel"' zu bieten und die entsprechend geplante Technologie mit ihren Funktionen und der Nutzerführung zu evaluieren, wird beim Prototyp bereits Qlik Sense \ref{technologieqlik} verwendet. 
Ein weiterer Vorteil ist, dass das notwendige Einarbeiten für die spätere Umsetzung in diese Technologie bereits gewährleistet wird. 
Dabei können auch vorab verschiedene Vor- und Nachteile der Technologie herausgefunden werden und dementsprechend können diese bei der weiteren Planung und Umsetzung berücksichtigt werden, was viel Zeit sparen kann.

Des Weiteren wurden Dummy-Daten unter anderem für den Prototypen generiert. 
Sie waren auch notwendig, um entsprechend andere Tests für die Datenbank und Auslastung durchzuführen \ref{db und lasttest}. 
Dabei wurde ein Python-Skript geschrieben, welches einige Daten zufällig generiert, jedoch auf einer realistischen Basis.

Die Evaluation soll wie oben beschrieben iterativ sein, was mehrere Termine mit unterschiedlichen Experten zur Konsequenz hat. 
Im Zuge dieser Bachelorarbeit wird die Evaluation größtenteils? mit Mitarbeitern der Firma \gls{GS} durchgeführt. 
Es gibt (viele) Abteilungen, in welchen genügend Fachleute sitzen, die in der Lage sind diese Dashboards kritisch prüfen zu können. 
Hierbei wird darauf geachtet, unterschiedliches Personal miteinzubeziehen, sodass es keine exklusive/alleinige Evaluierung durch die Abteilung medizinische Forschung ist, sondern auch Vertriebsmitarbeiter oder andere Stakeholder sollen den Prototyp kritisch untersuchen.

Bei der Konzeption ist auch das Datenmodell ein wichtiger Bestandteil. 
Hier können verschiedene Modelle ausprobiert werden und gegebenenfalls Änderungen vorgenommen werden, sollten sich bei der Evaluation Probleme oder andere Anforderungen herausstellen.


\section{Erstellung eines Prototypen}
\subsection{Verwendete Technologie}
Für den ersten Prototypen wurde sogleich die letztendlich umsetzende Technologie Qlik Sense \ref{qlik} verwendet. 
Der Einfachheit halber beschränkte sich dies zu Anfang auf Qlik Sense Desktop, einerseits aufgrund der vereinfachten Bedienung, hauptsächlich jedoch da der entsprechende Server mit der Qlik Sense Server Variante zu Anfang der Konzeption noch nicht eingerichtet war. 
Ein Umzug von Qlik Sense Desktop zu Server stellt keine größeren Probleme dar, man kann die erstellten Apps bidirektional ex- und importieren. 
Lediglich die Datenverbindungen, welche für die jeweiligen Applikationen die Daten aus einer Quelle beziehen, werden nicht fehlerlos übertragen.

\subsection{Datengrundlage}
Zu Beginn wurde überlegt, welche Daten die Grundlage für den Prototypen bilden sollte.
Dabei wurde auf die Demo-Datenbank des internen\cweba - Servers zurückgegriffen, welche zu diesem Zeitpunkt in etwa 100.000 Missionen enthielt, von denen jedoch grundsätzlich alle nur Test-Missionen sind. 
Dies sind solche, die entweder von der internen Testabteilung oder der Software-Entwickler beim Testen von Funktionen oder Herausfinden von Problemen am Gerät oder durch die Software erzeugt werden. 
Dementsprechend enthalten sie wenige spannende Daten, die auch selten realistisch sind und etwaige neue Erkenntnisse nicht darlegen können. 
Nichtsdestotrotz bilden sie eine gute Grundlage um die ersten Basis-Auswertungen von Missionen paradigmatisch darzustellen. 

Für diesen Zweck kann die Export-Funktion im CSV-Format von Analyse genutzt werden, welche in  \ref{istzustandanalysezurauswertung} näher beschrieben ist. 
So war es für den Anfang möglich, einen relativ großen Datensatz mit mäßig sinnvollen Daten zu erhalten.\footnote{(In späteren ?Szenarios\ref{lasttests,db} wurden eigens generierte Missionen verwendet, auf die Erzeugung dieser wird in \ref{dummydaten} genauer eingegangen.)}



\subsection{Erstellung exemplarischer Dashboards} 
Mit der Datengrundlage und der Technologie Qlik Sense Desktop können nun die ersten beispielhaften Dashboards entwickelt werden.
Die Basis für die verschiedenen Diagramme und Auswertungen bilden die iterativ erhobenen Fragestellungen aus Kap \ref{itErhebFrage}. 
Der Ansatz hierbei ist, mit einer grafischen Darstellung der Daten so viele Fragestellungen wie möglich beantworten zu können.
Dabei werden für die Art und Weise der Darstellungen verschiedene Aspekte berücksichtigt, wie zum Beispiel ob es ein zeitlicher Verlauf ist oder tendenziell eine Momentaufnahme, absolute gegen relative Kennzahlen, Veränderungen oder Trends und vieles mehr.
Auf Basis dieser Aspekte wird eine möglichst passende Darstellungsform gewählt und mit entsprechender Dimension und Kennzahl, eventuell auch mehrere Dimensionen und/oder Kennzahlen, gefüllt.

Eine sinnvolle Gruppierung oder Aufteilung der entsprechenden Arbeitsblätter oder Diagramme ist im Zuge des Prototyps von keiner hoher Priorität. 
Im Fokus steht die erstmalige Beantwortung möglichst vieler Fragestellungen, auch gegebenenfalls auf verschiedenen Wegen mit alternativen Darstellungsformen.
Nichtsdestotrotz wird später, zum Zeitpunkt der Evaluierung, eine einigermaßen angebrachte Gruppierung von Diagrammen und vernünftige Reihenfolge der Arbeitsblätter vorgenommen, damit für die entsprechenden Personen eine Struktur erkennbar ist, um mögliche Verwirrungen zu vermeiden.


\subsubsection{Datenladeskript}
, Was ist rel, was test, ....
\subsubsection{Datenmodell}
recht simpel. im prinzip eine Tabelle, keine Mehrdim, ...
% Vermutlich nur Umsetzung, vielleicht auch hier

%subsubsection Dashboards, und dann unterpunkte?
\subsubsection{Startseite}
\bildbreit
{Uebersicht_Prototyp}
{Startseite, Übersicht}
{Überblicks}

Die erste Seite, die der Kunde zu sehen bekommt wenn er die Software startet, soll ihm einen groben Überblick verschaffen. 
Es ist wie die Startseite einer Website, wo man die wichtigsten Informationen unmittelbar auf der ersten Seite finden kann.
Die Abbildung \ref{fig:Uebersicht_Prototyp} zeigt die erste Version dieser Startseite.

Sie wurde recht simpel mit vier Diagrammen und zwei Kennzahlen gefüllt, damit der erste Eindruck nicht von einer Informationsflut negativ beeinflusst wird.
Sollte der Nutzer weiterführende Informationen und tiefgreifendere Analysen durchführen wollen, kann er diesen Ansprüchen auf den folgenden Dashboards gerecht werden.
%\bildrechts{waterfall} {4cm} {w} {wa}
Das Wasserfall-Diagramm ganz links zeigt die absolute Anzahl an Einsätzen, die all seine Geräte durchgeführt haben.
Des Weiteren werden die absoluten Zahlen der verschiedenen Einsatzarten in Relation zur Gesamtmenge grafisch dargestellt.
Hierbei wird unter anderem zwischen Testeinsätzen, Reanimationen oder sonstigen Einsätzen unterschieden.
Bei den Reanimationen gibt es drei weitere Unterarten: (bulletlist?) Mit \& Ohne Feedbacksensor oder eine mechanische Reanimation.
Die Farbgebung soll Teilsummen von Gesamtmengen unterscheiden.
Ziel der Darstellung ist ein visueller Eindruck, wie viele Einsätze es gibt und welche Arten von Einsätzen in welchem Maße vorkommen.

Das Liniendiagramm stellt den zeitlichen Verlauf der Einsätze dar.
Als (zeitliche) Dimension wurde hier Jahr-Monat gewählt, eine Alternative wäre die tatsächliche Datumsangabe, dies würde jedoch zu einer Kurve führen, welche viele Zacken enthält (Vergleichsbild?) und damit sehr unruhig wirkt.
Die Aufsummierung in Monate, alternativ auch Wochen, bringt je nach Datenlage eine recht glatte Kurve mit sich. 
Die Kennzahlen dieser Visualisierung sind die Summe der relevanten Einsätze und zum Vergleich die Summe der Reanimationen. 
Eine Mini-Legende unterhalb der Grafik erleichtert die Orientierung, sollte der Nutzer in einen bestimmten Zeitraum "`reinzoomen"'.

%Es wurde die AutoCalendar-Funktion genutzt, damit ein fließender Übergang zwischen den Jahren möglich ist.

Darunter sind zwei Kreisdiagramme zu sehen, welche die Anzahl der Einsätze und der Reanimationen pro Gerätetyp anzeigen.
Diese Art der Visualisierung wurde gewählt, damit die Relation und Verteilung unter den Geräten deutlich gemacht wird.
Unter dem jeweiligen Kreisdiagramm ist als einzelner \gls{Key-Performance-Indicator} die \gls{KPI} die durchschnittliche Einsatz- und Reanimationsdauer.

% Filter?
\subsubsection{Einsatzzeitpunkt}
\bildbreit
{einsatzzeitpunkt}
{Dashboard zu den Einsatzzeitpunkten}
{Einsatzzeitpunkt-Visualisierung}

Es gab diverse Fragestellungen zu den Einsatzzeitpunkten. (Beispiele oder ref?)
Dieses Arbeitsblatt, in Abbildung \ref{fig:einsatzzeitpunkt} zu sehen, soll einen Überblick geben, zu welchen Tages- und Uhrzeiten die Einsätze stattfinden.
Hierbei gibt es drei verschiedene Detailstufen?: Tageszeit, Stündlich und Halbstündlich (bullet?). 
So kann der entsprechende Nutzer für seinen gewollte Bedarf die jeweilige Auslastung oder Einsatzhäufigkeit herausfinden.
Des Weiteren sind unten, unterhalb der Tageszyklen (Tageszeit und Stunden), zwei weitere Diagramme, welche die Anzahl der Einsätze und Reanimationen für Wochentage und Monate anzeigt. 
Somit ist jeder relevante zeitliche Zyklus abgedeckt und es können beispielsweise zu Tageszeiten wie nachts, Wochentage wie Wochenende und/oder saisonale wie winterliche Auswertungen betrieben werden.

Es wurden zur Darstellung Kombi-Diagramme gewählt, um normale Einsätze und Reanimationen getrennt, aber dennoch in Relation betrachten zu können.
Die Balken eignen sich gut, um das Volumen eines Zeitpunktes mit anderen leicht Vergleichen zu können und repräsentieren die Menge an relevanten Einsätzen.
Die Anzahl der Reanimationen wurde als Linie mit einer alternativen Y-Achsen-Skalierung visualisiert, damit auch geringe Vorkommen, wie in der Praxis häufig der Fall, noch gut sichtbar sind, da die obere und untere Grenze unabhängig der Anzahl aller Einsätze ist.
Würde man diese als Balken neben die normalen Einsätze legen, wären sie kaum zu sehen und Unterschiede zwischen den jeweiligen Zeitpunkten wären nur schwer zu erkennen.
Mit der alternativen Liniendarstellung ist die untere Kante der X-Achse nicht immer 0, wie es bei der Darstellung von Balken der Fall ist, sondern kann beim Minimum der entsprechenden Kennzahl beginnen.
%Timebuckets wurden selber geskriptet, damit nicht zu fein nach minute ?

\subsubsection{Einsatzdauer}
\bildbreit
{einsatzdauer}
{Dashboard zu der Dauer von Einsätzen}
{Einsatzdauer-Visualisierung}

Die Dauer von Einsätzen ist ebenfalls eine gefragte Information. (ref?)
Hierbei sind Fragen wie 
Je nach entsprechenden Vorgaben können hierbei unterschiedliche Auswertungen durchgeführt werden.
Sollen die Rettungskräfte beispielsweise das Gerät immer sofort beim Losfahren starten, können gegebenenfalls Auswertungen zu den Fahrtzeiten getroffen werden.
Oder wenn das Personal das Gerät erst vor Ort anschalten soll, kann die tatsächliche Einsatzzeit verglichen werden.
Es kommt also auf die Vorgaben der entsprechenden Instanzen drauf an, welche Analysen durchgeführt werden können.

Als generelles Modell werden auf diesem Dashboard, in Abbildung \ref{fig:einsatzdauer} zu sehen, die grundlegenden Informationen zur Dauer dargestellt.
Dazu zählt die globale durchschnittliche Missionsdauer, sowie die absolute Gesamteinsatzdauer, welches unter anderem eine durch die Bundeswehr gefragte \gls{KPI} ist.
Des Weiteren gibt es die mittlere Einsatzdauer nach Gerätetyp aufgeschlüsselt. 
Dies ist zugehörig unter der Geräteübergreifenden \gls{KPI} als Balkendiagramm dargestellt, damit die Werte gut miteinander vergleichbar sind.


Darunter sind zwei weitere Diagramme zu finden, welche als Histogramm (irgendwo erklären) die Häufigkeit verschiedener Einsatz- sowie \gls{CPR}-Dauer darstellen.
%\bildrechts{Dauer-klasse} {5cm} {w} {wa}
Dabei werden je nach maximaler Dauer dynamisch gleichgroße Bereiche oder ''Klassen'' definiert.
So gibt es beispielsweise den Bereich ''0 <= Reanimationsdauer in Min. < 16'', welcher hier >4000 Einsätze zählt, danach Reanimationsdauer größer gleich 16 und kleiner als 32 mit ca. 400 Missionen.
Dies wird in diesen 16-Minuten-Schritten weiter bis zur maximal vorhandenen Reanimationsdauer fortgeführt.
Dabei wird die Verteilung von Einsätzen und Reanimationen mit einer entsprechenden Dauer sichtbar und es können weiterführende Analysen auf Basis der Einsatzdauer vorgenommen werden.

\subsubsection{Geräte}
\bildbreit
{devices}
{Dashboard zu den entsprechend vorhandenen Geräten}
{Geräte-Visualisierung}

Das folgende in Abbildung \ref{fig:devices} präsentierte Arbeitsblatt gibt dem Nutzer eine Übersicht seiner eingesetzten Geräte.
Dabei gibt es die eindeutige \gls{KPI} ''Anzahl Geräte'', welche die distinkten Geräte zählt, von denen bis dato Missionen in dem vorliegenden\cweba - Server vorliegen.
Sofern der entsprechende Betreiber den Upload oder das nachträgliche importieren der Einsätze auf den Server anordnet, kann hiermit die gesamte Anzahl an Geräten überwacht werden. (Aktive Geräte? bsp Einsatz in letztem quartal? In umsetzung aufnehmen)
Ergänzend zu dieser (pauschalen) Zahl gibt es in intuitiver Leserichtung rechts eine Visualisierung, wie viele Geräte einer Art vorliegen.
Die Art der Darstellung ist ein Kreisdiagramm, da in der Regel eine Geräteart überwiegen wird, und somit diese in Relation zu der Gesamtheit gesetzt wird.

Darunterliegend finden sich zwei Balkendiagramme wieder, welche die Anzahl der Einsätze und Reanimationen nach den einzelnen Geräten mittels Seriennummer und/oder Geräte-ID aufschlüsseln.
Dies präsentiert die Einsatzhäufigkeit von den Einzelgeräten und kann dadurch beispielsweise als Grundlage für Schichtplanung oder Neuanschaffung von Geräten dienen.

\subsubsection{Reanimation}
\bildbreit
{reanimation}
{Dashboard zu den Daten des \gls{CPR}-Feedbacksensors}
{Reanimations-Visualisierung}

Die nachträgliche Auswertung einer Reanimation ist für viele Stakeholder eine wichtige Aufgabe.
Dabei gibt es beispielsweise die klassische Nachbesprechung zwischen Notfallsanitäter(erklärung Rettungssani?) und Auszubildenden oder aber auch die medizinische Forschung, welche Ereignisse und Werte einer Reanimationen untersuchen um gegebenfalls neue Erkenntnisse zu gewinnen.
Eine Analyse über viele Reanimationen hinweg ist hierbei für viele Nutzergruppen (außer MRaA) ein neuer Weg, welcher bisher unentdeckte Zusammenhänge offenbaren oder bereits angenommene Hypothesen bestätigen kann.

Ein wichtiger Erfolgsfaktor einer Reanimation ist eine adäquate Drucktiefe, welche zwischen 5-6cm liegen soll. (ref erc?)
Sofern ein \gls{CPR}-Feedbacksensor (erklären?) bei einer Reanimation verwendet wird, kann die Drucktiefe jeder Kompression eingesehen werden. (auch vorOrtFeedback)
Eine Auswertung zur Tiefe über viele Reanimationen hinweg kann im obersten Balkendiagramm in Abbildung \ref{fig:reanimation} betrachtet werden.
Dabei sind horizontal die verschiedenen Drucktiefen und vertikal die Anzahl der Missionen mit dieser Tiefe angeordnet.
Die Farbgebung mit rot und grün soll dem Benutzer helfen, schnell die ''guten'' Kompressionen von den unzureichenden zu unterscheiden.

Ein wichtiger Punkt hierbei ist, dass derzeit pro Reanimation genau ein Durchschnittswert der Drucktiefe gebildet wird und zum Export bereitsteht.
Es ist somit eine stark voraggregierte Information, welche einen eventuellen falschen Wert widerspiegelt. 
So kann beispielsweise eine Reanimation 100 zu flache Kompressionen mit 4cm, sowie 100 zu tiefe Kompressionen  mit 7cm enthalten.
Der nun aggregierte Mittelwert dieser Reanimation lautet 5,5cm und suggeriert eine an der Drucktiefe gemessene, ''perfekte'' Reanimation, obwohl nicht eine korrekte Kompression vorliegt.
Dieser Aspekt muss bei der Betrachtung dieser Auswertung berücksichtigt werden.
Eine mögliche Lösung oder zum Mindesten eine Verbesserung der Aussagekraft zur Drucktiefe wird in Kap blabla beschrieben.
%cpr nur avg, keine einzel cpr, gefordertes neues Modell, wie in kap 123 beschrieben 

Zur rechten Seite der oben beschriebenen Visualisierung ist ein Boxplot-Diagramm, welches eine Zusammenfassung zur Drucktiefe liefert.
Es zeigt kompakt die maximale und minimale durchschnittliche Tiefe, sowie die Quartile und den Median aller Reanimationen an.

%ccf, frequenz, 

\subsubsection{Schocks}
\bildbreit
{shocks}
{Dashboard zu den abgegebenen Defibrillationen}
{Schocks-Visualisierung}

%schocks die nicht 200j hatten, leider nur max schock, keine einzel shcocks, gefordertes neues Modell, wie in kap 123 beschrieben 
gh

\subsubsection{Sensorik}
sd
\subsubsection{Blutdruck}

%nibd, leider nur min max avg, keiine einzel messungen, gefordertes neues Modell, wie in kap 123 beschrieben 
sd

\subsubsection{Patientendaten}
sd
\subsubsection{Pacer?Sonstiges}
sd


\section{Generierung von Dummy-Daten?}
% Kap. Umsetzung?

\section{Evaluierung  des Prototyps}
\subsection{Vorgehen bei der Prototyp-Evaluierung}
Zur Evaluierung des Prototyps werden vorerst, wie in \ref{vorgehenkonzept} beschrieben, firmenintern geeignete Personen gesucht. 
Dabei wurden Mitarbeiter aus den Abteilungen Medizinische Forschung und Anwendung, Applikations- bzw. Anwendungsspezialisten und Produktmanagement gewählt, da hier ein guter/großer Praxisbezug herrscht und die Erfahrungen, Wissen und Anforderungen ähnlich zu denen der zukünftigen Kunden sind.

%\citeDürrenberger für Fokusgruppen!!

Für die Evaluierung wurde eine (abgewandelte Form) Art der Fokusgruppen-Evaluation gewählt. \cite{Christoforakos.2017} 
Hierbei wird eine kleinere Gruppe von Teilnehmern zusammengestellt, welche die voraussichtlichen Kunden generell abbilden soll. 
Danach ist es im Grunde genommen eine offene Gruppendiskussion, welche durch einen Moderator geleitet wird, sodass der entsprechende Fokus nicht außer Acht gelassen wird und etwaige Diskussionen und Gedankenaustausche in die richtige Richtung gelenkt werden. \cite{Durrenberger.1999} vgl. \cite{UsabilityinGermany.}
In diesen Beispielen spielt die Beobachtung der Teilnehmer keine große Rolle, wie normalerweise in den meisten Fokusgruppen üblich, da sie dass System als solche nicht bedienen und in diesem Fall aus der Beobachtung keine großen Erkenntnisse gewonnen werden können.

Die Vorteile einer Fokusgruppe sind, dass sich Personen in diesem Gebiet bereits auskennen und direktes Feedback geben können. 
So ergibt sich in relativ kurzer Zeit eine große Datenmenge und kritische Rückmeldungen zum Prototyp, welche bei Bedarf im gleichen Zug weiter ausgeführt und/oder diskutiert werden können.
Auch möglicherweise auftretende Fragen von den Personen können, im Gegensatz zu reinen Umfragen, direkt beantwortet werden um so Missverständnisse und Unklarheiten aufzuklären.

Für die Evaluierung wurde eine Gruppengröße von ca. 3-4 Personen gewählt, was etwas unter der üblichen Fokusgruppengröße von ca. 6-8 Personen liegt (vgl. \cite{UsabilityinGermany.})?. 
Grund hierfür sind zum einen die erwünschten 3-4 unterschiedlichen Gruppen, welche bei den ca. 16 geeigneten Personen diese Gruppengröße vorgibt, zum anderen ist es auch von Vorteil was das Einbringen von Ideen und Kritik von jeder Einzelperson angeht.
Dennoch kann ein fachlicher Austausch zwischen verschiedenen Personen stattfinden, welcher für weiterführende Diskussionen sehr hilfreich sein kann. 

Bei der Einteilung der Personen in die jeweiligen Fokusgruppen wurde auf verschiedene Eigenschaften und Bedingungen, wie zum Beispiel Alter, Abteilung, Wissenstand, Erfahrungen, und vieles mehr geachtet. 
Ziel waren einerseits homogene Gruppen, wo die Personen ähnliche Hintergründe und Altersklassen haben, andererseits sollte auch eine gewisse Heterogenität herrschen, damit verschiedene Meinungen aufeinander treffen und kontroverse Diskussionen angeregt/gefördert werden.
So ist beispielsweise eine Gruppe von eher jungen medizinischen Forschern in einer ähnlichen Altersklasse zwischen 23-27 Jahren, aber mit verschiedenen Hintergründen, beziehungsweise Spezialisierungen zu Informatik, Rettungsdienst und medizinische Studien und Auswertungen.

Mit der kleineren Größe der Gruppe wurde auch ein etwas kürzerer Zeitraum der jeweiligen Evaluierung gewählt.
Demnach liegt sie (des Öfteren) bei einer Gruppe von 6-8 Personen bei ca. 90 Minuten (vgl. \cite{UsabilityinGermany.})? und für die hier durchgeführten Termine wurden 60 Minuten angesetzt, was pro Person gerechnet sogar etwas mehr Zeit.
Jedoch ist diese Rechnung/Vergleich nicht repräsentativ und daher mit Vorsicht zu betrachten.

Es werden die erstellten Dashboards aus \ref{erstellungprototyp} in der Gruppe vorgestellt. 
Dabei wird anfangs den Personen erläutert, was der Sinn und Zweck der Visualisierungen ist und für welche Endnutzer er von Relevanz sein wird.
Somit können sie sich in die Lage der Kunden versetzen und aus deren Perspektive die präsentierten Ergebnisse kritisch betrachten.

Anschließend wird jedes Arbeitsblatt präsentiert, gefolgt von einer kurzen Pause, damit sich der erste Eindruck bilden kann und darauffolgend eine kurze Einführung/Erklärung zum aktuellen Fenster geliefert, um eine einheitliche Diskussionsbasis zu schaffen.
Des Weiteren wurden Hand-Outs ausgehändigt, damit jede Person zu jeder Zeit jedes Dashboard vor sich liegen hat um gegebenenfalls Anmerkungen, Kommentare, Verbesserungsvorschläge, Fragen oder ähnliches an die entsprechende Stelle notieren zu können.
Dies hat den weiteren Vorteil, dass es zur Nachbereitung verwendet werden kann, da die Gedanken und Ideen der Teilnehmer im Nachhinein dokumentiert zur Verfügung stehen und weitere Schritte oder Änderungen darauf basierend vorgenommen werden können. 
Ein Auszug dieser Hand-Outs können im Anhang \ref{evaluierunghandout} betrachtet werden.
% Die dargestellten Abbildungen können von \ref{konzeptERstellung} abweichen, da im iterativen Prozess immer Änderungen durchgeführt wurden, und somit keine zeitliche Reihenfolge gegeben ist.

\subsection{Gruppe 1: medizinische Forschung und Anwendung}
j
\subsection{Gruppe 2: Applikationsspezialisten}
j
\subsection{Gruppe 3: Produktmanagement und medizinische Forschung und Anwendung}
j


\subsection{Analyse der Ergebnisse}
%notwendig? oder in jeweil. subs integrieren
