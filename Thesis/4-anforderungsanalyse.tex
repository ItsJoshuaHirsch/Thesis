\chapter{Anforderungsanalyse}
\label{anforderungsanalyse}
\minitoc\pagebreak
%\section{Requirements Engineering}
% \lipsum[1-52]
\section{Nutzergruppen}
Die unterschiedlichen Nutzergruppen, oftmals auch Benutzerklassen genannt (vgl. \cite[S. 125]{Herczeg.2018}), einer Anwendung sind ein essentieller, beeinflussender Faktor, wenn es um die Entwicklung einer Anwendung oder, wie in dieser Arbeit, die Erweiterung einer Anwendung um ein umfangreiches Feature geht. Sie beschränken, aber auch erweitern die geforderten Funktionalitäten und beeinflussen die Art und Weise, wie die Anwendung gestaltet wird. Beispielsweise welche Erfahrungswerte vorausgesetzt oder angelernt werden müssen, welche Icons oder Symbole bekannt oder unbekannt sind, welche Ziele, Erwartungen oder Befürchtungen sie haben spielen eine große Rolle bei der Entscheidung, welche Informationen in welchem Format dargestellt werden.

\begin{quote}
"`Ein Stakeholder ist eine Person oder Organisation, die einen direkten oder indirekten Einfluss auf die Systemanforderung hat"' \cite{Pohl.2011} 
\end{quote}
Auch eine "`Stakeholder-Liste"' ist ein hilfreiches Mittel, um einen Überblick an den beteiligten Personen zu erhalten und damit einhergehend so viele Anforderungen wie möglich an das System zu erheben (vgl. \cite[S. 83]{Bergsmann.2018}).
Dieser Prozess ist vor allem zu Beginn der Anforderungsanalyse von besonderer Bedeutung, da so frühzeitig viele Informationen und Hinweise erkennbar werden, wie das System aufgebaut werden muss.
Ein nachträgliches anpassen oder hinzufügen von Anforderungen kann unter Umständen schwere grundlegende Probleme mit sich ziehen, da gewisse fundamentale Architekturentscheidung auf der Basis dieser Ansprüche aufbauen können.

Dabei kann grundsätzlich zwischen zwei Gruppen von Stakeholdern unterschieden werden \cite{Leffingwell.2011}:
\begin{description}
\item [Requirements-Provider] \hfill \\
Diese Personen oder Organisationen sind in der Regel jene, welche die Software oder das Produkt im Tagesgeschäft nutzen beziehungsweise voraussichtlich nutzen werden.
Sie haben neue Ideen oder Verbesserungsvorschläge aus vorherigen Anwendungen.
Dies können auch interne Mitarbeiter der Firma sein, welche besondere Erfahrungen in den zu entwickelnden Bereichen haben. %Ref to internal Stakeholders somehow?
\item [Constraint-Provider] \hfill \\
Hierbei handelt es sich um Personen, welche keine Anforderungen als solche definieren, sondern die Umsetzbarkeit der Anforderungen von den Requirements-Providern testen und daraus folgend technische Rahmenbedingungen, wie zum Beispiel die zu verwendende  Programmiersprache oder Datenbank, empfehlen oder festlegen.
\end{description}
\todotext{Referenzierung oder Verbindung zu meinen Nutzergruppen, Stakeholdern. \\Abgrenzung nur Anwender,keine Deployer etc.?, \\ Anforderungen weniger, sondern nur Fragestellungen }

Daher ist zuerst zu klären, welche Nutzergruppen bei dem bisherigen Produkt vertreten sind.
Anschließend sollten diese Benutzerklassen in Hinblick auf das erarbeitende Thema analysiert werden und daraufhin ist zu definieren, welche Eigenschaften oder Charakteristiken diese besitzen und ob gegebenenfalls Neue hinzukommen oder Alte hinfällig sind. 

\subsection{Stakeholder von corpuls.web ANALYSE}

% not sure if I#ll keep this table or just make a Auszug of it and more text?
\begin{table}
\centering
\setlength{\extrarowheight}{4pt}
\begin{tabular}{ | p{0.05\linewidth} | p{0.2\linewidth} | p{0.4\linewidth} | p{0.35\linewidth} |}
  \hline
  \textbf{ID} & \textbf{Stakeholder-Bezeichnung} & \textbf{Organisation} & \textbf{Kontaktperson}
  \\\hline
  1			& Geschäftsführer & GS Elektromedizinische Geräte & Dr. Klimmer
  \\\hline  
  2			& Entwickler & GS Elektromedizinische Geräte & Florian Lehmann
  \\\hline
\end{tabular} 
  \caption[Stakeholder-Liste]{Stakeholder-Liste nach \cite[S. 85]{Bergsmann.2018}}
  \label{tbl:Stakeholder-Liste}
\end{table}



\subsection{Personas}
\section{Aufgabenanalyse}

\section{Nutzungskontext}

\section{Iterative Erhebung der zu beantwortenden Fragestellungen}
\subsection{Ermittlung von Fragestellungen}
\subsection{Analyse der Anforderungen}
\subsection{Spezifikation der Fragen und benötigte Daten}
\subsection{Validierung der spezifizierten Fragestellungen}