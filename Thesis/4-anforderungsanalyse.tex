\chapter{Anforderungsanalyse}
\label{anforderungsanalyse}
\minitoc\pagebreak
%\section{Requirements Engineering}
% \lipsum[1-52]
\section{Nutzergruppen}
Die unterschiedlichen Nutzergruppen, oftmals auch Benutzerklassen genannt \cite{Herczeg.2018}, einer Anwendung sind ein essentieller, beeinflussender Faktor, wenn es um die Entwicklung einer Anwendung oder, wie in dieser Arbeit, die Erweiterung einer Anwendung um ein umfangreiches Feature geht. Sie beschränken, aber auch erweitern die geforderten Funktionalitäten und beeinflussen die Art und Weise, wie die Anwendung gestaltet wird. Beispielsweise welche Erfahrungswerte vorausgesetzt oder angelernt werden müssen, welche Icons oder Symbole bekannt oder unbekannt sind, welche Ziele, Erwartungen oder Befürchtungen sie haben spielen eine große Rolle bei der Entscheidung, welche Informationen in welchem Format dargestellt werden.

Daher ist zuerst zu klären, welche Nutzergruppen bei dem bisherigen Produkt vertreten sind. Anschließend sollten diese Benutzerklassen in Hinblick auf das erarbeitende Thema analysiert werden und daraufhin ist zu definieren, welche Eigenschaften oder Charakteristiken diese besitzen und ob gegebenenfalls Neue hinzukommen oder Alte hinfällig sind. 
\subsection{Personas}
\section{Aufgabenanalyse}

\section{Nutzungskontext}

\section{Iterative Erhebung der zu beantwortenden Fragestellungen}
\subsection{Ermittlung von Fragestellungen}
\subsection{Analyse der Anforderungen}
\subsection{Spezifikation der Fragen und benötigte Daten}
\subsection{Validierung der spezifizierten Fragestellungen}