\chapter{Grundlagen}
\label{chap:Grundlagen}
\minitoc\pagebreak


\section{Visualisierung von Daten}
\subsection{Gestaltungsgrundsätze}
\subsection{Statistische Auswertungen}
%\subsubsection{Diagramme}
\subsection{Multidimensionale Daten darstellen}

\section{Begriffe zum Thema Data Analytics}
\subsection{Übersicht der Zusammenhänge}
\subsection{Big Data}
\subsection{Business Intelligence}
\subsection{Data Warehousing}
\subsection{Dashboards}

\section{Daten und deren Verarbeitung in der Notfallmedizin} % Rettungsdienst  
\subsection{Business Intelligence in der Notfallmedizin}
\subsection{Big Data in der Notfallmedizin}
Big Data im Gesundheitswesen\\

%Noch umschreiben! Fast 100% aus Buch
3A: Aggregation, Analyse, Auswertung
4V: Volume, Varierty, Velocity \& !med wichtig Veracity! 
Grundsätzlich nur 3V, aber Veracity ist im GEsundheitsweesen von besonderer Bedeutung. 
Volume Viele Daten: Statista DAtenvolumen.
Variety: Häufig unstrukturiert, Papier, Memo, Blog, Freteixt; Im GW besonder Rezepte, Arztmemos, Arztbriefe, Emails
Velocity: Geschwindigkeit auch wichig, besonders auch in GW, zb. weitere Befunde/ANalysen während initialer Diagnose oder erste Behandlung...
Veracity: DAten sind per se nicht gut/sclhect, qualitativ hochwertig oder mangelhaft. Daten sind neutral, wenn auch komplex sozial und technologisch Ref Gitelmann 2ff.Entscheidend ist die richtige Fragstellung im Kontext und Algorithmus. Qualitativ hochwertige Bearbeitung der Daten! \\

eHealth <-> Big Data (S.37f)
eHealth beschreibt Interface, welches mittels App, online-Plattform,... gesundheitsbezogene Dienstleistungen zur verfügung stellt, Menschen miteineander verbindet eoder technoglogische Erkenntnisse für Menschen sichtbart macht. 
- ehealth Endgeräteübergreifend, mHealth nur mobile, vHealth VR, aHealth Augemnted
- eHealth bietet Basis für Big Data, Big Data bietet Basis für eHealth, ...; nicht alles was ehealth ist, ist big data and vice versa; Trennung notwendig! picture\\

Datenquellen (relevanten Auszug davon hervorheben)(S.43f); 

\begin{table}
\centering
\caption{My caption}
\label{my-label}
\begin{tabular}{|l|l|} 
\hline
\textbf{Kategorie der Datenquelle}  & \textbf{Ausgewählte Datenquellen}                                                \\ 
\hline
Medizinische Daten                  & \begin{tabular}[c]{@{}l@{}}Vitalparameter\\ Länge des Aufenthalts \end{tabular}  \\ 
\hline
Versicherungsdaten                  & \begin{tabular}[c]{@{}l@{}}Alter\\ Name \end{tabular}                            \\ 
\hline
Öffentliche Gesundheitsdaten        & \begin{tabular}[c]{@{}l@{}}Ämter\\Gemeinden\\...\end{tabular}                    \\ 
\hline
Forschungsdaten                     & \begin{tabular}[c]{@{}l@{}}Studien\\Biobanken\end{tabular}                       \\ 
\hline
Individ. Daten                      & \begin{tabular}[c]{@{}l@{}}Ernährung\\Wellness\end{tabular}                      \\ 
\hline
Pharmadaten                         & \begin{tabular}[c]{@{}l@{}}Medikamente\\Beschwerden\end{tabular}                 \\ 
\hline
Nichtklassische Ges.DAten           & \begin{tabular}[c]{@{}l@{}}Meinungen\\Telekomm.\end{tabular}                     \\
\hline
\end{tabular}
\end{table}

strukturiert, un- \& polystrukturiert (S.45): 
unstrukturiert: MRT, Röntgen, Studien,....
polystruk.: Grundlage für Big Data; bspw. Laborwerte mit SocialMedia, ...
img Statista\\


Anwendungsmöglichkeiten (S.46ff and BMG S.60ff) Epidemiologie, Epidemieprognose und Gesundheitsmonitoring: Überwachung von Krankheitsbildern, Symptome und Ursachen dieser. Verschiedene Studien, wie Engpässe der Ärzte durch Krankheuitsprognosen, Krebsrisiko durch lokale Luft u.a., oder NAKO Deutschland: neue Erkenntnisse über Volkskrankheiten, Risiken und Symptome. Prognose auch durch Verkehrswege, Luft und Frachtschiffverkehr, ... und weiter audh soziale Faktoren.
Gesundheitsprävention: Vorhersgaen durch z.B. Wetter und Gebiete, und individuelles Krankheitsbild von einem Patienten. Warnungen um vorzubeugen, Allergien und Asthmathiker, Pollenflüge, bestimmte Orte zu bestimmten Zeiten warnen.
Entscheidungsunterstützung: Auswertung von z.B. Tumordaten wie Gene und Proteine mit anderen um die Wirksamkeit verschiedener Medikamente zu überprüfen und anschließend ein geeignetes auszuwählen. Aber auch kleinere Entscheidungen (hier für uns / postum),
(Versorgungs-)Forschung: Unterstützung der bisherigen Forschungen mittels neuer Daten, auch Alltagsdaten oder andere med. Daten. , auch indiv. Patientenbehandlung mit Tumordaten etc.
Leistungs- und Qualitätsbeurteilung: Messung/Bewertung  der Usetzung von Vorgaben und Leitlinien ( §137 SGB V ? ) und Qualität, auch Behandlungen bewerten oder permanente Überachung von z.B. neugeborneren (Miskad/abernethy 2018) (Für uns sehr relevant!)
Betrugsbekämpfung: Missbrauch, Falschabrechnungen und Betrug, Medikamtentenmissbrauch,
(Interne) Prozessverbesserung: Für uns auch serh relelvatn; Personalplanung, Marketing, Controlling; AP-HP Frankreich -> Vorhersage über Patienten aufgund von Krankenhausdaten
\\

Limitationen(S.51) Auswahl des geeigneten Datensatzes, bzw. Wissen über die Limitation des DS, ! Auswahl der geeigneten Fragestellung, Analyseziel für gegebenen Datensatz !\\

Ausblick (S.53)  Deutscher Ethikrat 2017 Big data und Gesundheit; Islam, n.t. 2017: provably secure -> Auch gut für meinen Ausblick und besonders rechtliche Aspekte!\\ \\


S.68, 74, 77 Fotos Handy\\ \\


S.101ff, 116ff Fotos

\section{Technologien}
\subsection{Qlik Sense}

\section{Requirements Engineering}
\subsection{Methodik}