

\newglossaryentry{Zoomen}
{
     name=Zoomen,
     description={Vergrößerung ("`Zoom in"') oder Verkleinerung ("`Zoom out"') eines Bildschirminhaltes. Vom Englischen "`to zoom"', inzwischen aber auch in deutscher Umgangssprache gebräuchlich.}
}

\newglossaryentry{Key-Performance-Indicator}
{
     name=Key-Performance-Indicator,
     description={Zu Deutsch Kennzahl oder Leistungsindikator sind \glqq Zahlen, die zur Beurteilung der Leistung des betrachteten Objektes dienen. Es kann sich bei den Leistungsindikatoren entsprechend der verfolgten Ziele um Zeit-, Mengen- oder Wertgrößen handeln.\grqq{}  \cite[S. 342, S. 3f]{Friedl.2003, Maute.2009} }
}

\newglossaryentry{C.P.R}
{
     name=\textbf{C}ardio\textbf{p}ulmonary \textbf{r}esuscitation,
     description={Zu Deutsch kardiopulmonale Reanimation oder Herz-Lungen-Wiederbelebung ist die Sofortmaßnahme, wenn es zu einem Atem- und Kreislaufstillstand kommt. Dabei sollen etwa 100-120 Kompressionen des Brustkorbes mit einer optimalen Tiefe zwischen 5-6cm, sowie falls professionelles Personal vor Ort ist, Beatmungen durchgeführt werden. (vgl.\cite{Nolan.2010, Monsieurs.2015}) }
}

\newglossaryentry{CPR-Feedbacksensor}
{
     name=CPR-Feedbacksensor,
     description={\todo }
}

\newglossaryentry{C.C.F}
{
     name=\textbf{C}hest \textbf{C}ompression \textbf{F}raction,
     description={Zu Deutsch Brust-Kompressions-Anteil ist die \todo }
}

\newglossaryentry{A.E.D}
{
     name=\textbf{A}utomatisierter \textbf{e}xterner \textbf{D}efibrillator,
     description={\todo }
}

\newglossaryentry{pacer}
{
     name=Pacer,
     description={\todo }
}

\newglossaryentry{filter}
{
     name=Filter?,
     description={\todo }
}

\newglossaryentry{Drilldown}
{
     name=Drilldown,
     description={\todo }
}

\newglossaryentry{Hovern}
{
     name=Hovern,
     description={\todo }
}

\newglossaryentry{Button}
{
     name=Button,
     description={\todo }
}

\newglossaryentry{Kapnografie}
{
     name=Kapnografie,
     description={\todo }
}

\newglossaryentry{Binning}
{
     name=Binning,
     description={\todo }
}

\newglossaryentry{Feature}
{
     name=Feature,
     description={\todo }
}

\newglossaryentry{Dashboard}
{
     name=Dashboard,
     description={\todo }
}

\abk{BRK}{Bayerisches Rotes Kreuz}
\abk{DRK}{Deutsches Rotes Kreuz}
\abk{GS}{GS Elektromedizinische Geräte G. Stemple GmbH}
\abk{KPI}{\gls{Key-Performance-Indicator}}
\abk{CPR}{\gls{C.P.R}}
\abk{CCF}{\gls{C.C.F}}
\abk{AED}{\gls{A.E.D}}
\abk{NIBD}{nicht-invasive Blutdruckmessung}
\abk{C3}{\textsf{corpuls\textsuperscript{\color{corpulsred}{3}}}}
\abk{C1}{\textsf{corpuls\textsuperscript{\color{corpulsred}{1}}}}
\abk{cCPR}{\textsf{corpuls\textsuperscript{\color{corpulsred}{cpr}}}}
\abk{cAED}{\textsf{corpuls\textsuperscript{\color{corpulsred}{aed}}}}
\abk{LIVE}{\textsf{corpuls\color{corpulsred}{.web} \color{black}{LIVE}}}
\abk{ANALYSE}{\textsf{corpuls\color{corpulsred}{.web} \color{black}{ANALYSE}}}
\abk{REVIEW}{\textsf{corpuls\color{corpulsred}{.web} \color{black}{REVIEW}}}


