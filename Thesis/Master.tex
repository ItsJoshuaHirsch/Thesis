
% Diese Datei gilt als Master für die Latex-Kompilierung.
% Hier werden die einzelnen Unterdateien zusammengeführt.
% Im Latex-Programm daher immer diese Datei kompilieren und nicht die Unterdateien selbst!
% Wenn eine neue Datei hinzugefügt wurde, muss diese hier unbedingt eingebunden werden!!
% Gemeinsame Einstellungen aus der Präambel
% PRÄAMBEL für Latexdokumente
% Diese Datei muss als erstes in einem neuen Latexdokument eingebunden werden (per \insert )
% und darf NICHT (!) in einen document-Block eingebettet werden!

%\documentclass[fontsize=11pt, paper=a4, headinclude, twoside=false, parskip=half+, pagesize=auto, numbers=noenddot, open=right, toc=listof, toc=bibliography]{scrreprt}

\documentclass[headsepline,footsepline,footinclude=false,parskip=half+, oneside,fontsize=11pt,paper=a4,listof=totoc,bibliography=totoc]{scrbook} % one-sided

% PDF-Kompression
\pdfminorversion=5
\pdfobjcompresslevel=1

% Allgemeines
%\usepackage[automark]{scrpage2} % Kopf- und Fußzeilen
\usepackage{amsmath,marvosym} % Mathesachen
\usepackage[T1]{fontenc} % Ligaturen, richtige Umlaute im PDF
\usepackage[utf8]{inputenc}% UTF8-Kodierung für Umlaute usw

% Schriften
%\usepackage{fourier}
%\usepackage[light]{kpfonts}
%\usepackage{gfsdidot}
\usepackage{setspace} % Zeilenabstand

% Microtype
%\usepackage[activate={true,nocompatibility},final,tracking=true,kerning=true,spacing=true,factor=1100,stretch=10,shrink=10]{microtype}
\usepackage[tracking=true]{microtype}
\DeclareMicrotypeSet*[tracking]{my}% 
  { font = */*/*/sc/* }% 
\SetTracking{ encoding = *, shape = sc }{ 45 }% Hier wird festgelegt,
            % dass alle Passagen in Kapitälchen automatisch leicht
            % gesperrt werden.

% Schriften-Größen
\setkomafont{disposition}{\normalfont\bfseries} % use serif font for headings
%\linespread{1.05} % adjust line spread for mathpazo font


% Sprache: Deutsch
\usepackage[ngerman]{babel} % Silbentrennung
\usepackage[ngerman]{translator} % Deutsche Überschriften

% PDF
\usepackage[ngerman,pdfauthor={Joshua Hirsch},  pdfauthor={Joshua Hirsch}, pdftitle={Bachelorarbeit}, breaklinks=true,baseurl={http://www.corpuls.world}]{hyperref}
\usepackage{url}
\usepackage{pdflscape} % einzelne Seiten drehen können

%  Bibliographie
\usepackage{bibgerm} % Umlaute in BibTeX
\usepackage{epigraph} % Hervorgehobene Zitate

% Glossar
\usepackage[acronym,nonumberlist,toc]{glossaries}
%\renewcommand*{\glspostdescription}{} %Den Punkt am Ende jeder Beschreibung deaktivieren
\makeglossaries

% Tabellen
\usepackage{multirow} % Tabellen-Zellen über mehrere Zeilen
\usepackage{multicol} % mehre Spalten auf eine Seite
\usepackage{tabularx} % Für Tabellen mit vorgegeben Größen
\usepackage{longtable} % Tabellen über mehrere Seiten
\usepackage{array}
\usepackage{float}
\usepackage{rotating} % Tabellen im Querformat

% Bilder
\usepackage{graphicx} % Bilder
\usepackage{color} % Farben
\usepackage{floatflt} % Textumfluss
\usepackage{caption} % Verbesserte Untertitel
\graphicspath{{images/}}
\DeclareGraphicsExtensions{.pdf,.png,.jpg} % bevorzuge pdf-Dateien
\usepackage{subfigure} % mehrere Abbildungen nebeneinander/übereinander
\newcommand{\subfigureautorefname}{\figurename} % um \autoref auch für subfigures benutzen
\usepackage[all]{hypcap} % Beim Klicken auf Links zum Bild und nicht zu Caption gehen
\usepackage[section]{placeins} % Bilder nur in zugehöriger Section unterbringen

% Bildunterschrift
\setcapindent{0em} % kein Einrücken der Caption von Figures und Tabellen
\setcapwidth{0.9\textwidth}
\setlength{\abovecaptionskip}{0.2cm} % Abstand der zwischen Bild- und Bildunterschrift

% Quellcode
\usepackage{listings} % für Formatierung in Quelltexten
\definecolor{grau}{gray}{0.25}
\lstset{
	extendedchars=true,
	basicstyle=\tiny\ttfamily,
	%basicstyle=\footnotesize\ttfamily,
	tabsize=2,
	keywordstyle=\textbf,
	commentstyle=\color{grau},
	stringstyle=\textit,
	numbers=left,
	numberstyle=\tiny,
	% für schönen Zeilenumbruch
	breakautoindent  = true,
	breakindent      = 2em,
	breaklines       = true,
	postbreak        = ,
	prebreak         = \raisebox{-.8ex}[0ex][0ex]{\Righttorque},
}

% linksbündige Fußnoten
%\deffootnote{1.5em}{1em}{\makebox[1.5em][l]{\thefootnotemark}}


% für autoref von Gleichungen in itemize-Umgebungen
%\makeatletter
%\newcommand{\saved@equation}{}
%\let\saved@equation\equation
%\def\equation{\@hyper@itemfalse\saved@equation}
%\makeatother 

% Einstellungen der Seitenränder
% Change it! Look at Master-Thesis CGU
\usepackage[left=2.5cm,right=3cm,top=2cm,bottom=2cm,includeheadfoot]{geometry}
%\typearea{14} % typearea am Schluss berechnen lassen, damit die Einstellungen oben 

% Stuff
\usepackage{lipsum}

% Mini tableofcontent at each chapter
\usepackage[nohints]{minitoc} % Table of content at each chapter
% Mini tables of content at a chapter
\dominitoc
% More toc depth in minitoc
\setcounter{minitocdepth}{5}
\setcounter{secnumdepth}{5}
%\setcounter{tocdepth}{5}
% Only subsections in main TOC (no subsub)
% \setcounter{tocdepth}{1}
% Deutsches Mini-Verzeichnis
\mtcselectlanguage{german}
\addto{\captionsngerman}{% Making babel aware of special titles
  \renewcommand{\mtctitle}{Inhalt}
}



% Eigene Befehle %%%%%%%%%%%%%%%%%%%%%%%%%%%%%%%%%%%%%%%%%%%%%%%%%5

% Bearbeitungshinweise im Text
\newcommand{\todo}[1]{
      {\colorbox{red}{ TODO: #1 }}
}
\newcommand{\todotext}[1]{
      {\color{red} TODO: #1} \normalfont
}
\newcommand{\info}[1]{
      {\colorbox{blue}{\color{white}(INFO: #1)}}
}

% Einfache Abkürzung
\newcommand{\abk}[2] {
	\newacronym{#1}{#1}{#2}
}

% bild mit defnierter Breite einfügen
\newcommand{\bild}[4]{
  \begin{figure}[!hbt]
    \centering
      \vspace{1ex}
      \includegraphics[width=#2]{img/#1}
      \caption[#4]{\label{fig:#1} #3}
    \vspace{1ex}
  \end{figure}
}
% bild mit eigener Breite
\newcommand{\bildfix}[3]{
  \begin{figure}[!hbt]
    \centering
      \vspace{1ex}
      \includegraphics{img/#1}
      \caption[#3]{\label{fig:#1} #2}
      \vspace{1ex}
  \end{figure}
}
% Bild todo
\newcommand{\bildtodo}[2]{
  \begin{figure}[!hbt]
    \begin{center}
      \vspace{2ex}
	      \includegraphics[width=6cm]{../common/todo}
      %\caption{\label{#1} \color{red}{ TODO: #2}}
      \caption{\label{fig:#1} \todotext{#2}}
      %{\caption{\label{#1} {\todo{#2}}}}
      \vspace{2ex}
    \end{center}
  \end{figure}
}
% Bild im Anhang
\newcommand{\bildanhang}[3]
{
  \begin{figure}[!hbt]
    \centering
    \vspace{1ex}
    \includegraphics[width=\linewidth]{img/anhang/#1}
    \caption[#3]{\label{fig:#1} #2}
    \vspace{1ex}
  \end{figure}
}
% Bild am rechten Rand
\newcommand{\bildrechts}[4]
{
	\begin{floatingfigure}[r]{#2}
		%\centering
		\includegraphics[width=#2]{img/#1}
		\captionsetup{width=#2}
		\caption[#4]{\label{fig:#1} #3}		
	\end{floatingfigure}
}

\usepackage{xcolor}
\definecolor{corpulsred}{HTML}{DD0B2F}

\newcommand{\cweba}{
	\textsf{\small
	%\textbf{
	corpuls.\color{corpulsred}{web} \color{black}{ANALYSE}
	%}
	}
}


% Basic information for cover & title page
\newcommand*{\getUniversity}{Hochschule Fulda}
\newcommand*{\getFaculty}{Fachbereich Angewandte Informatik}
\newcommand*{\getTitle}{Big Data Analytics mit Medizintechnik im Rettungsdienst}
\newcommand*{\getAuthor}{Joshua Hirsch}
\newcommand*{\getDoctype}{Bachelor Thesis in "Digitale Medien"}
\newcommand*{\getSupervisor}{Prof.~Dr.~Jan-Torsten Milde}
\newcommand*{\getAdvisor}{Christoph Graumann,~M.Sc.}
\newcommand*{\getSubmissionDate}{29.03.2019}
\newcommand*{\getSubmissionLocation}{Fulda}

% Lokale Definitionen


\newglossaryentry{Zoomen}
{
     name=Zoomen,
     description={Vergrößerung ("`Zoom in"') oder Verkleinerung ("`Zoom out"') eines Bildschirminhaltes. Vom Englischen "`to zoom"', inzwischen aber auch in deutscher Umgangssprache gebräuchlich.}
}

\newglossaryentry{Key-Performance-Indicator}
{
     name=Key-Performance-Indicator,
     description={Zu Deutsch Kennzahl oder Leistungsindikator sind \glqq Zahlen, die zur Beurteilung der Leistung des betrachteten Objektes dienen. Es kann sich bei den Leistungsindikatoren entsprechend der verfolgten Ziele um Zeit-, Mengen- oder Wertgrößen handeln.\grqq{}  \cite[S. 342, S. 3f]{Friedl.2003, Maute.2009} }
}

\newglossaryentry{C.P.R}
{
     name=\textbf{C}ardio\textbf{p}ulmonary \textbf{r}esuscitation,
     description={Zu Deutsch kardiopulmonale Reanimation oder Herz-Lungen-Wiederbelebung ist die Sofortmaßnahme, wenn es zu einem Atem- und Kreislaufstillstand kommt. Dabei sollen etwa 100-120 Kompressionen des Brustkorbes mit einer optimalen Tiefe zwischen 5-6cm, sowie falls professionelles Personal vor Ort ist, Beatmungen durchgeführt werden. (vgl.\cite{Nolan.2010, Monsieurs.2015}) }
}

\newglossaryentry{CPR-Feedbacksensor}
{
     name=CPR-Feedbacksensor,
     description={\todo }
}

\newglossaryentry{C.C.F}
{
     name=\textbf{C}hest \textbf{C}ompression \textbf{F}raction,
     description={Zu Deutsch Brust-Kompressions-Anteil ist die \todo }
}

\newglossaryentry{A.E.D}
{
     name=\textbf{A}utomatisierter \textbf{e}xterner \textbf{D}efibrillator,
     description={\todo }
}

\newglossaryentry{pacer}
{
     name=Pacer,
     description={\todo }
}

\newglossaryentry{filter}
{
     name=Filter?,
     description={\todo }
}

\newglossaryentry{Drilldown}
{
     name=Drilldown,
     description={\todo }
}

\newglossaryentry{Hovern}
{
     name=Hovern,
     description={\todo }
}

\newglossaryentry{Button}
{
     name=Button,
     description={\todo }
}

\newglossaryentry{Kapnografie}
{
     name=Kapnografie,
     description={\todo }
}

\newglossaryentry{Binning}
{
     name=Binning,
     description={\todo }
}

\newglossaryentry{Feature}
{
     name=Feature,
     description={\todo }
}

\newglossaryentry{Dashboard}
{
     name=Dashboard,
     description={\todo }
}
\newglossaryentry{Archetyp}
{
     name=Archetyp,
     description={\glqq 2.a. (Psychologie) eins der ererbten, im kollektiven Unbewussten bereitliegenden urtümlichen Bilder, die Gestaltungen [vor]menschlicher Grunderfahrungen sind und zusammen die genetische Grundlage der Persönlichkeitsstruktur repräsentieren (nach C. G. Jung).
     2.b. (bildungssprachlich) Urform, Musterbild\grqq \cite{.2011}}
}

\abk{BRK}{Bayerisches Rotes Kreuz}
\abk{DRK}{Deutsches Rotes Kreuz}
\abk{GS}{GS Elektromedizinische Geräte G. Stemple GmbH}
\abk{KPI}{\gls{Key-Performance-Indicator}}
\abk{CPR}{\gls{C.P.R}}
\abk{CCF}{\gls{C.C.F}}
\abk{AED}{\gls{A.E.D}}
\abk{NIBD}{nicht-invasive Blutdruckmessung}
\abk{C3}{\textsf{corpuls\textsuperscript{\color{corpulsred}{3}}}}
\abk{C1}{\textsf{corpuls\textsuperscript{\color{corpulsred}{1}}}}
\abk{cCPR}{\textsf{corpuls\textsuperscript{\color{corpulsred}{cpr}}}}
\abk{cAED}{\textsf{corpuls\textsuperscript{\color{corpulsred}{aed}}}}
\abk{LIVE}{\textsf{corpuls\color{corpulsred}{.web} \color{black}{LIVE}}}
\abk{ANALYSE}{\textsf{corpuls\color{corpulsred}{.web} \color{black}{ANALYSE}}}
\abk{REVIEW}{\textsf{corpuls\color{corpulsred}{.web} \color{black}{REVIEW}}}




% hier beginnt der eigentliche Inhalt
\begin{document}

\begin{titlepage}
  \centering

  \vspace{30mm}
  \includegraphics[width=40mm]{logos/hs-small}  

  \vspace{5mm}
  {\huge\MakeUppercase{\getUniversity{}}}\\
  
  \vspace{5mm}
  \includegraphics[width=12mm]{logos/ai}
  
  %\vspace{5mm}
  {\large\MakeUppercase{\getFaculty{}}}\\
  

  \vspace{23mm}
  {\Large \getDoctype{}}

  \vspace{7mm}
  \begin{doublespace}
    {\huge\bfseries \getTitle{}}
  \end{doublespace}


  \vspace{35mm}
  \begin{tabular}{l l}
    Verfasser: & \getAuthor{} \\
    \small{Matrikelnummer:} & \small{\getNumber{}} \\
    Referent: & \getSupervisor{} \\
    Korreferent: & \getAdvisor{} \\
    Abgabedatum: & \getSubmissionDate{} \\
  \end{tabular}

  \vspace{10mm}
  \includegraphics[width=75mm]{logos/gs}
\end{titlepage}
 % Die Titelseite
\newpage\thispagestyle{empty}
\section*{Zusammenfassung}

abs % Zusammenfassung


% Inhaltsverzeichnis
\pagenumbering{roman} % römische Ziffern
\singlespacing
\tableofcontents{}
\newpage

\pagenumbering{arabic} % arabische Seitenzahlen für den Rest des Dokuments


%%%%%%%%%% HIER DIE EINZELNEN KAPITEL EINBINDEN:

\onehalfspacing
\chapter{Einleitung}
\label{einleitung}
\minitoc\pagebreak


\section{Firmenvorstellung}
\bild
{../logos/gs}
{8cm}
{Logo der Marke \glqq corpuls\grqq{} von \acrlong*{GS}. Bildquelle: Intern}
{Logo der Marke \glqq corpuls\grqq{} von GS}

Das Unternehmen \gls{GS}, ist in Fachkreisen unter dem Markennamen \glqq corpuls\grqq{} wohl bekannt. 
Die Firma beschäftigt sich seit über 35 Jahren (1982) mit Medizintechnik.
Dabei beinhaltet das Produktportfolio vor allem Rettungsgeräte für den präklinischen Einsatz.
Darunter fallen Aufgaben wie Reanimation, Defibrillation sowie Monitoring von Patienten. 

Als Garagenfirma von G. Stemple gegründet, hat sich das bayerische Unternehmen mittlerweile in den Mittelstand entwickelt.
Dabei hat es diverse Auszeichnungen vor allem für die innovativen Entwicklungen erhalten.

Von Beginn an überzeugten die ersten Geräte corpuls200 \& corpuls300 den Markt der präklinischen Notfallmedizin mit Beständigkeit, leichtem Gewicht und hoher Funktionalität. \todo{dreiteiligkeit?}

Ebenso Pioniertätigkeiten in der Telemedizin sind \gls{GS} zuzuschreiben, da 1984 bereits eine Übertragung eines 1-Kanal-EKGs über den Globus nach Deutschland stattgefunden hat (\todo{add gs corpuls citavi}).

\todo{more}
%Mit dem corpuls 08/16 (1992) und dem corpuls3 (2007) gelang es GS, sich entgültig im
%Weltmarkt zu etablieren.
%In insgesamt drei Gerätegenerationen wurde der corpuls3 mit einem schmalen Defibrillationsmodul
%zum einen leichter, zum anderen über diverse Soft- und Hardwareupdates
%kontinuierlich verbessert.
%Speziell für die Anwendergruppen in den Bereichen Feuerwehr, First-Responder, Krankentransport
%und Katastrophenschutz wurde 2011 der corpuls1 (Abbildung 2) vorgestellt.
%Vor Allem zeichnet der sich der corpuls1 durch seine Kompaktheit aus, während dennoch
%umfangreiche Funktionalitäten bereit gestellt werden.
%Abgerundet wird die Versorgungskette in verschiedenen Kompetenzbereichen durch den
%2018 vorgestellten Automatisierter externer Defibrillator (AED) corpulsAED (Abbildung 2).
%Damit werden nun auch die Anwendergruppe der Laien abgedeckt, aber auch im professionellen
%Bereich ist der corpulsAED vielseitig einsetzbar.
%Durch klare, strukturierte
%Anweisungen führt der corpulsAED einen Laien kompetent durch ein Reanimationsgeschehen
%Einen neuen Schritt wagte GS 2015, als diese in den Bereich der Thoraxkompressionsger
%äte expandierte und den corpuls cpr (Abbildung 3) vorstellten.
%Besonderheit am corpuls cpr ist, dass dieser nur ein Arm auf einer Seite besitzt, was die Nutzerfreundlichkeit
%extrem erhöht.
%Durch die automatischen Thoraxkompressionsgeräte werden die
%Helfer entlastet und können die patientenkritische Zeit besser nutzen.
%Auch wird durch
%diese Geräte eine kontinuerliche Reanimationsqualität und dadurch eine bessere Patientenversorgung
%erreicht.
%Abgerundet wird das Firmenprofil durch die zahlreichen corpuls.web Produkte.
%So wird
%jedem Käufer die kostenlose Anwendung corpuls.web REVIEW zur Verfügung gestellt.
%Durch die Anwendung hat jeder Nutzer die Möglichkeit, Einsätze - egal ob auf dem
%corpuls3, corpuls1, corpulsAED oder corpuls cpr - nach dem Einsatz nachzubereiten.
%Zum
%Beispiel bekommt der Nutzer nach einer Reanimation direkt Informationen zu Druckfrequenz,
%Drucktiefe, Reanmiationspausen, Schocks und Events.
%Ergänzt wird corpuls.web REVIEW durch die zentrale Datenverwaltungssoftware corpuls.
%web ANALYSE.
%Hier hat der Nutzer die Möglichkeit, in einer zentralen Datenbank
%alle Einsatzdaten (anonymisiert) zu speichern und auf diese statistisch zuzugreifen.
%Auch Funktionalitäten eines bestimmten Produkts kann durch corpuls.web Software
%erweitert werden.
%So steht Kunden vieler corpulsAED die Gerätemanagementsoftware
%corpuls.web MANAGER zur Verfügung.
%Mithilfe des corpuls.web MANAGER gibt es
%eine zentrale Möglichkeit, alle corpulsAED zu überwachen.
%Die corpulsAED senden regelm
%äÿig die Ergebnisse ihrer Selbsttests an den corpuls.web MANAGER, womit sich die Einsatzbereitschaft der Geräte überwachen lässt.
%Aber auch Softwareupdates und Kon-
%figurationsprofile lassen sich über die Software corpuls.web MANAGER zentral für alle
%Geräte verwalten.
%Abgerundet wird das Softwareangebot durch die Telemetriesoftware corpuls.web LIVE.
%Mithilfe von corpuls.web LIVE hat präklinisches Personal wie Rettungsdienste die Möglichkeit,
%vorab erhobene Patientendaten wie Stammdaten, Vitalparameter oder Ruhe-
%EKG-Daten in Echtzeit an Fachärzte zu übertragen.
%Dadurch kann sich die aufnehmende
%Klinik bereits vorab ein erstes Krankheitsbild des Patienten schaffen und sich besser auf
%seine Ankunft vorbereiten.
%Diese Arbeit beschäftigt sich maÿgeblich mit der Softwarelösung corpuls.web LIVE und
%dem corpuls3, da das Alarmsystem in corpuls.web LIVE integriert werden soll.
%Deshalb werden diese beiden Geräte nun nachfolgend detaillierter vorgestellt.
\subsection{Relevante Produkte}
\subsubsection{\acrlong*{C3}}
\todo{}
%Der primäre Zielmarkt für den corpuls3 ist die professionelle, präklinische Notfallrettung.
%Das auffälligste Merkmal ist die Modularität des corpuls3.
%Dieser kann in die Module Monitoreinheit, Patientenbox sowie Defibrillator/ Schrittmachereinheit
%aufgeteilt werden (Abbildung 4).
%
%Eine weitere, wichtige Eigenschaft des Geräts ist die Möglichkeit, Stammdaten, Vitaldaten,
%12-Kanal-EKGs sowie Nachrichten über die Telemetrieschnittstelle an corpuls.web
%LIVE zu senden.
%Dadurch kann der Patient zum Beispiel an zentraler Stelle überwacht
%werden oder die aufnehmende Klinik kann sich bereits vorab ein erstes Bild des Patienten
%schaffen, bevor dieser ins Klinikum eingeliefert wurde.
%Diese Telemetrieverbindung
%kann entweder per LAN, WLAN oder GSM erfolgen.
\subsubsection{\acrlong*{ANALYSE}}
\todo{}

\section{Problemstellung}
\label{problem}
Ein Einsatz im Rettungsdienst ist jedes Mal ein gänzlich neuer Fall. 
Es kommen verschiedene Faktoren in unterschiedlicher Gewichtung hinzu und machen einen solchen Einsatz einzigartig. 
Dennoch lassen sich Korrelationen feststellen, welche in einigen Einsätzen zu gleichen oder ähnlichen Ereignissen führen.

Die heutzutage eingesetzten Geräte in diesen Bereichen besitzen viel Technik und Möglichkeiten der Datensammlung und -haltung.
Diese Daten müssen, wegen rechtlicher Aspekte zumindest in Deutschland, permanent abgespeichert werden.
Jedoch liegen diese meistens im Anschluss der Persistierung ungenutzt auf einem Datenträger oder Server.

Dabei verbergen sich in diesen Daten neue Erkenntnisse und unzählige Antworten auf Fragen, welche sich die Leiter entsprechender Einrichtungen jährlich, monatlich oder gar täglich stellen, und zum jetzigen Zeitpunkt keine adäquate, schnelle und simple Möglichkeit zur Beantwortung jener zur Verfügung haben.

Mit dieser Arbeit sollen genau diese ungenutzten Daten so aufbereitet werden, damit sie für eine grafische Auswertung sinnvoll sind und Rettungswachen, Krankenhäusern oder anderen Einrichtungen einen deutlichen Mehrwert bieten.

\subsection{Erkenntnisinteresse}
\label{erkenntnis}
Mit der grafischen Auswertung von vielen Einsätzen über längere Zeiträume mit unterschiedlichen Geräten sollen bereits vermutete Fragestellungen bestätigt oder widerlegt werden können.
Fragen wie \glqq Wird tagsüber besser reanimiert als nachts? Wie steht dies im Verhältnis zur Einsatzdichte der jeweiligen Schichten?\grqq{} sind nur einige der vielen Fragen, die derzeit im Raum stehen und nur schwerlich mit einer Antwort ausgestattet werden können.
Auch sehr forschungsnahe Fragen, beispielsweise ob der Anstieg des Blutdrucks in Kombination mit der Senkung der Sauerstoffsättigung zu einer Apnoe führt, sind denkbar spannend.

Die Daten sollen entsprechend vorliegen, damit die zu erwartenden Fragen in adäquater Form und Zeit beantwortet werden können.
Hierbei soll auch die Möglichkeit der Überprüfung gewährleistet werden, dass entsprechende gesetzliche Richtlinien \cite{Maconochie.2015} oder lokale Vorgaben eingehalten werden oder die Stärken und Schwächen von Menschen und Geräten identifiziert werden können.
Auch Prognosen für die Zukunft sollen möglich gemacht werden oder gar neue Forschungsfragen gefunden  und im besten Fall gleich beantwortet werden können.

Eine gute Möglichkeit um diese Daten übersichtlich zu visualisieren sind sogenannte \glqq \gls{Dashboard}s\grqq, ein englischer Begriff welcher \glqq Armaturenbrett\grqq{} bedeutet.
Dieser hat sich in der digitalen Welt als Schlagwort etabliert, eine Übersicht über viele verschiedene Informationen und Daten zu liefern, so wie es ein Armaturenbrett im Auto vollbringt (siehe \ref{sub:dashboards}).

\subsection{Fragestellungen}
\label{fragen}
Aus der oben genannten Problemstellung und Erkenntnisinteresse ergeben sich folgende Fragen:
\begin{description}
\item[Fragestellungen]~\par
\begin{itemize}
      \item Welche Fragen haben unsere Anwender, die wir mit unseren Daten beantworten können?    
      \item Wie müssen wir Dashboards gestalten, damit diese Fragen beantwortet werden?
      \begin{itemize}
        \item Wie müssen die Dashboards entworfen werden, damit sie medizinisch korrekt sind?  
      \end{itemize}
      \item Was müssen wir in unserem Datenmodell beachten, damit wir diese Dashboards erstellen 					können?
      \begin{itemize}
        \item Was muss bei der Schnittstelle beachtet werden?
      \end{itemize}
\end{itemize}
\end{description}

\subsection{Zielsetzung}
\label{ziel}
Ziel ist es, so viele Fragestellungen wie möglich der entsprechenden Anwender, wie zum Beispiel Rettungswachenleiter, Ärztlicher Leiter, Qualitätsmanagementbeauftragte u.a. herauszufinden und zu konkretisieren.
Anschließend sollen die erhobenen Fragen bezüglich der benötigten Daten und deren Format analysiert werden.
Daraufhin folgt die Konzipierung von Dashboards, welche so viele Fragestellungen wie möglich beantworten sollen.
Erste Entwürfe und letztendlich präsentierfähige Dashboards sollen das visuelle Ergebnis dieser
Arbeit werden.

Simultan werden geeignete Datenmodelle zur reibungslosen Darstellung sowie Anforderungen an die Schnittstelle erarbeitet.

\subsection{Abgrenzung}
\label{abgrenzung}
Im Rahmen der Bachelorarbeit ist es nicht notwendig, die entstandenen Dashboards in die derzeit laufende Software zu implementieren.
%Handlungsempfehlungen für die entsprechenden Entwickler sind hierbei ausreichend.
Des Weiteren ist keine produktive Anbindung an die entsprechende Schnittstelle vorgesehen.
Etwaige Rückschlüsse oder das Testen von Technologien sind hierbei zureichend.

\section{Vorgehensbeschreibung}
Die Erhebung der Fragestellungen wird vollumfänglich durch Interviews stattfinden.
Dies passiert in einem zyklisch-iterativen Verfahren, gemäß etablierter Prozesse des Requirements Engineering \cite{Pohl.2011} (siehe Kapitel \ref{sub:methodik}).
Dabei werden vorerst die internen Mitarbeiter gefragt, welche Fragestellungen sie für sinnvoll erachten.
Dabei wird darauf geachtet, dass die befragten Personen einen engen Bezug zum Rettungsdienst oder zu Kunden, beziehungsweise Mitarbeitern in dieser Branche haben.
Das Unternehmen liefert hierbei eine Menge in Frage kommender Stakeholder, da die Quote der ehemaligen und noch aktiven Rettungsdienstmitarbeiter überdurchschnittlich hoch ist (siehe Kapitel \ref{sec:nutzergruppen}). 

%Des Weiteren werden Interviews oder Gespräche mit Key-Opinion-Leaders dieser Thematik angestrebt, um Einschätzungen und Erkenntnisse aus professioneller, erster Hand zu gewinnen.

Im Anschluss der Erhebung werden die ermittelten Fragestellungen analysiert, hierbei wird untersucht:
\begin{itemize}
      \item Welche Daten für jene Frage benötigt werden
      \item Liegen die benötigten Daten vor
      \begin{itemize}
        \item Wenn ja, sind die Daten in dem richtigen Format
      \end{itemize}
      \item Wie relevant, beziehungsweise interessant die Frage allgemein ist
\end{itemize}
In Bezug darauf werden exemplarisch Dashboards für die Fragestellungen entworfen, bei welchen die benötigten Daten im richtigen Format vorliegen.
Danach werden diese vorgestellt und von entsprechenden Mitarbeitern validiert und evaluiert.
Parallel dazu werden für die Fragestellungen, welche die Daten nicht im richtigen Format vorliegen haben, entsprechende Datenmodelle oder Datenformate erarbeitet und Hinweise an die Softwareabteilung bezüglich der Schnittstelle kommuniziert.
Anschließend beginnt der gesamte Prozess mit den herausgefundenen Validierungsergebnissen erneut.


%Nach den ersten Zyklen, sobald die ersten Dashboards und Metriken intern validiert wurden, werden diese entsprechend interessierten Kunden vorgestellt.
%Die Erkenntnisse hierbei werden besonders in die Evaluierung und erneute Spezifizierung einfließen, da die Erfahrungen und der Wissensstand dieser Personen von enormer Wichtigkeit sind.

\subsection{Technologien}\label{tech}
Für das Erstellen der Dashboards wird ein Business-Intelligence-Werkzeug namens \glqq Qlik Sense\grqq{} eingesetzt.
Hierbei handelt es sich um eine leistungsstarke Software, welche auch zum Beispiel das Laden der Daten oder Umstrukturieren der geladenen Daten per Skript erlaubt.
Im Unternehmen gibt es Lizenzen für die Enterprise Variante, welche auch für den späteren produktiven Einsatz bei Kunden zum Tragen kommen wird (siehe Kapitel \ref{sub:qlik}).
Wie in Kapitel \ref{abgrenzung} beschrieben werden sich die Entwickler mit der Programmierung Schnittstelle beschäftigen, sodass die Daten vom entsprechenden Server im hier erarbeiteten Format vorliegen.
% Eine mögliche Evaluation meinerseits von Technologien, welche für die Schnittstelle hilfreich sein könnten, ist optional.

\subsection{Datenmodellierung}
Das Erarbeiten der benötigten Datenmodelle wird vorerst ohne produktive Daten erfolgen.
Hierbei ist das Ziel, eine grundlegende Struktur abzubilden, welche eine breite und nutzbare Basis von Daten schaffen soll.
Diese wird mithilfe von visuellen Datenmodellen ebenfalls in der im Kapitel \ref{tech} genannten
Software dargestellt.

Die Daten der Geräte liegen größtenteils im eigenen \glqq corpuls\grqq{}-Format vor. 
Hierbei finden sich auch sogenannte \glqq Events\grqq{}, welche bestimmte Ereignisse im Laufe einer Mission mit Parametern beschreiben, wie zum Beispiel eine Blutdruckmessung mit dem systolischen und diastolischen Druck als Parameter.
Somit ist ein einzelner Einsatz in der Auswertung als abgeschlossener atomarer Datensatz anzusehen, wobei es genau genommen eine definierte Zeitspanne mit \textit{n} Ereignissen und Messungen ist. 

Damit ein solches mehrdimensionales Datenpaket derzeit zur einfachen manuellen Auswertung geeignet ist, besteht die Notwendigkeit einer gewissen internen Vorverarbeitung dieser Daten.
Dies wird zurzeit unter anderem mit sogenannten \glqq \gls{MM} \grqq{} abgebildet.
Diese bilden Aggregationen von bestimmten Events, sodass beispielsweise die Anzahl an Blutdruckmessungen zusammengefasst wird. 
Mit diesen Aggregationen lassen sich bereits viele Auswertungen und Erkenntnisse finden.

Jedoch ist für eine tiefere Analyse eine komplexere Datenstruktur notwendig, sodass die eigentlich gebotene Mehrdimensionalität zur Verfügung stehen sollte. 
Die konfliktfreie und sinnvolle Modellierung von mehreren solcher mehrdimensionalen Daten ist eine der großen Herausforderungen dieser Bachelorarbeit (siehe Kapitel \ref{sub:erweiterung} \& \ref{sub:datenmodell}).

\chapter{Grundlagen}
\label{chap:Grundlagen}
\minitoc\pagebreak


\section{Visualisierung von Daten}
\subsection{Gestaltungsgrundsätze}
\subsection{Statistische Auswertungen}
%\subsubsection{Diagramme}
\subsection{Multidimensionale Daten darstellen}

\section{Begriffe zum Thema Data Analytics}
\subsection{Übersicht der Zusammenhänge}
\subsection{Big Data}
\subsection{Business Intelligence}
\subsection{Data Warehousing}
\subsection{Dashboards}

\section{Daten und deren Verarbeitung in der Notfallmedizin} % Rettungsdienst  
\subsection{Business Intelligence in der Notfallmedizin}
\subsection{Big Data in der Notfallmedizin}
Big Data im Gesundheitswesen\\

%Noch umschreiben! Fast 100% aus Buch
3A: Aggregation, Analyse, Auswertung
4V: Volume, Varierty, Velocity \& !med wichtig Veracity! 
Grundsätzlich nur 3V, aber Veracity ist im GEsundheitsweesen von besonderer Bedeutung. 
Volume Viele Daten: Statista DAtenvolumen.
Variety: Häufig unstrukturiert, Papier, Memo, Blog, Freteixt; Im GW besonder Rezepte, Arztmemos, Arztbriefe, Emails
Velocity: Geschwindigkeit auch wichig, besonders auch in GW, zb. weitere Befunde/ANalysen während initialer Diagnose oder erste Behandlung...
Veracity: DAten sind per se nicht gut/sclhect, qualitativ hochwertig oder mangelhaft. Daten sind neutral, wenn auch komplex sozial und technologisch Ref Gitelmann 2ff.Entscheidend ist die richtige Fragstellung im Kontext und Algorithmus. Qualitativ hochwertige Bearbeitung der Daten! \\

eHealth <-> Big Data (S.37f)
eHealth beschreibt Interface, welches mittels App, online-Plattform,... gesundheitsbezogene Dienstleistungen zur verfügung stellt, Menschen miteineander verbindet eoder technoglogische Erkenntnisse für Menschen sichtbart macht. 
- ehealth Endgeräteübergreifend, mHealth nur mobile, vHealth VR, aHealth Augemnted
- eHealth bietet Basis für Big Data, Big Data bietet Basis für eHealth, ...; nicht alles was ehealth ist, ist big data and vice versa; Trennung notwendig! picture\\

Datenquellen (relevanten Auszug davon hervorheben)(S.43f); 

\begin{table}
\centering
\caption{My caption}
\label{my-label}
\begin{tabular}{|l|l|} 
\hline
\textbf{Kategorie der Datenquelle}  & \textbf{Ausgewählte Datenquellen}                                                \\ 
\hline
Medizinische Daten                  & \begin{tabular}[c]{@{}l@{}}Vitalparameter\\ Länge des Aufenthalts \end{tabular}  \\ 
\hline
Versicherungsdaten                  & \begin{tabular}[c]{@{}l@{}}Alter\\ Name \end{tabular}                            \\ 
\hline
Öffentliche Gesundheitsdaten        & \begin{tabular}[c]{@{}l@{}}Ämter\\Gemeinden\\...\end{tabular}                    \\ 
\hline
Forschungsdaten                     & \begin{tabular}[c]{@{}l@{}}Studien\\Biobanken\end{tabular}                       \\ 
\hline
Individ. Daten                      & \begin{tabular}[c]{@{}l@{}}Ernährung\\Wellness\end{tabular}                      \\ 
\hline
Pharmadaten                         & \begin{tabular}[c]{@{}l@{}}Medikamente\\Beschwerden\end{tabular}                 \\ 
\hline
Nichtklassische Ges.DAten           & \begin{tabular}[c]{@{}l@{}}Meinungen\\Telekomm.\end{tabular}                     \\
\hline
\end{tabular}
\end{table}

strukturiert, un- \& polystrukturiert (S.45): 
unstrukturiert: MRT, Röntgen, Studien,....
polystruk.: Grundlage für Big Data; bspw. Laborwerte mit SocialMedia, ...
img Statista\\


Anwendungsmöglichkeiten (S.46ff and BMG S.60ff) Epidemiologie, Epidemieprognose und Gesundheitsmonitoring: Überwachung von Krankheitsbildern, Symptome und Ursachen dieser. Verschiedene Studien, wie Engpässe der Ärzte durch Krankheuitsprognosen, Krebsrisiko durch lokale Luft u.a., oder NAKO Deutschland: neue Erkenntnisse über Volkskrankheiten, Risiken und Symptome. Prognose auch durch Verkehrswege, Luft und Frachtschiffverkehr, ... und weiter audh soziale Faktoren.
Gesundheitsprävention: Vorhersgaen durch z.B. Wetter und Gebiete, und individuelles Krankheitsbild von einem Patienten. Warnungen um vorzubeugen, Allergien und Asthmathiker, Pollenflüge, bestimmte Orte zu bestimmten Zeiten warnen.
Entscheidungsunterstützung: Auswertung von z.B. Tumordaten wie Gene und Proteine mit anderen um die Wirksamkeit verschiedener Medikamente zu überprüfen und anschließend ein geeignetes auszuwählen. Aber auch kleinere Entscheidungen (hier für uns / postum),
(Versorgungs-)Forschung: Unterstützung der bisherigen Forschungen mittels neuer Daten, auch Alltagsdaten oder andere med. Daten. , auch indiv. Patientenbehandlung mit Tumordaten etc.
Leistungs- und Qualitätsbeurteilung: Messung/Bewertung  der Usetzung von Vorgaben und Leitlinien ( §137 SGB V ? ) und Qualität, auch Behandlungen bewerten oder permanente Überachung von z.B. neugeborneren (Miskad/abernethy 2018) (Für uns sehr relevant!)
Betrugsbekämpfung: Missbrauch, Falschabrechnungen und Betrug, Medikamtentenmissbrauch,
(Interne) Prozessverbesserung: Für uns auch serh relelvatn; Personalplanung, Marketing, Controlling; AP-HP Frankreich -> Vorhersage über Patienten aufgund von Krankenhausdaten
\\

Limitationen(S.51) Auswahl des geeigneten Datensatzes, bzw. Wissen über die Limitation des DS, ! Auswahl der geeigneten Fragestellung, Analyseziel für gegebenen Datensatz !\\

Ausblick (S.53)  Deutscher Ethikrat 2017 Big data und Gesundheit; Islam, n.t. 2017: provably secure -> Auch gut für meinen Ausblick und besonders rechtliche Aspekte!\\ \\


S.68, 74, 77 Fotos Handy\\ \\


S.101ff, 116ff Fotos

\section{Technologien}
\subsection{Qlik Sense}

\section{Requirements Engineering}
\subsection{Methodik}
\chapter{Stand der Technik} 
\label{technikstand]}
\minitoc\pagebreak
%\section{Analyse} Eventuell eigenes Kapitel


\section{Ist-Zustand corpuls.web ANALYSE}
\subsection{Verwendung zur Auswertung}
\subsection{Vorliegende Daten der Geräte}


\section{Qlik Sense Server}
\subsection{Lizenzierungsmodell}
\subsection{?}
\chapter{Anforderungsanalyse}
\label{kap:anforderungsanalyse}
\minitoc\pagebreak
%\section{Requirements Engineering}
% \lipsum[1-52]

\section{Nutzungskontext}
\label{sec:kontext}

Die Anforderungsanalyse beginnt meist mit der Untersuchung des Nutzungskontextes \cite{Bergsmann.2018}.
Dabei ist es wichtig festzuhalten, wo das entsprechende Produkt eingesetzt wird.

Dadurch stellt sich zuerst die Frage, welche Einrichtungen im Besitz solcher medizinischen und technischen Daten der Geräte von \gls{GS} sind oder welche einen Mehrwert aus diesen generieren könnten.
Bekanntermaßen sind generell Rettungsdienste im Besitz solcher Geräte und den damit einhergehenden Daten.
Sie können auch einen Mehrwert aus diesen ziehen, da sich beispielsweise die Auswirkungen getätigter Planungen oder Investitionen in den Daten herauslesen lassen können.
Ein Beispiel hierfür wäre die Schichtplanung aufgrund von angenommenen Einsatzhäufigkeiten.
Anhand der vorliegenden Daten der Geräte kann sich diese Annahme bestätigen oder als falsch herausstellen und gegebenenfalls neu geplant werden.
Aber auch andere Einsichten können potentiell aus den Daten entspringen, wie etwa Schulungsmaßnahmen, da aufgrund der Daten eine schlechte Reanimationsqualität der Wache ersichtlich wird.

Diese und ähnliche weitere Aufgaben lassen sich nicht nur in Rettungswachen oder Verbänden solcher finden, sondern auch in Krankenhäusern und Kliniken.
Auch wenn die Geräte hier seltener als im präklinischen Bereich eingesetzt werden, ist es dennoch ein Ort und Kontext, welcher in dieser Analyse berücksichtigt wird.
In diesen Einrichtungen ist es üblich, im Gegensatz zu einer Rettungswache, dass es eine eigene IT-Abteilung gibt, welche die Software installieren und in das \acrlong{KIS} einpflegen muss.

Auch eine Forschungseinrichtung ist beispielsweise in einem Universitätsklinikum denkbar.
Dies ist ein weiterer, von den bisherigen Einrichtungen unterschiedlicher Anwendungsfall.
Hier sind weniger organisatorische Maßnahmen und Daten von Bedeutung, sondern die tatsächlichen medizinischen Daten.
Gleiches gilt für eigenständige Forschungseinrichtungen, welche etwa Daten zur Verfügung gestellt bekommt und diverse Studien durchführt.
Dieser Anwendungsfall ist nicht zuletzt auch bei \gls{GS} von Relevanz.
Daher gibt es eine eigene Abteilung \glqq Medizinische Forschung und Anwendung\grqq{}, welche sich mit forschungsorientierten Fragen und medizinischen Studien insbesondere in Anbetracht der hauseigenen Geräte auseinandersetzt.

Aus diesen Untersuchungen lassen sich drei wesentliche Einrichtungen und deren Aufgaben mit Daten festlegen, welche zurzeit oder voraussichtlich zukünftig Nutzer von \gls{ANALYSE} und der Erweiterung durch diese Arbeit sind. 
Diese werden in den darauffolgenden Kapiteln auf entsprechende Nutzergruppen weiter untersucht:
\begin{itemize}
 \item Rettungsdienst
	 \begin{itemize}
	 \item Administrative \& Organisatorische Aufgaben
	 \item Personelle und Materielle Planung
	 \end{itemize}
 \item Krankenhäuser und Kliniken
	 \begin{itemize}
	 \item Administrative \& Organisatorische Aufgaben
	 \item Forschung bei beispielsweise Universitätskliniken
	 \end{itemize}
 \item Forschungseinrichtungen
\end{itemize} 



%\subsection{Organisationen}
%Betroffene Organisationen sind unter anderem \todotext{orgas?}
%\subsubsection{Plätze?}

\section{Nutzergruppen}
\label{sec:nutzergruppen}
\epigraph{"`Ein Stakeholder ist eine Person oder Organisation, die einen direkten oder indirekten Einfluss auf die Systemanforderung hat"'} {Klaus Pohl \& Chriss Rupp \cite[S.29]{Pohl.2011}}

Bei Stakeholdern kann grundsätzlich zwischen zwei Gruppen unterschieden werden \cite{Leffingwell.2011}:
\begin{description}
\item [Requirements-Provider] \hfill \\
Diese Personen oder Organisationen sind in der Regel jene, welche die Software oder das Produkt im Tagesgeschäft nutzen beziehungsweise voraussichtlich nutzen werden.
Sie haben neue Ideen oder Verbesserungsvorschläge aus vorherigen Anwendungen.
Dies können auch interne Mitarbeiter der Firma sein, welche besondere Erfahrungen in den zu entwickelnden Bereichen haben. %Ref to internal Stakeholders somehow?
\item [Constraint-Provider] \hfill \\
Hierbei handelt es sich um Personen, welche keine Anforderungen als solche definieren, sondern die Umsetzbarkeit der Anforderungen von den Requirements-Providern testen und daraus folgend technische Rahmenbedingungen, wie zum Beispiel die zu verwendende  Programmiersprache oder Datenbank, empfehlen oder festlegen.
\end{description}

Eine \glqq Stakeholder-Liste\grqq{} ist ein hilfreiches Mittel, um einen Überblick an den beteiligten Personen zu erhalten und damit einhergehend so viele Anforderungen wie möglich an das System zu erheben (vgl. \cite[S. 83]{Bergsmann.2018}).
Dieser Prozess ist vor allem zu Beginn der Anforderungsanalyse von besonderer Bedeutung, da so frühzeitig viele Informationen und Hinweise erkennbar werden, wie das System aufgebaut werden muss.
Ein nachträgliches anpassen oder hinzufügen von Anforderungen kann unter Umständen schwere grundlegende Probleme mit sich ziehen, da gewisse fundamentale Architekturentscheidungen auf dieser Basis aufbauen können.
Ein Beispiel einer solchen Liste für \gls{ANALYSE} ist in Tabelle \ref{tbl:Stakeholder-Liste} zu sehen.

Die Requirements-Provider, fortan Nutzergruppen genannt, werden für das Erheben der Anforderungen hinzugezogen.
Sie sind ein essentieller, beeinflussender Faktor, wenn es um die Entwicklung einer Anwendung oder die Erweiterung einer Anwendung um ein umfangreiches Feature geht \cite[S. 125]{Herczeg.2018}.
Indirekt beschränken, aber auch erweitern sie die Funktionalitäten und beeinflussen die Art und Weise, wie die Anwendung gestaltet wird. 
Beispielsweise welche Erfahrungswerte vorausgesetzt oder angelernt werden müssen, welche Icons oder Symbole bekannt oder unbekannt sind, welche Ziele, Erwartungen oder Befürchtungen sie haben spielen eine große Rolle bei der Entscheidung, welche Informationen in welchem Format dargestellt werden.

%\todotext{Constraint-Provider: der Produktmanager, die Software-Entwickler, welche die neue Funktionalität implementieren und Ich?} 
Die Aufgaben der Constraint-Provider für die Erweiterung um die grafische Darstellung in \gls{ANALYSE} sind unter anderem wie oben genannt das Testen der Umsetzbarkeit.
Hierbei ist ein wesentlicher Bestandteil die Prüfung, ob eine Fragestellung mit den derzeit vorhandenen Daten der Geräte beantwortet werden kann.
Weitere Aufgaben sind unter anderem Entscheidungen über das Format und die Bereitstellung der Daten oder die Haltung und Persistierung in den entsprechenden Datenbanken.

Im Rahmen dieser Arbeit gibt es nicht die klassischen Anforderungen, wie beispielsweise \glqq ein Export sollte im CSV- und Zip-Format möglich sein\grqq. Die Anforderungen sind hierbei Fragestellungen welche beantwortet werden sollen, wie zum Beispiel \glqq Wie viele Reanimationen hat meine Wache?\grqq.

Daher ist zunächst zu klären, welche Nutzergruppen bei dem bisherigen Produkt vertreten sind.
Anschließend sollten Sie in Hinblick auf das zu erarbeitende Thema analysiert werden und daraufhin ist zu definieren, welche Eigenschaften oder Charakteristiken diese besitzen und ob gegebenenfalls neue hinzukommen oder alte hinfällig sind. 
%\todotext{Referenzierung oder Verbindung zu meinen Nutzergruppen, Stakeholdern. \\Abgrenzung nur Anwender,keine Deployer etc.?, }

\subsection{Stakeholder von \acrlong*{ANALYSE}}
\label{subsec:stakeholder}
Eine tabellarische Auflistung der Stakeholder soll, wie in \ref{sec:nutzergruppen} beschrieben, einen Überblick der Personen(-gruppen) liefern, die einen Einfluss auf das Produkt haben.
Somit soll gewährleistet werden, dass während der Konzeption und später auch bei der Umsetzung keine wichtigen Stakeholder vergessen werden, da dies schlimmere Nachwirkungen hat, je später sie bemerkt werden.


% not sure if I#ll keep this table or just make a Auszug of it and more text?
\begin{table}[htb]
\centering
\setlength{\extrarowheight}{4pt}
%\begin{tabular}{ |p{0.05\linewidth} | p{0.2\linewidth} | p{0.4\linewidth} | p{0.25\linewidth} |}
\begin{tabular}{ |p{0.5cm} | p{4cm} | p{5.5cm} |p{4cm} |}
  \hline
	\multicolumn{4}{|c|}{\textbf{Stakeholder}} \\
  \hline
\textbf{ID} & \textbf{Stakeholder-Bezeichnung} 	& \textbf{Organisation (Beispiel)} & \textbf{Gruppe}
  \\\hline
  1			& Geschäftsführer 					& \gls{GS} 				& Requirements-Provider
  \\\hline  
  2			& Entwickler 						& \gls{GS} 				& Constraint-Provider
  \\\hline
  3			& medizinische Forscher 			& Forschungseinrichtungen oder \gls{GS} & Requirements-Provider
  \\\hline
  4			& Leiter Rettungsdienst				& \gls{DRK}				& Requirements-Provider
  \\\hline
  5			& Produktmanager 					& \gls{GS} 				& Constraint-\& Requirements-Provider
  \\\hline
  6			& Klinikpersonal					& Kliniken			& Requirements-Provider
  \\\hline
  7			& Qualitätsmanagement 				& \gls{GS} 				& Requirements-Provider
  \\\hline
  8			& Ausbilder 						& Rettungsdienst 			& Requirements-Provider
  \\\hline
  9			& Systemadministrator 				& IT-Abteilung Krankenhaus				& Constraint-Provider
  \\\hline  
  10		& Tester 							& \gls{GS} 				& Constraint-Provider
  \\\hline
  11		& Rettungsdienstpersonal			& Rettungsdienst				& Requirements-Provider
  \\\hline
  12		& Vertrieb							& GS und Vertriebspartner	& Requirements-Provider
  \\\hline
\end{tabular} 
  \caption[Stakeholder-Liste \acrlong*{ANALYSE}]{Stakeholder-Liste von \gls{ANALYSE} nach Bergsmann \cite[S. 85]{Bergsmann.2018}}
  \label{tbl:Stakeholder-Liste}
\end{table}

Eine solche exemplarische Stakeholder-Liste ist in Tabelle \ref{tbl:Stakeholder-Liste} zu sehen.
Dabei wurde gegenüber der Liste von Bergsmann die Anzahl der Spalten verändert, da beispielsweise eine explizite Nennung der Kontaktperson für diese Arbeit nicht von Nöten ist.
In der Liste wird die entsprechende Bezeichnung, eine beispielhafte Organisation und die Einordnung, ob es sich um ein Requirements- oder Constraint-Provider handelt, festgehalten.
Hiermit ist es möglich, in den entsprechenden Phasen der Anforderungsanalyse, Konzeption oder Umsetzung die jeweils notwendigen Stakeholder zu berücksichtigen und gegebenenfalls zu kontaktieren. 
Diese Liste dient als Orientierung, sie ist nicht vollständig und kann beliebig erweitert, ergänzt oder eingeschränkt werden.
Auch eine höhere Abstraktion oder detailliertere Auflistung der Personen und Gruppen kann je nach Bedarf vorgenommen werden. 

Mithilfe dieser Tabelle und der Spalte \glqq Gruppe\grqq{} können beispielsweise die Nutzergruppen aus den gelisteten Stakeholdern extrahiert werden.
Diese werden im folgenden Abschnitt weiter erläutert.

\subsection{Nutzergruppen von \acrlong*{ANALYSE}}
\label{sub:NutzergruppenAnalyse}
In Tabelle \ref{tbl:Stakeholder-Liste} ist eine Auswahl an Stakeholdern von \gls{ANALYSE} zu sehen.
Diese kann hinzugezogen werden, wenn man zu einem Zeitpunkt der Planung oder Kozeptionierung eine bestimmte Gruppe dieser Personen miteinbeziehen möchte.
Für die Anforderungsanalyse der zu erstellenden \gls{Dashboard}s ist es sinnvoll zu wissen, was die Nutzer haben möchten.
Oder in diesem Fall welche Fragen sie haben, die womöglich mit den Daten der Geräte beantwortet werden können.

Um viele dieser Fragen zu erhalten, werden die Nutzergruppen von \gls{ANALYSE} benötigt.
Also jene Personen, die das Produkt oder die Daten, mit welchen das Produkt arbeitet, möglichst im Tagesgeschäft verwenden.
Aus der Liste \ref{tbl:Stakeholder-Liste} lassen sich die Personen oder Gruppen rauslesen, welche entsprechende Fragestellungen haben könnten. 
Wenn man nur die Requirements-Provider betrachtet, bleiben folgende relevante Nutzergruppen übrig:
\begin{multicols}{2}
\begin{itemize}
\item Geschäftsführer
\item medizinische Forscher
\item Leiter Rettungsdienst
\item Produktmanager
\item Mitarbeiter im Krankenhaus
\item Mitarbeiter des Qualitätsmanagement
\item Ausbilder von Rettungsdienstpersonal
\item Rettungsdienstpersonal
% \item Produktmanager
\end{itemize}
\end{multicols}

Aus dieser Liste von Personen und Gruppen kann ein weiterer Schritt der Reduzierung durchgeführt werden. 
So betrachtet man jene fünf Nutzergruppen, welche \gls{ANALYSE} sehr häufig verwenden oder voraussichtlich verwenden werden:
\begin{enumerate}
\item \textbf{Leiter Rettungsdienst} Ein Leiter Rettungsdienst oder auch ein Rettungswachenleiter hat die Verantwortung über einen Rettungsdienst und damit gegebenenfalls über mehrere Wachen.
Somit stehen unter ihm mehrere Personen in Schichtgruppen, die er koordinieren und einteilen muss.
Organisatorische Aufgaben wie Prozess- und Mitarbeitermanagement gehören zu seinem Tagesgeschäft.
Hierbei kann eine Datenspeicherung und daraus resultierende Auswertung und Analyse von Prozessen hilfreich und von besonderer Wichtigkeit sein.
\item \textbf{Forscher} Sie haben das Ziel, mit wissenschaftlichen Methoden neue Erkenntnisse zu gewinnen, Hypothesen aufzustellen und diese zu beweisen oder zu widerlegen.
Sie haben besonders viele Fragestellungen, die mit einfachen Daten wie Einsatzzeitpunkten oder komplexeren Daten, wie  die Entwicklung von medizinischen Vitalparametern, beantwortet werden können.
\item \textbf{Ausbilder von Rettungsdienstpersonal} Diese Gruppe bildet neues Personal nach entsprechenden Vorgaben im Rettungswesen aus.
Das Ziel ist eine umfangreiche und vor allem qualitativ hochwertige Ausbildung, da es um Patientenleben gehen kann.
Ein gewisses Controlling und daraus resultierendes Feedback an Einzelpersonen oder Gruppen ist von besonderer Bedeutung.
Mittels Datenauswertung kann dieses Feedback zielgenauer und spezifischer sein und so die Qualität der Ausbildung merklich verbessern.
\item \textbf{Qualitätsmanagementbeauftragte} Mitarbeiter in der Qualitätssicherung oder im Qualitätsmanagement sin verantwortlich dafür, dass gewisse Vorgaben kommuniziert und eingehalten werden.
Da sie meist bei größeren Trägern zum Einsatz kommen, ist die gebündelte Analyse von großen vorliegenden Datenmengen notwendig.
\item \textbf{ärztliches Personal} Diese Gruppe umfasst unter anderem Chef-, Not-, Tele-, Fachärzte, sowie auch weiteres klinisches Personal.
Sie können mittels einer Auswertung ihrer Daten entsprechende Therapiemaßnahmen evaluieren und etwaige Auswirkungen wahrnehmen.
\end{enumerate}


\subsubsection{Personas}
Eine Persona ist eine fiktive, spezifische Beschreibung einer Nutzergruppe.
Es ist ein \gls{Archetyp}, welcher eine Klasse von Nutzern zusammenfasst und so die Entwickler beim Entwerfen von Mensch-Computer-Anwendungen unterstützen soll.
Dabei werden die Charakteristiken von ein oder mehreren Nutzern kurz und prägnant zusammengefasst.
Beispielhafte Details können unter anderem die Berufsbeschreibung, Ziele, Erwartungen, Anforderungen sowie Alter und Hobbies sein.
Sie sollen den Entwicklern die zukünftigen Anwender vor Augen halten, damit eine angemessene Bedientauglichkeit der Software für alle Nutzergruppen gewährleistet wird (vgl. \cite[3.2, S.11]{Karwowski.2011,Pruitt.2006}).

Aus den in \ref{sub:NutzergruppenAnalyse} analysierten Nutzergruppen von \gls{ANALYSE} können mindestens fünf Personas erstellt werden.
Im Zuge der Entwicklung der \cweb-Produkte wurden bereits Personas angelegt.
Dabei sind aus \ref{sub:NutzergruppenAnalyse} die Nutzergruppen 2 - 5 bereits abgedeckt: \todo{personas anhang}
\begin{description}
\item[Nutzergruppe 2: Jörn]
	\glqq Der Reanimations-Wissenschaftler\grqq 
\item[Nutzergruppe 3: Bernd]
	\glqq Der Praxisanleiter\grqq 
\item[Nutzergruppe 4: Juliane]
	\glqq Miss Quality\grqq
\item[Nutzergruppe 5: Hermann]
	\glqq Der Land(Not)arzt\grqq
\end{description}

Folglich fehlt Nutzergruppe 1: Leiter Rettungsdienst.
Angelehnt an die zuvor erstellten Personas wird exemplarische eine neue Persona für die Nutzergruppe 1 erstellt:

\begin{table} [htb]
    \begin{tabular}{| p{4cm} | p{10cm} |}
    \hline
    \textbf{Avatar}                      & \parbox[c]{3em}{\center{ \includegraphics[width=0.3\textwidth]{img/lrd}}}\vspace{1em} \\
    \hline
    \textbf{Persona}                      & \makecell[cl]{\tabitem Name: Andreas\\\tabitem Alter: 47\\\tabitem Hobbies: Schach, Golf\\\tabitem Beruf: Leiter Rettungsdienst}                                           \\ \hline
    \textbf{Berufsbeschreibung}           & \makecell[cl]{\tabitem Einteilen der Personen in Schichtgruppen und \\Schichtpläne erstellen\\\tabitem Interne Prozesse überwachen und verbessern}            \\ \hline
    \textbf{Motivation \& Ziele}        & \makecell[cl]{\tabitem Ich möchte meinen Rettungsdienst stetig verbessern\\ \tabitem Zufriedenes Personal und gesunde Patienten sind meine\\ höchste Priorität} \\ \hline
    \makecell[cl]{\textbf{Anforderungen }\\ \textbf{\& Erwartungen}} & \makecell[cl]{\tabitem Ich möchte schnell sehen, welche Einsätze in meinen\\ Wachen passieren\\\tabitem Zeit um große Tabellen durchzuschauen habe ich nicht} \\ \hline
    \textbf{Schmerzpunkte}                	& \makecell[cl]{\tabitem Veraltete Software die mir die Zeit raubt\\\tabitem Unübersichtliche Darstellung von Daten}                                          \\ \hline
    \end{tabular}
    \caption[Persona: Leiter Rettungsdienst]{Exemplarische Persona für Nutzergruppe 1: Leiter Rettungsdienst}
  \label{tbl:Persona}
\end{table}

In der Regel wird einer Persona ein Avatar hinzugefügt, um dem Entwickler das Gefühl zu geben, für einen realen Menschen zu entwickeln. 
Dies wurde ebenfalls bei der entwickelten Persona in Tabelle \ref{tbl:Persona} berücksichtigt.


Es wurden die gängigen Beschreibungen wie bei den bereits vorhandenen Personas aufgeführt und entsprechend mit den Charakteristiken der Nutzergruppe \glqq Leiter Rettungsdienst\grqq{} versehen.
%\section{Aufgabenanalyse} ?

\section{Iterative Erhebung der zu beantwortenden Fragestellungen}
\label{sec:erhebung}
Im folgenden Abschnitt wird der Prozess der Erhebung von den Fragestellungen erläutert.
Hierbei wird die Methodik gemäß Requirements Engineering \cite{Pohl.2011} im iterativen Prozess angewandt.
Dieser Prozess geschieht zwar iterativ, zugunsten der Lesbarkeit dieser Arbeit werden die jeweiligen Schritte jedoch als ein Kapitel zusammengefasst.

\bild
{requireEngineering}
{12cm}
{Zyklus des Requirements Engineering. Bildquelle: \cite{Patig.}}
{Zyklus Requirements Engineering}

Demnach werden, wie in Abbildung \ref{fig:requireEngineering} zu sehen, im ersten Schritt die Fragestellungen ermittelt.
Wichtig hierbei ist die Berücksichtigung der verschiedenen Stakeholder des Produktes, welche in Tabelle \ref{tbl:Stakeholder-Liste} (S.\pageref{tbl:Stakeholder-Liste}) zusammengetragen wurden.

Anschließend werden die ermittelten Fragestellungen analysiert beziehungsweise geprüft.
Dabei sollen die vielen Daten aus dem vorherigen Schritt klassifiziert und strukturiert werden.
So lassen sich Redundanzen oder Diskordanzen vermeiden und aus der Häufigkeit der Vorkommnisse Prioritäten ableiten \cite[S.100f]{Sommerville.2012}.

Beim spezifizieren werden die geordneten Anforderungen üblicherweise in eine Standardform gebracht.
Für diese Arbeit ist jedoch die Spezifikation der benötigten Daten ebenso eine zielführende Handlung.
Dadurch wird die weitere Validierung, anschließende Umsetzung und Erarbeitung der Handlungshinweise vereinfacht und gewährleistet.

Im vierten Schritt aus Abbildung \ref{fig:requireEngineering} werden die spezifizierten Fragestellungen validiert.
Dies geschieht im fließenden Übergang zu Kapitel \ref{sec:evaluierung}, da hier geprüft wird, ob die spezifizierten Fragestellungen auch tatsächlich mit denen der Stakeholder übereinstimmen.
Damit soll garantiert werden, dass das entsprechende Endprodukt den Anforderungen der Stakeholder entspricht \cite{Patig.}.


\subsection{Ermittlung von Fragestellungen}
Um zu wissen, welche Fragestellungen beantwortet werden sollen, müssen diese an erster Stelle ermittelt werden.
Wie bei Anforderungen an eine Software werden hierfür die entsprechenden Stakeholder herangezogen.
Dabei werden primär adäquat qualifizierte Mitarbeiter der Firma \gls{GS} befragt, welche einer Stakeholder-Gruppe aus Tabelle \ref{tbl:Stakeholder-Liste} zugeordnet werden können.
Hierfür wurden mehrere geeignete Personen der Abteilungen \glqq Medizinische Forschung und Anwendung\grqq, Anwendungsspezialisten, Produktmanagement, Vertrieb und Kundendienst, sowie ein aktueller Kunde aus Dresden miteinbezogen. %\todotext{China?}

Zur Ermittlung werden zu Beginn Einzelinterviews durchgeführt. 
Diese haben der Vorteil, dass sich die entsprechende Person vollends auf die Thematik fokussieren kann.
Außerdem ist dadurch die Beeinflussung durch Dritte ausgeschlossen, sodass die alleinige Perspektive der entsprechenden Nutzergruppe gewährleistet ist.
Die Einzelinterviews haben eine durchschnittliche Länge von 60-90 Minuten und werden in einem Raum mit Computer und Beamer durchgeführt.
Mittels der genannten Technik ist es möglich, dem Interviewpartner zu Beginn Eindrücke zu gewähren, inwiefern Fragestellungen beantwortet werden können.
Dabei wird der Person eine Beispiel-App von Qlik präsentiert und die zugrundeliegende Fragestellung zu einer Visualisierung dargelegt, damit jedes Interview mit einer gleichen Wissensbasis vollzogen werden kann.
So ist es möglich dem Befragten eine Vorstellung zu geben, um welche Art von Fragestellung es sich handelt und welche realisierbar sind.

Anschließend werden eingangs offene Fragen gestellt und versucht, dem Interviewpartner wenig Input zu liefern, sondern so viel persönliche und nicht beeinflusste Informationen und Meinungen wie möglich zu erhalten.
Im Verlaufe des Gesprächs werden, falls notwendig, verschiedene Bereiche, welche die jeweilige Person noch nicht berücksichtigt hatte, nur genannt.
Auch Ideen von anderen Interviews werden bei einem fortschreitend kontextlosem Gespräch eingebracht, um etwaige übersehene Bereiche abzudecken oder neue Ideen hierfür zu erlangen. 

Im Anschluss werden die Notizen beider Beteiligten in digitaler Form möglichst unkomprimiert dokumentiert.
Ein exemplarischer Auszug ist hier zu sehen: \todotext{Anhang alles!}
\bildbreit
{notes}
{Kurzer Auszug der digitalisierten, möglichst unkomprimierten Notizen aus den Gesprächen mit Personen der \glqq Medizinischen Forschung und Anwendung\grqq}
{Auszug Notizen der Ermittlung von Fragestellung}



\subsection{Analyse der Fragestellungen}
\label{sub:analyseFragen}
Abbildung \ref{fig:requireEngineering} ist zu entnehmen, dass nach der Ermittlung die erste Phase der Analyse oder Prüfung der gesammelten Anforderungen ansteht.
Hier ist bereits ein Pfeil zurück auf den ersten Schritt zu erkennen.
Dies bedeutet, dass nach Abschluss der Analyse ein erneutes ermitteln Anforderungen mit den gleichen oder anderen Personen erfolgen kann.
Dabei können die gewonnenen Erfahrungen aus dem ersten Durchlauf mit eingebracht werden und so schneller bessere Ergebnisse erzielt werden.

%\todo{BILD IST WEG}
\bild
{listeanalyse}
{9cm}
{Analysierte Informationen strukturiert in Kategorien}
{Auszug analysierte Liste der Fragestellungen}
Bei der Analyse werden die unstrukturierten gesammelten Daten und Informationen grob vorstrukturiert.
Hierbei ist beispielsweise eine Einordnung in Kategorien eine hilfreiche Handlung, welche die Lesbarkeit enorm verbessert und die Komplexität auf mehrere Ebenen verteilt.
Durch diesen Schritt entstehen Kategorien wie Reanimation, Einsätze, Planung, Bedienung und viele mehr.

Ein weiterer Prozess ist die Vermeidung oder das Auflösen von Redundanzen und Diskordanzen.
Dabei werden ähnliche Informationen zusammengefasst und innerhalb der Kategorie nach oben verschoben.
Dadurch entsteht eine erste Priorisierung basierend auf der Häufigkeit, wie viele Personen die gleiche Information oder Fragestellung hatten.
Auch Diskordanzen werden eliminiert, indem beispielsweise die Auswertung nach einzelnen Mitarbeitern herausgenommen wird, da in einem Gespräch herauskam, dass dies als Leistungs- oder Mitarbeiterüberwachung beziehungsweise aus Datenschutzgründen rechtlich nicht trivial ist.

So entsteht eine Liste mit einer ersten Priorisierung in den Kategorien und ohne offensichtliche Duplikate oder Konflikte.
Ein Ausschnitt dieser Liste ist in Abbildung \ref{fig:listeanalyse} zu sehen.
\clearpage
\subsection{Spezifikation der Fragen und deren benötigte Daten}
\label{sub:spezifikation}
Mithilfe der in \ref{sub:analyseFragen} analysierten und strukturierten Liste ist es nun möglich, den nächsten Schritt gemäß Requirements Engineering durchzuführen: Die Spezifikation der Fragestellungen \cite{Pohl.2011}.

Für diese Arbeit bedeutet das in dem Fall zwei große Schritte: 
\begin{enumerate}
\item Zum einen müssen die Daten nach \cite{Patig.} in eine Standardform überführt werden. 
Da sie bisher aus den Interview-Notizen digital festgehalten wurden, sind es heterogene Formen wie Stichpunkte, Schlagwörter oder Fragen.
Eine zielorientierte Standardform wäre hierbei die Formulierung in Fragestellungen aus Sicht des Benutzers.
Dieser Vorgang wird in \ref{subsub:standardform} näher beschrieben.
\item Zum anderen sollten die standardisierten Fragestellungen auf die benötigten Daten überprüft werden.
Dabei können unterschiedliche Kategorien entstehen, wenn die Daten beispielsweise bereits im richtigen Format vorliegen oder aber die benötigten Daten zum aktuellen Zeitpunkt schlicht nicht vorhanden oder exportierbar sind.
So können etwaige Fragestellungen bereits ausgeschlossen oder höher priorisiert werden, wenn deren Daten bereits optimal vorliegen.
Auch ist es für später folgende Prozesse leichter, wenn die benötigten Daten spezifiziert sind.
Diese Maßnahme wird in \ref{subsub:datenspez} weiter erläutert.
\end{enumerate}

\subsubsection{Überführen der Informationen in eine Standardform}
\label{subsub:standardform}
Das Übertragen von Anforderungen oder Informationen in eine Standardform ist eine essentielle und wichtige Handlung, welche die Vergleichbarkeit und Handhabung der Anforderungen beziehungsweise Fragestellungen garantiert.

Als Standardform für diese Arbeit wurde \glqq Fragestellung aus Sicht des Benutzers\grqq{} gewählt, da jede Nutzergruppe (siehe S.\pageref{sub:NutzergruppenAnalyse} \ref{sub:NutzergruppenAnalyse}) Fragen hat.
Hierdurch ist eine klare Form erkennbar und weitere bisher unentdeckte Duplikate können weiter dezimiert und in Prioritäten überführt werden.
Auch für die spätere Entwicklung der Dashboards ist diese Form hilfreich, da gut überprüft werden kann, ob ein Diagramm eine oder mehrere Fragestellungen beantwortet.
Ein exemplarischer Auszug dieser Überführung ist in Abbildung \ref{fig:standardform} zu erkennen.
\bildbreit
{standardform}
{Übertragung von Informationen in die hier zielführende Standardform einer Fragestellung aus Sicht des Benutzers}
{Überführung von Informationen in Standardform}

\subsubsection{Spezifizieren der benötigten Daten}
\label{subsub:datenspez}
Ein für dieser Arbeit zusätzlicher Schritt bei der Spezifikation ist das Darlegen der benötigten Daten, um die entsprechenden Fragestellungen beantworten zu können.
Hierfür kann in drei Schritten vorgegangen werden:

\begin{enumerate}
\item \label{enum:grob} Grobe Einteilung der Fragestellungen mittels Farbgebung in vier Kategorien: % \todotext{4 oder 3 Kategorien sehr interessant?}
	\begin{enumerate}
	\item Rot - Nicht möglich: Diese Fragestellungen werden ausgeschlossen, da sie aufgrund von technischen oder rechtlichen Aspekten nicht beantwortbar sind.
	\item Gelb - Fraglich: Die Datengrundlage oder andere Aspekte sind zurzeit ungewiss und müssen geklärt werden
	\item Grün - Daten vorhanden: Hierfür sind die Daten garantiert bereits vorhanden und diese Fragen können beantwortet werden
	\item Blau - Sehr interessant (optional): Daten sind ebenfalls vorhanden und zusätzlich werden die Fragestellungen mit hoher Priorität hier eingeordnet
	\end{enumerate}
\item Visuelles kategorisieren der eingeteilten Fragestellungen in die oben genannten Kategorien. 
Zusätzlich kann eine erste Einschätzung der Komplexität beziehungsweise Ein- oder Mehrdimensionalität vorgenommen werden
\item Recherchieren der Datenquellen
\end{enumerate}

Analog zu Schritt \ref{enum:grob} wird nun zuerst eine grobe Kategorisierung der standardisierten Fragestellungen vorgenommen. 
Dabei werden unter anderem Entwickler oder Produktmanager befragt oder eigenes Wissen angewandt.
Vorteil hiervon ist die Aussortierung von nicht umsetzbaren Fragestellungen, was den späteren Aufwand reduziert.
Der Abbildung \ref{fig:einteilung2} kann ein Auszug dieser Einteilung entnommen werden.
Zusätzlich zu der Farbgebung werden rechts zugehörig zur Kategorie entsprechende Bemerkungen festgehalten, sofern ein oder mehrere Fragestellungen fraglich oder nicht möglich sind.
\bildbreit
{einteilung2}
{Auszug der farblichen Kategorisierung nach Umsetzbarkeit der Fragestellungen}
{Kategorisierung nach Umsetzbarkeit der Fragestellungen}

Der nächste Schritt gruppiert die zuvor kategorisierten Fragestellungen.
Dies geschieht in einer Tabelle und wird visuell durch die entsprechende Kategorie-Farbe unterstützt.
Des Weiteren wird eine erste Einschätzung der Dimensionalität vorgenommen.
Dies soll für die weitere Entwicklung eine Hilfestellung darbieten, damit schnell ersichtlich ist, wo Klärungsbedarf ist oder welche Fragestellungen bereits beantwortet werden können.
In Abbildung \ref{fig:viseinteilung} kann die große Tabelle grob erkannt werden.
Die Reihen sind die Einschätzungen der Umsetzbarkeit und die Spalten die Kategorien aus \ref{sub:analyseFragen}.
\bildbreit
{viseinteilung}
{Überblick über die Tabelle zur Gruppierung der Fragestellungen nach Umsetzbarkeit und Kategorie}
{Überblick über die Tabelle zur Gruppierung der Fragestellungen}

Im dritten, ergänzenden Akt werden bereits die benötigten Datenquellen recherchiert und spezifiziert.
Hierfür werden bereits vorhandene Daten und deren Quellen den entsprechenden Fragestellungen zugeordnet.
Bei den unklaren Fragen werden Ideen, Vorschläge und weitere Kommentare hinzugefügt.
Damit ist die Grundlage zur Erstellung eines Prototypen und später auch bei der Umsetzung gelegt.
Daten von Fragen, die beantwortet werden können, liegen dokumentiert vor.
Die fraglichen Anforderungen sind mit ebenfalls mit ihren bisherigen Ideen und Informationen aufgelistet. 

%\todotext{bild?}


%\bildrechts
%{komplexitaet}
%{6cm}
%{Auszug zur Einschätzung der Komplexität}
%{Überblick über die Tabelle zur Gruppierung der Fragestellungen}
%In Abbildung \ref{fig:komplexitaet} sds

\subsection{Validierung der spezifizierten Fragestellungen}
Die Validierung der ermittelten, analysierten und schließlich spezifizierten Fragestellungen ist der letzte Schritt im Requirements Engineering, was nicht bedeutet, dass es damit endet.
Dieser geschieht, wie auf S.\pageref{sec:erhebung} beschrieben, im fließenden Übergang mit der Evaluierung des Prototypen in \ref{sec:evaluierung}.
Hier wird geprüft, ob die in \ref{sub:spezifikation} spezifizierten Fragestellungen den Erwartungen der Nutzergruppen entsprechen.

Alle Schritte zuvor wurden, wie in Abbildung \ref{fig:requireEngineering} zu sehen, iterativ und teilweise parallel durchgeführt.
Es können jederzeit neue Befragungen und somit Ermittlungen von Fragestellungen durchgeführt werden und mit den neu erhobenen Daten die zuvor beschriebenen Schritte erneut iterativ vollzogen werden.
Im Rahmen dieser Arbeit wurde ab einem bestimmten Zeitfenster die Anforderungsanalyse abgeschlossen, damit ein Prototyp (\ref{sec:erstellungPrototyp}) und anschließend eine präsentierfähige Version (\ref{kap:umsetzung}) auf Basis der bis dato spezifizierten Fragestellungen erstellt werden kann.
\chapter{Konzeption}
\label{konzept}

\minitoc\pagebreak
%\lipsum[1-60]

\section{Vorgehen bei der Konzeption}

%\bildrechts{iterative} {2cm} {it} {iter}

Gemäß der Methoden des Requirements Engineering \ref{require} wird anhand der erhobenen und ausgearbeiteten Fragestellungen bzw. Anforderungen aus \ref{4.4 fragestellungen} ein Konzept erarbeitet. 
Auch hierbei wurde eine iterative Herangehensweise gewählt, unter anderem um in einer kurzen Zeit grobe Fehler zu erkennen und viele Verbesserungen herauszufinden. 
Auch wird hierdurch gewährleistet, dass die exemplarischen und später produktiven Auswertungen und Visualisierungen nicht gänzlich in eine falsche Richtung gehen, sondern von Anfang an etwaige Missverständnisse aufgedeckt werden oder grundlegende Entscheidungen frühzeitig überdacht werden können.

Um von vornherein ein realistisches "`Look-and-Feel"' zu bieten und die entsprechend geplante Technologie mit ihren Funktionen und der Nutzerführung zu evaluieren, wird beim Prototyp bereits Qlik Sense \ref{technologieqlik} verwendet. 
Ein weiterer Vorteil ist, dass das notwendige Einarbeiten für die spätere Umsetzung in diese Technologie bereits gewährleistet wird. 
Dabei können auch vorab verschiedene Vor- und Nachteile der Technologie herausgefunden werden und dementsprechend können diese bei der weiteren Planung und Umsetzung berücksichtigt werden, was viel Zeit sparen kann.

Des Weiteren wurden Dummy-Daten unter anderem für den Prototypen generiert. 
Sie waren auch notwendig, um entsprechend andere Tests für die Datenbank und Auslastung durchzuführen \ref{db und lasttest}. 
Dabei wurde ein Python-Skript geschrieben, welches einige Daten zufällig generiert, jedoch auf einer realistischen Basis.

Die Evaluation soll wie oben beschrieben iterativ sein, was mehrere Termine mit unterschiedlichen Experten zur Konsequenz hat. 
Im Zuge dieser Bachelorarbeit wird die Evaluation größtenteils? mit Mitarbeitern der Firma \gls{GS} durchgeführt. 
Es gibt (viele) Abteilungen, in welchen genügend Fachleute sitzen, die in der Lage sind diese Dashboards kritisch prüfen zu können. 
Hierbei wird darauf geachtet, unterschiedliches Personal miteinzubeziehen, sodass es keine exklusive/alleinige Evaluierung durch die Abteilung medizinische Forschung ist, sondern auch Vertriebsmitarbeiter oder andere Stakeholder sollen den Prototyp kritisch untersuchen.

Bei der Konzeption ist auch das Datenmodell ein wichtiger Bestandteil. 
Hier können verschiedene Modelle ausprobiert werden und gegebenenfalls Änderungen vorgenommen werden, sollten sich bei der Evaluation Probleme oder andere Anforderungen herausstellen.


\section{Erstellung eines Prototypen}
\subsection{Verwendete Technologie}
Für den ersten Prototypen wurde sogleich die letztendlich umsetzende Technologie Qlik Sense \ref{qlik} verwendet. 
Der Einfachheit halber beschränkte sich dies zu Anfang auf Qlik Sense Desktop, einerseits aufgrund der vereinfachten Bedienung, hauptsächlich jedoch da der entsprechende Server mit der Qlik Sense Server Variante zu Anfang der Konzeption noch nicht eingerichtet war. 
Ein Umzug von Qlik Sense Desktop zu Server stellt keine größeren Probleme dar, man kann die erstellten Apps bidirektional ex- und importieren. 
Lediglich die Datenverbindungen, welche für die jeweiligen Applikationen die Daten aus einer Quelle beziehen, werden nicht fehlerlos übertragen.

\subsection{Datengrundlage}
Zu Beginn wurde überlegt, welche Daten die Grundlage für den Prototypen bilden sollte.
Dabei wurde auf die Demo-Datenbank des internen\cweba - Servers zurückgegriffen, welche zu diesem Zeitpunkt in etwa 100.000 Missionen enthielt, von denen jedoch grundsätzlich alle nur Test-Missionen sind. 
Dies sind solche, die entweder von der internen Testabteilung oder der Software-Entwickler beim Testen von Funktionen oder Herausfinden von Problemen am Gerät oder durch die Software erzeugt werden. 
Dementsprechend enthalten sie wenige spannende Daten, die auch selten realistisch sind und etwaige neue Erkenntnisse nicht darlegen können. 
Nichtsdestotrotz bilden sie eine gute Grundlage um die ersten Basis-Auswertungen von Missionen paradigmatisch darzustellen. 

Für diesen Zweck kann die Export-Funktion im CSV-Format von Analyse genutzt werden, welche in  \ref{istzustandanalysezurauswertung} näher beschrieben ist. 
So war es für den Anfang möglich, einen relativ großen Datensatz mit mäßig sinnvollen Daten zu erhalten.\footnote{(In späteren ?Szenarios\ref{lasttests,db} wurden eigens generierte Missionen verwendet, auf die Erzeugung dieser wird in \ref{dummydaten} genauer eingegangen.)}



\subsection{Erstellung exemplarischer Dashboards} 
Mit der Datengrundlage und der Technologie Qlik Sense Desktop können nun die ersten beispielhaften Dashboards entwickelt werden.
Die Basis für die verschiedenen Diagramme und Auswertungen bilden die iterativ erhobenen Fragestellungen aus Kap \ref{itErhebFrage}. 
Der Ansatz hierbei ist, mit einer grafischen Darstellung der Daten so viele Fragestellungen wie möglich beantworten zu können.
Dabei werden für die Art und Weise der Darstellungen verschiedene Aspekte berücksichtigt, wie zum Beispiel ob es ein zeitlicher Verlauf ist oder tendenziell eine Momentaufnahme, absolute gegen relative Kennzahlen, Veränderungen oder Trends und vieles mehr.
Auf Basis dieser Aspekte wird eine möglichst passende Darstellungsform gewählt und mit entsprechender Dimension und Kennzahl, eventuell auch mehrere Dimensionen und/oder Kennzahlen, gefüllt.

Eine sinnvolle Gruppierung oder Aufteilung der entsprechenden Arbeitsblätter oder Diagramme ist im Zuge des Prototyps von keiner hoher Priorität. 
Im Fokus steht die erstmalige Beantwortung möglichst vieler Fragestellungen, auch gegebenenfalls auf verschiedenen Wegen mit alternativen Darstellungsformen.
Nichtsdestotrotz wird später, zum Zeitpunkt der Evaluierung, eine einigermaßen angebrachte Gruppierung von Diagrammen und vernünftige Reihenfolge der Arbeitsblätter vorgenommen, damit für die entsprechenden Personen eine Struktur erkennbar ist, um mögliche Verwirrungen zu vermeiden.


\subsubsection{Datenladeskript}
, Was ist rel, was test, ....
\subsubsection{Datenmodell}
recht simpel. im prinzip eine Tabelle, keine Mehrdim, ...
% Vermutlich nur Umsetzung, vielleicht auch hier

%subsubsection Dashboards, und dann unterpunkte?
\subsubsection{Startseite}
\bildbreit
{Uebersicht_Prototyp}
{Startseite, Übersicht}
{Überblicks}

Die erste Seite, die der Kunde zu sehen bekommt wenn er die Software startet, soll ihm einen groben Überblick verschaffen. 
Es ist wie die Startseite einer Website, wo man die wichtigsten Informationen unmittelbar auf der ersten Seite finden kann.
Die Abbildung \ref{fig:Uebersicht_Prototyp} zeigt die erste Version dieser Startseite.

Sie wurde recht simpel mit vier Diagrammen und zwei Kennzahlen gefüllt, damit der erste Eindruck nicht von einer Informationsflut negativ beeinflusst wird.
Sollte der Nutzer weiterführende Informationen und tiefgreifendere Analysen durchführen wollen, kann er diesen Ansprüchen auf den folgenden Dashboards gerecht werden.
%\bildrechts{waterfall} {4cm} {w} {wa}
Das Wasserfall-Diagramm ganz links zeigt die absolute Anzahl an Einsätzen, die all seine Geräte durchgeführt haben.
Des Weiteren werden die absoluten Zahlen der verschiedenen Einsatzarten in Relation zur Gesamtmenge grafisch dargestellt.
Hierbei wird unter anderem zwischen Testeinsätzen, Reanimationen oder sonstigen Einsätzen unterschieden.
Bei den Reanimationen gibt es drei weitere Unterarten: (bulletlist?) Mit \& Ohne Feedbacksensor oder eine mechanische Reanimation.
Die Farbgebung soll Teilsummen von Gesamtmengen unterscheiden.
Ziel der Darstellung ist ein visueller Eindruck, wie viele Einsätze es gibt und welche Arten von Einsätzen in welchem Maße vorkommen.

Das Liniendiagramm stellt den zeitlichen Verlauf der Einsätze dar.
Als (zeitliche) Dimension wurde hier Jahr-Monat gewählt, eine Alternative wäre die tatsächliche Datumsangabe, dies würde jedoch zu einer Kurve führen, welche viele Zacken enthält (Vergleichsbild?) und damit sehr unruhig wirkt.
Die Aufsummierung in Monate, alternativ auch Wochen, bringt je nach Datenlage eine recht glatte Kurve mit sich. 
Die Kennzahlen dieser Visualisierung sind die Summe der relevanten Einsätze und zum Vergleich die Summe der Reanimationen. 
Eine Mini-Legende unterhalb der Grafik erleichtert die Orientierung, sollte der Nutzer in einen bestimmten Zeitraum "`reinzoomen"'.

%Es wurde die AutoCalendar-Funktion genutzt, damit ein fließender Übergang zwischen den Jahren möglich ist.

Darunter sind zwei Kreisdiagramme zu sehen, welche die Anzahl der Einsätze und der Reanimationen pro Gerätetyp anzeigen.
Diese Art der Visualisierung wurde gewählt, damit die Relation und Verteilung unter den Geräten deutlich gemacht wird.
Unter dem jeweiligen Kreisdiagramm ist als einzelner \gls{Key-Performance-Indicator} die \gls{KPI} die durchschnittliche Einsatz- und Reanimationsdauer.

% Filter?
\subsubsection{Einsatzzeitpunkt}
\bildbreit
{einsatzzeitpunkt}
{Dashboard zu den Einsatzzeitpunkten}
{Einsatzzeitpunkt-Visualisierung}

Es gab diverse Fragestellungen zu den Einsatzzeitpunkten. (Beispiele oder ref?)
Dieses Arbeitsblatt, in Abbildung \ref{fig:einsatzzeitpunkt} zu sehen, soll einen Überblick geben, zu welchen Tages- und Uhrzeiten die Einsätze stattfinden.
Hierbei gibt es drei verschiedene Detailstufen?: Tageszeit, Stündlich und Halbstündlich (bullet?). 
So kann der entsprechende Nutzer für seinen gewollte Bedarf die jeweilige Auslastung oder Einsatzhäufigkeit herausfinden.
Des Weiteren sind unten, unterhalb der Tageszyklen (Tageszeit und Stunden), zwei weitere Diagramme, welche die Anzahl der Einsätze und Reanimationen für Wochentage und Monate anzeigt. 
Somit ist jeder relevante zeitliche Zyklus abgedeckt und es können beispielsweise zu Tageszeiten wie nachts, Wochentage wie Wochenende und/oder saisonale wie winterliche Auswertungen betrieben werden.

Es wurden zur Darstellung Kombi-Diagramme gewählt, um normale Einsätze und Reanimationen getrennt, aber dennoch in Relation betrachten zu können.
Die Balken eignen sich gut, um das Volumen eines Zeitpunktes mit anderen leicht Vergleichen zu können und repräsentieren die Menge an relevanten Einsätzen.
Die Anzahl der Reanimationen wurde als Linie mit einer alternativen Y-Achsen-Skalierung visualisiert, damit auch geringe Vorkommen, wie in der Praxis häufig der Fall, noch gut sichtbar sind, da die obere und untere Grenze unabhängig der Anzahl aller Einsätze ist.
Würde man diese als Balken neben die normalen Einsätze legen, wären sie kaum zu sehen und Unterschiede zwischen den jeweiligen Zeitpunkten wären nur schwer zu erkennen.
Mit der alternativen Liniendarstellung ist die untere Kante der X-Achse nicht immer 0, wie es bei der Darstellung von Balken der Fall ist, sondern kann beim Minimum der entsprechenden Kennzahl beginnen.
%Timebuckets wurden selber geskriptet, damit nicht zu fein nach minute ?

\subsubsection{Einsatzdauer}
\bildbreit
{einsatzdauer}
{Dashboard zu der Dauer von Einsätzen}
{Einsatzdauer-Visualisierung}

Die Dauer von Einsätzen ist ebenfalls eine gefragte Information. (ref?)
Hierbei sind Fragen wie 
Je nach entsprechenden Vorgaben können hierbei unterschiedliche Auswertungen durchgeführt werden.
Sollen die Rettungskräfte beispielsweise das Gerät immer sofort beim Losfahren starten, können gegebenenfalls Auswertungen zu den Fahrtzeiten getroffen werden.
Oder wenn das Personal das Gerät erst vor Ort anschalten soll, kann die tatsächliche Einsatzzeit verglichen werden.
Es kommt also auf die Vorgaben der entsprechenden Instanzen drauf an, welche Analysen durchgeführt werden können.

Als generelles Modell werden auf diesem Dashboard, in Abbildung \ref{fig:einsatzdauer} zu sehen, die grundlegenden Informationen zur Dauer dargestellt.
Dazu zählt die globale durchschnittliche Missionsdauer, sowie die absolute Gesamteinsatzdauer, welches unter anderem eine durch die Bundeswehr gefragte \gls{KPI} ist.
Des Weiteren gibt es die mittlere Einsatzdauer nach Gerätetyp aufgeschlüsselt. 
Dies ist zugehörig unter der Geräteübergreifenden \gls{KPI} als Balkendiagramm dargestellt, damit die Werte gut miteinander vergleichbar sind.


Darunter sind zwei weitere Diagramme zu finden, welche als Histogramm (irgendwo erklären) die Häufigkeit verschiedener Einsatz- sowie \gls{CPR}-Dauer darstellen.
%\bildrechts{Dauer-klasse} {5cm} {w} {wa}
Dabei werden je nach maximaler Dauer dynamisch gleichgroße Bereiche oder ''Klassen'' definiert.
So gibt es beispielsweise den Bereich ''0 <= Reanimationsdauer in Min. < 16'', welcher hier >4000 Einsätze zählt, danach Reanimationsdauer größer gleich 16 und kleiner als 32 mit ca. 400 Missionen.
Dies wird in diesen 16-Minuten-Schritten weiter bis zur maximal vorhandenen Reanimationsdauer fortgeführt.
Dabei wird die Verteilung von Einsätzen und Reanimationen mit einer entsprechenden Dauer sichtbar und es können weiterführende Analysen auf Basis der Einsatzdauer vorgenommen werden.

\subsubsection{Geräte}
\bildbreit
{devices}
{Dashboard zu den entsprechend vorhandenen Geräten}
{Geräte-Visualisierung}

Das folgende in Abbildung \ref{fig:devices} präsentierte Arbeitsblatt gibt dem Nutzer eine Übersicht seiner eingesetzten Geräte.
Dabei gibt es die eindeutige \gls{KPI} ''Anzahl Geräte'', welche die distinkten Geräte zählt, von denen bis dato Missionen in dem vorliegenden\cweba - Server vorliegen.
Sofern der entsprechende Betreiber den Upload oder das nachträgliche importieren der Einsätze auf den Server anordnet, kann hiermit die gesamte Anzahl an Geräten überwacht werden. (Aktive Geräte? bsp Einsatz in letztem quartal? In umsetzung aufnehmen)
Ergänzend zu dieser (pauschalen) Zahl gibt es in intuitiver Leserichtung rechts eine Visualisierung, wie viele Geräte einer Art vorliegen.
Die Art der Darstellung ist ein Kreisdiagramm, da in der Regel eine Geräteart überwiegen wird, und somit diese in Relation zu der Gesamtheit gesetzt wird.

Darunterliegend finden sich zwei Balkendiagramme wieder, welche die Anzahl der Einsätze und Reanimationen nach den einzelnen Geräten mittels Seriennummer und/oder Geräte-ID aufschlüsseln.
Dies präsentiert die Einsatzhäufigkeit von den Einzelgeräten und kann dadurch beispielsweise als Grundlage für Schichtplanung oder Neuanschaffung von Geräten dienen.

\subsubsection{Reanimation}
\bildbreit
{reanimation}
{Dashboard zu den Daten des \gls{CPR}-Feedbacksensors}
{Reanimations-Visualisierung}

Die nachträgliche Auswertung einer Reanimation ist für viele Stakeholder eine wichtige Aufgabe.
Dabei gibt es beispielsweise die klassische Nachbesprechung zwischen Notfallsanitäter(erklärung Rettungssani?) und Auszubildenden oder aber auch die medizinische Forschung, welche Ereignisse und Werte einer Reanimationen untersuchen um gegebenfalls neue Erkenntnisse zu gewinnen.
Eine Analyse über viele Reanimationen hinweg ist hierbei für viele Nutzergruppen (außer MRaA) ein neuer Weg, welcher bisher unentdeckte Zusammenhänge offenbaren oder bereits angenommene Hypothesen bestätigen kann.

Ein wichtiger Erfolgsfaktor einer Reanimation ist eine adäquate Drucktiefe, welche zwischen 5-6cm liegen soll. (ref erc?)
Sofern ein \gls{CPR}-Feedbacksensor (erklären?) bei einer Reanimation verwendet wird, kann die Drucktiefe jeder Kompression eingesehen werden. (auch vorOrtFeedback)
Eine Auswertung zur Tiefe über viele Reanimationen hinweg kann im obersten Balkendiagramm in Abbildung \ref{fig:reanimation} betrachtet werden.
Dabei sind horizontal die verschiedenen Drucktiefen und vertikal die Anzahl der Missionen mit dieser Tiefe angeordnet.
Die Farbgebung mit rot und grün soll dem Benutzer helfen, schnell die ''guten'' Kompressionen von den unzureichenden zu unterscheiden.

Ein wichtiger Punkt hierbei ist, dass derzeit pro Reanimation genau ein Durchschnittswert der Drucktiefe gebildet wird und zum Export bereitsteht.
Es ist somit eine stark voraggregierte Information, welche einen eventuellen falschen Wert widerspiegelt. 
So kann beispielsweise eine Reanimation 100 zu flache Kompressionen mit 4cm, sowie 100 zu tiefe Kompressionen  mit 7cm enthalten.
Der nun aggregierte Mittelwert dieser Reanimation lautet 5,5cm und suggeriert eine an der Drucktiefe gemessene, ''perfekte'' Reanimation, obwohl nicht eine korrekte Kompression vorliegt.
Dieser Aspekt muss bei der Betrachtung dieser Auswertung berücksichtigt werden.
Eine mögliche Lösung oder zum Mindesten eine Verbesserung der Aussagekraft zur Drucktiefe wird in Kap blabla beschrieben.
%cpr nur avg, keine einzel cpr, gefordertes neues Modell, wie in kap 123 beschrieben 

Zur rechten Seite der oben beschriebenen Visualisierung ist ein Boxplot-Diagramm, welches eine Zusammenfassung zur Drucktiefe liefert.
Es zeigt kompakt die maximale und minimale durchschnittliche Tiefe, sowie die Quartile und den Median aller Reanimationen an.

%ccf, frequenz, 

\subsubsection{Schocks}
\bildbreit
{shocks}
{Dashboard zu den abgegebenen Defibrillationen}
{Schocks-Visualisierung}

%schocks die nicht 200j hatten, leider nur max schock, keine einzel shcocks, gefordertes neues Modell, wie in kap 123 beschrieben 
gh

\subsubsection{Sensorik}
sd
\subsubsection{Blutdruck}

%nibd, leider nur min max avg, keiine einzel messungen, gefordertes neues Modell, wie in kap 123 beschrieben 
sd

\subsubsection{Patientendaten}
sd
\subsubsection{Pacer?Sonstiges}
sd


\section{Generierung von Dummy-Daten?}
% Kap. Umsetzung?

\section{Evaluierung  des Prototyps}
\subsection{Vorgehen bei der Prototyp-Evaluierung}
Zur Evaluierung des Prototyps werden vorerst, wie in \ref{vorgehenkonzept} beschrieben, firmenintern geeignete Personen gesucht. 
Dabei wurden Mitarbeiter aus den Abteilungen Medizinische Forschung und Anwendung, Applikations- bzw. Anwendungsspezialisten und Produktmanagement gewählt, da hier ein guter/großer Praxisbezug herrscht und die Erfahrungen, Wissen und Anforderungen ähnlich zu denen der zukünftigen Kunden sind.

%\citeDürrenberger für Fokusgruppen!!

Für die Evaluierung wurde eine (abgewandelte Form) Art der Fokusgruppen-Evaluation gewählt. \cite{Christoforakos.2017} 
Hierbei wird eine kleinere Gruppe von Teilnehmern zusammengestellt, welche die voraussichtlichen Kunden generell abbilden soll. 
Danach ist es im Grunde genommen eine offene Gruppendiskussion, welche durch einen Moderator geleitet wird, sodass der entsprechende Fokus nicht außer Acht gelassen wird und etwaige Diskussionen und Gedankenaustausche in die richtige Richtung gelenkt werden. \cite{Durrenberger.1999} vgl. \cite{UsabilityinGermany.}
In diesen Beispielen spielt die Beobachtung der Teilnehmer keine große Rolle, wie normalerweise in den meisten Fokusgruppen üblich, da sie dass System als solche nicht bedienen und in diesem Fall aus der Beobachtung keine großen Erkenntnisse gewonnen werden können.

Die Vorteile einer Fokusgruppe sind, dass sich Personen in diesem Gebiet bereits auskennen und direktes Feedback geben können. 
So ergibt sich in relativ kurzer Zeit eine große Datenmenge und kritische Rückmeldungen zum Prototyp, welche bei Bedarf im gleichen Zug weiter ausgeführt und/oder diskutiert werden können.
Auch möglicherweise auftretende Fragen von den Personen können, im Gegensatz zu reinen Umfragen, direkt beantwortet werden um so Missverständnisse und Unklarheiten aufzuklären.

Für die Evaluierung wurde eine Gruppengröße von ca. 3-4 Personen gewählt, was etwas unter der üblichen Fokusgruppengröße von ca. 6-8 Personen liegt (vgl. \cite{UsabilityinGermany.})?. 
Grund hierfür sind zum einen die erwünschten 3-4 unterschiedlichen Gruppen, welche bei den ca. 16 geeigneten Personen diese Gruppengröße vorgibt, zum anderen ist es auch von Vorteil was das Einbringen von Ideen und Kritik von jeder Einzelperson angeht.
Dennoch kann ein fachlicher Austausch zwischen verschiedenen Personen stattfinden, welcher für weiterführende Diskussionen sehr hilfreich sein kann. 

Bei der Einteilung der Personen in die jeweiligen Fokusgruppen wurde auf verschiedene Eigenschaften und Bedingungen, wie zum Beispiel Alter, Abteilung, Wissenstand, Erfahrungen, und vieles mehr geachtet. 
Ziel waren einerseits homogene Gruppen, wo die Personen ähnliche Hintergründe und Altersklassen haben, andererseits sollte auch eine gewisse Heterogenität herrschen, damit verschiedene Meinungen aufeinander treffen und kontroverse Diskussionen angeregt/gefördert werden.
So ist beispielsweise eine Gruppe von eher jungen medizinischen Forschern in einer ähnlichen Altersklasse zwischen 23-27 Jahren, aber mit verschiedenen Hintergründen, beziehungsweise Spezialisierungen zu Informatik, Rettungsdienst und medizinische Studien und Auswertungen.

Mit der kleineren Größe der Gruppe wurde auch ein etwas kürzerer Zeitraum der jeweiligen Evaluierung gewählt.
Demnach liegt sie (des Öfteren) bei einer Gruppe von 6-8 Personen bei ca. 90 Minuten (vgl. \cite{UsabilityinGermany.})? und für die hier durchgeführten Termine wurden 60 Minuten angesetzt, was pro Person gerechnet sogar etwas mehr Zeit.
Jedoch ist diese Rechnung/Vergleich nicht repräsentativ und daher mit Vorsicht zu betrachten.

Es werden die erstellten Dashboards aus \ref{erstellungprototyp} in der Gruppe vorgestellt. 
Dabei wird anfangs den Personen erläutert, was der Sinn und Zweck der Visualisierungen ist und für welche Endnutzer er von Relevanz sein wird.
Somit können sie sich in die Lage der Kunden versetzen und aus deren Perspektive die präsentierten Ergebnisse kritisch betrachten.

Anschließend wird jedes Arbeitsblatt präsentiert, gefolgt von einer kurzen Pause, damit sich der erste Eindruck bilden kann und darauffolgend eine kurze Einführung/Erklärung zum aktuellen Fenster geliefert, um eine einheitliche Diskussionsbasis zu schaffen.
Des Weiteren wurden Hand-Outs ausgehändigt, damit jede Person zu jeder Zeit jedes Dashboard vor sich liegen hat um gegebenenfalls Anmerkungen, Kommentare, Verbesserungsvorschläge, Fragen oder ähnliches an die entsprechende Stelle notieren zu können.
Dies hat den weiteren Vorteil, dass es zur Nachbereitung verwendet werden kann, da die Gedanken und Ideen der Teilnehmer im Nachhinein dokumentiert zur Verfügung stehen und weitere Schritte oder Änderungen darauf basierend vorgenommen werden können. 
Ein Auszug dieser Hand-Outs können im Anhang \ref{evaluierunghandout} betrachtet werden.
% Die dargestellten Abbildungen können von \ref{konzeptERstellung} abweichen, da im iterativen Prozess immer Änderungen durchgeführt wurden, und somit keine zeitliche Reihenfolge gegeben ist.

\subsection{Gruppe 1: medizinische Forschung und Anwendung}
j
\subsection{Gruppe 2: Applikationsspezialisten}
j
\subsection{Gruppe 3: Produktmanagement und medizinische Forschung und Anwendung}
j


\subsection{Analyse der Ergebnisse}
%notwendig? oder in jeweil. subs integrieren

\chapter{Umsetzung}
\label{umsetzung}
\minitoc\pagebreak

%Evtl. Reihenfolge anders?
\section{Erstellung der Qlik-Apps}
%subs umsetzung der evaluierungsergebnise
% Zielgruppenunterschiedliche Apps/Startseiten
\subsection{ETL-Prozess}
%subsubs Datenmodell?
\subsection{Dimensionen}
\subsection{Kennzahlen}
\subsection{Dashboards}
%subs testeinsätze filtern
%subsubs lesezeichen?
% Einstellungen (wie zB. Farben bei Auswahl beibehalten)
%CustomThemes?

\section{Technische Aspekte}
% Added new Data models like cpr, shocks, nibp!! as subsections!
\subsection{Schnittstelle ANALYSE und Qlik}
json format cpr (siehe mails) (Fotos whiteboard als anhang?),
null values bei cpr trends
\subsection{Incremental Load?}
\subsection{Datenbankhaltung?}
\subsection{JIRA-Stories}
\subsection{Lasttests?}
\subsection{Auslieferungsprozess?}
%subs internationalisierung

\section{Rechtliche Aspekte}
\subsection{Anonymisierung}

\section{Evaluierung der Ergebnisse?}
%subs Usability-Tests?

\chapter{Fazit}
\label{fazit}
\minitoc\pagebreak

%\section{Vorstellung der Ergebnisse}

\section{Rückschlüsse für weitere Entwicklung}

%als sub hier rein?
\subsection{Sonstige Aspekte?}
\subsubsection{Auslieferungsprozess?}
\subsubsection{Internationalisierung}
\subsubsection{Incremental Load?}
\label{sub:incremental}

% Aspekte vor Erstellung??
\subsection{Rechtliche Aspekte?}
\label{sub:recht}
\subsubsection{Datenschutz}
\subsubsection{Anonymisierung}

\section{Aufgetretene Probleme?}

%\section{Erfüllte Anforderungen}
%\subsection{Nutzeranforderungen}
%\subsection{Theoriebasierte Anforderungen}

\section{Ausblick}
%\subsection{Mehrwert}

%%%%%%%%%% ENDE KAPITEL
\pagenumbering{Roman} % römische Ziffern für die Anhänge

\begin{appendix}
% Glossar
%\setglossarysection{section}
%\glsaddall
\setglossarysection{chapter}
\printglossary[numberedsection,style=altlist,title=Begriffsdefinitionen]
\label{chap:Definitionen}

\chapter{Anforderungsermittlung}

\chapter{Evaluation}
%\includepdf[pages=-, nup=1x2]{attachments/ALL_EVALUATION.pdf}

%\chapter{Dashboards}
%\includepdf[pages=-, nup=1x2, scale=0.6, landscape=true]{attachments/d12.pdf}

\includepdf[pages=-, nup=1x2, scale=0.7, pagecommand=\chapter{Dashboards}\label{att:dashboards}, offset=0 -2cm]{attachments/d12A1b.pdf}
\end{appendix}

% Verzeichnisse (Literatur, Abkürzungen, etc)
\singlespacing
% VERZEICHNISSE

%%%% BEGINN VERZEICHNISSE

% Abkürzungen 
\deftranslation[to=German]{Acronyms}{Abkürzungsverzeichnis}
\printglossary[type=\acronymtype,style=long]

% Abbildungsverzeichnis
\listoffigures

% Tabellenverzeichnis
\listoftables

% Bibliographie
\nocite{*} % Diese Zeile löschen, wenn nur verwendete Literaturangaben auftauchen sollen
\bibliographystyle{common/abbrvdin}
\bibliography{bibliographie}

%%%% ENDE VERZEICHNISSE
\onehalfspacing

% Danksagungen
\newpage
\chapter*{Danksagung}
\thispagestyle{empty}

dank

% Eidesstattliche Erklärung
\newpage
\chapter*{Eidesstattliche Erklärung}
\thispagestyle{empty}
Ich erkläre hiermit an Eides statt, dass ich die vorliegende Arbeit selbstständig und ohne Benutzung anderer als der angegebenen Hilfsmittel angefertigt habe.
Die aus fremden Quellen direkt oder indirekt übernommenen Gedanken sind als solche gekennzeichnet.

Die Abbildungen in dieser Arbeit sind eigens erstellt worden oder mit einem entsprechenden Quellennachweis versehen.

Die Arbeit wurde noch nicht veröffentlicht und auch bisher keiner Prüfungsbehörde vorgelegt.
\\\\\\
\noindent Fulda, den 08.04.2019
\begin{flushright}
$\overline{~~~~~~~~~\mbox{(Verfasser)}~~~~~~~~~}$
\end{flushright}

\end{document}
