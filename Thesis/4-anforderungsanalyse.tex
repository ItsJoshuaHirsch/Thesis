\chapter{Anforderungsanalyse}
\label{kap:anforderungsanalyse}
\minitoc\pagebreak
%\section{Requirements Engineering}
% \lipsum[1-52]

\section{Nutzungskontext}
\label{sec:kontext}

%\todotext{STANDDERTECHNIK ANALYSE FÜR AUSWERTUNG?!}
\gls{REVIEW} wird hauptsächlich in der präklinischen Rettung, beispielsweise einer Rettungswache, ansonsten auch in der klinischen Umgebung wie der Notaufnahme verwendet.
Es dient dabei dem Nachvollziehen einer Mission im Detail.

\gls{ANALYSE} ist eine Erweiterung als Server, welcher per Upload oder Import alle Einsätze abspeichert. 
Mithilfe einer Suchmaske können in einer Tabelle entsprechende Einsätze nach bestimmten Kriterien gesucht und gefiltert werden.
Ein Ausschnitt dieser Tabelle mit Suchmaske ist der Abbildung \ref{fig:missionlist} zu entnehmen.
Anschließend kann ein Einsatz in \gls{REVIEW} geöffnet und im Detail betrachtet werden.

\bildbreit
{missionlist}
{Ausschnitt aus der \glqq Missionsliste\grqq{} in \gls{ANALYSE}}
{Missionsliste \gls{ANALYSE}}

Das Auswerten von vielen Einsätzen, gegebenenfalls auch mit bestimmten Kriterien wie \glqq nur Nachts\grqq, ist zurzeit bedingt möglich.
Hierfür können entweder die Einsätze in der Tabelle direkt miteinander verglichen werden oder eine Mehrfachauswahl an Einsätzen vorgenommen, siehe Abbildung \ref{fig:missionlist}, und anschließend ein Export im CSV-, BDF+-, oder \glqq corpuls ZIP\grqq -Format durchgeführt werden.
Daraufhin kann mit dem jeweiligen Export in einem Drittanbieterprogramm eine oberflächliche Auswertung der ausgewählten Einsätze vollzogen werden.

%\subsection{Organisationen}
%Betroffene Organisationen sind unter anderem \todotext{orgas?}
%\subsubsection{Plätze?}

\section{Nutzergruppen}
\label{sec:nutzergruppen}
\epigraph{"`Ein Stakeholder ist eine Person oder Organisation, die einen direkten oder indirekten Einfluss auf die Systemanforderung hat"'} {Klaus Pohl \& Chriss Rupp \cite[S.29]{Pohl.2011}}

Bei Stakeholdern kann grundsätzlich zwischen zwei Gruppen unterschieden werden \cite{Leffingwell.2011}:
\begin{description}
\item [Requirements-Provider] \hfill \\
Diese Personen oder Organisationen sind in der Regel jene, welche die Software oder das Produkt im Tagesgeschäft nutzen beziehungsweise voraussichtlich nutzen werden.
Sie haben neue Ideen oder Verbesserungsvorschläge aus vorherigen Anwendungen.
Dies können auch interne Mitarbeiter der Firma sein, welche besondere Erfahrungen in den zu entwickelnden Bereichen haben. %Ref to internal Stakeholders somehow?
\item [Constraint-Provider] \hfill \\
Hierbei handelt es sich um Personen, welche keine Anforderungen als solche definieren, sondern die Umsetzbarkeit der Anforderungen von den Requirements-Providern testen und daraus folgend technische Rahmenbedingungen, wie zum Beispiel die zu verwendende  Programmiersprache oder Datenbank, empfehlen oder festlegen.
\end{description}

Eine \glqq Stakeholder-Liste\grqq ist ein hilfreiches Mittel, um einen Überblick an den beteiligten Personen zu erhalten und damit einhergehend so viele Anforderungen wie möglich an das System zu erheben (vgl. \cite[S. 83]{Bergsmann.2018}).
Dieser Prozess ist vor allem zu Beginn der Anforderungsanalyse von besonderer Bedeutung, da so frühzeitig viele Informationen und Hinweise erkennbar werden, wie das System aufgebaut werden muss.
Ein nachträgliches anpassen oder hinzufügen von Anforderungen kann unter Umständen schwere grundlegende Probleme mit sich ziehen, da gewisse fundamentale Architekturentscheidungen auf dieser Basis aufbauen können.
Ein Beispiel einer solchen Liste für \gls{ANALYSE} ist in Tabelle \ref{tbl:Stakeholder-Liste} zu sehen.

Die Requirements-Provider, fortan Nutzergruppen genannt, werden für das Erheben der Anforderungen hinzugezogen.
Sie sind ein essentieller, beeinflussender Faktor, wenn es um die Entwicklung einer Anwendung oder die Erweiterung einer Anwendung um ein umfangreiches Feature geht \cite[S. 125]{Herczeg.2018}.
Indirekt beschränken, aber auch erweitern sie die Funktionalitäten und beeinflussen die Art und Weise, wie die Anwendung gestaltet wird. 
Beispielsweise welche Erfahrungswerte vorausgesetzt oder angelernt werden müssen, welche Icons oder Symbole bekannt oder unbekannt sind, welche Ziele, Erwartungen oder Befürchtungen sie haben spielen eine große Rolle bei der Entscheidung, welche Informationen in welchem Format dargestellt werden.

%\todotext{Constraint-Provider: der Produktmanager, die Software-Entwickler, welche die neue Funktionalität implementieren und Ich?} 
Die Aufgaben der Constraint-Provider für dieses \gls{Feature} sind unter anderem wie oben genannt das Testen der Umsetzbarkeit.
Hierbei ist ein wesentlicher Bestandteil die Prüfung, ob eine Fragestellung mit den derzeit vorhandenen Daten der Geräte beantwortet werden kann.
Weitere Aufgaben sind unter anderem Entscheidungen über das Format und die Bereitstellung der Daten oder die Haltung und Persistierung in den entsprechenden Datenbanken.

Im Rahmen dieser Arbeit gibt es nicht die klassischen Anforderungen, wie beispielsweise \glqq ein Export sollte im CSV- und Zip-Format möglich sein\grqq. Die Anforderungen sind hierbei Fragestellungen welche beantwortet werden sollen, wie zum Beispiel \glqq Wie viele Reanimationen hat meine Wache?\grqq.

Daher ist zunächst zu klären, welche Nutzergruppen bei dem bisherigen Produkt vertreten sind.
Anschließend sollten Sie in Hinblick auf das zu erarbeitende Thema analysiert werden und daraufhin ist zu definieren, welche Eigenschaften oder Charakteristiken diese besitzen und ob gegebenenfalls neue hinzukommen oder alte hinfällig sind. 
%\todotext{Referenzierung oder Verbindung zu meinen Nutzergruppen, Stakeholdern. \\Abgrenzung nur Anwender,keine Deployer etc.?, }

\subsection{Stakeholder von \acrlong*{ANALYSE}}
\label{subsec:stakeholder}
Eine tabellarische Auflistung der Stakeholder soll, wie in \ref{sec:nutzergruppen} beschrieben, einen Überblick der Personen(-gruppen) liefern, die einen Einfluss auf das Produkt haben.
Somit soll gewährleistet werden, dass während der Konzeption und später auch bei der Umsetzung keine wichtigen Stakeholder vergessen werden, da dies schlimmere Nachwirkungen hat, je später sie bemerkt werden.


% not sure if I#ll keep this table or just make a Auszug of it and more text?
\begin{table}[htb]
\centering
\setlength{\extrarowheight}{4pt}
%\begin{tabular}{ |p{0.05\linewidth} | p{0.2\linewidth} | p{0.4\linewidth} | p{0.25\linewidth} |}
\begin{tabular}{ |p{0.5cm} | p{4cm} | p{5.5cm} |p{4cm} |}
  \hline
	\multicolumn{4}{|c|}{\textbf{Stakeholder}} \\
  \hline
\textbf{ID} & \textbf{Stakeholder-Bezeichnung} 	& \textbf{Organisation (Beispiel)} & \textbf{Gruppe}
  \\\hline
  1			& Geschäftsführer 					& \gls{GS} 				& Requirements-Provider
  \\\hline  
  2			& Entwickler 						& \gls{GS} 				& Constraint-Provider
  \\\hline
  3			& medizinische Forscher 			& Forschungseinrichtungen oder \gls{GS} & Requirements-Provider
  \\\hline
  4			& Leiter Rettungsdienst				& \gls{DRK}				& Requirements-Provider
  \\\hline
  5			& Produktmanager 					& \gls{GS} 				& Constraint-\& Requirements-Provider
  \\\hline
  6			& Klinikpersonal					& Kliniken			& Requirements-Provider
  \\\hline
  7			& Qualitätsmanagement 				& \gls{GS} 				& Requirements-Provider
  \\\hline
  8			& Ausbilder 						& \gls{BRK}	 			& Requirements-Provider
  \\\hline
  9			& Systemadministrator 				& IT-Abteilung Krankenhaus				& Constraint-Provider
  \\\hline  
  10		& Tester 							& \gls{GS} 				& Constraint-Provider
  \\\hline
  11		& Rettungsdienstpersonal			& Malteser				& Requirements-Provider
  \\\hline
\end{tabular} 
  \caption[Stakeholder-Liste \acrlong*{ANALYSE}]{Stakeholder-Liste von \gls{ANALYSE} nach Bergsmann \cite[S. 85]{Bergsmann.2018}}
  \label{tbl:Stakeholder-Liste}
\end{table}

Eine solche exemplarische Stakeholder-Liste ist in Tabelle \ref{tbl:Stakeholder-Liste} zu sehen.
Dabei wurde die Anzahl der Spalten verändert, da beispielsweise eine explizite Nennung der Kontaktperson für diese Arbeit nicht von Nöten ist.
In der Liste wird die entsprechende Bezeichnung, eine beispielhafte Organisation und die Einordnung, ob es sich um ein Requirements- oder Constraint-Provider handelt, festgehalten.
Hiermit ist es möglich, in den entsprechenden Phasen der Anforderungsanalyse, Konzeption oder Umsetzung die jeweils notwendigen Stakeholder zu berücksichtigen und gegebenenfalls zu kontaktieren. 
Diese Liste dient als Orientierung, sie ist nicht vollständig und kann beliebig erweitert, ergänzt oder eingeschränkt werden.
Auch eine höhere Abstraktion oder detailliertere Auflistung der Personen und Gruppen kann je nach Bedarf vorgenommen werden. 

Mithilfe dieser Tabelle und der Spalte \glqq Gruppe\grqq{} können beispielsweise die Nutzergruppen aus den gelisteten Stakeholdern extrahiert werden.
Diese werden im folgenden Abschnitt weiter erläutert.

\subsection{Nutzergruppen von \acrlong*{ANALYSE}}
\label{sub:NutzergruppenAnalyse}
In Tabelle \ref{tbl:Stakeholder-Liste} ist eine Auswahl an Stakeholdern von \gls{ANALYSE} zu sehen.
Diese kann hinzugezogen werden, wenn man zu einem Zeitpunkt der Planung oder Kozeptionierung eine bestimmte Gruppe dieser Personen miteinbeziehen möchte.
Für die Anforderungsanalyse der zu erstellenden \gls{Dashboard}s ist es sinnvoll zu wissen, was die Nutzer haben möchten.
Oder in diesem Fall welche Fragen sie haben, die womöglich mit den Daten der Geräte beantwortet werden können.

Um viele dieser Fragen zu erhalten, werden die Nutzergruppen von \gls{ANALYSE} benötigt.
Also jene Personen, die das Produkt oder die Daten, mit welchen das Produkt arbeitet, möglichst im Tagesgeschäft verwenden.
Aus der Liste \ref{tbl:Stakeholder-Liste} lassen sich die Personen oder Gruppen rauslesen, welche entsprechende Fragestellungen haben könnten. 
Wenn man nur die Requirements-Provider betrachtet, bleiben folgende relevante Nutzergruppen übrig:
\begin{multicols}{2}
\begin{itemize}
\item Geschäftsführer
\item medizinische Forscher
\item Leiter Rettungsdienst
\item Produktmanager
\item Mitarbeiter im Krankenhaus
\item Mitarbeiter des Qualitätsmanagement
\item Ausbilder von Rettungsdienstpersonal
\item Rettungsdienstpersonal
% \item Produktmanager
\end{itemize}
\end{multicols}

Aus dieser Liste von Personen und Gruppen kann ein weiterer Schritt der Reduzierung durchgeführt werden. 
So betrachtet man jene fünf Nutzergruppen, welche \gls{ANALYSE} sehr häufig verwenden oder voraussichtlich verwenden werden:
\begin{enumerate}
\item \textbf{Leiter Rettungsdienst} Ein Leiter Rettungsdienst oder auch ein Rettungswachenleiter hat die Verantwortung über einen Rettungsdienst und damit gegebenenfalls über mehrere Wachen.
Somit stehen unter ihm mehrere Personen in Schichtgruppen, die er koordiniere und einteilen muss.
Organisatorische Aufgaben wie Prozess- und Mitarbeitermanagement gehören zu seinem Tagesgeschäft.
Hierbei kann eine Datenspeicherung und daraus resultierende Auswertung und Analyse von Prozessen hilfreich und von besonderer Wichtigkeit sein.
\item \textbf{Forscher} Sie haben das Ziel, mit wissenschaftlichen Methoden neue Erkenntnisse zu gewinnen, Hypothesen aufzustellen und diese zu beweisen oder zu widerlegen.
Sie haben besonders viele Fragestellungen, die mit einfachen Daten wie Einsatzzeitpunkte oder komplexeren Daten, wie  die Entwicklung von medizinischen Vitalparametern, beantwortet werden können.
\item \textbf{Ausbilder von Rettungsdienstpersonal} Diese Gruppe bildet neues Personal nach entsprechenden Vorgaben im Rettungswesen aus.
Das Ziel ist eine umfangreiche und vor allem qualitativ hochwertige Ausbildung, da es um Patientenleben gehen kann.
Ein gewisses Controlling und daraus resultierendes Feedback an Einzelpersonen oder Gruppen ist von besonderer Bedeutung.
Mittels Datenauswertung kann dieses Feedback zielgenauer und spezifischer sein und so die Qualität der Ausbildung merklich verbessern.
\item \textbf{Qualitätsmanagementbeauftragte} Mitarbeiter in der Qualitätssicherung oder im Qualitätsmanagement sin verantwortlich dafür, dass gewisse Vorgaben kommuniziert und eingehalten werden.
Da sie meist bei größeren Trägern zum Einsatz kommen, ist die gebündelte Analyse von großen vorliegenden Datenmengen notwendig.
\item \textbf{ärztliches Personal} Diese Gruppe umfasst unter anderem Chef-, Not-, Tele-, Fachärzte, sowie auch weiteres klinisches Personal.
Sie können mittels einer Auswertung ihrer Daten entsprechende Therapiemaßnahmen evaluieren und etwaige Auswirkungen wahrnehmen.
\end{enumerate}


\subsubsection{Personas}
Eine Persona ist eine fiktive, spezifische Beschreibung einer Nutzergruppe.
Es ist ein \gls{Archetyp}, welcher eine Klasse von Nutzern zusammenfasst und so die Entwickler beim Entwerfen von Mensch-Computer-Anwendungen unterstützen soll.
Dabei werden die Charakteristiken von ein oder mehreren Nutzern kurz und prägnant zusammengefasst.
Beispielhafte Details können unter anderem die Berufsbeschreibung, Ziele, Erwartungen, Anforderungen sowie Alter und Hobbies sein.
Sie sollen den Entwicklern die zukünftigen Anwender vor Augen halten, damit eine angemessene Bedientauglichkeit der Software für alle Nutzergruppen gewährleistet wird (vgl. \cite[3.2, S.11]{Karwowski.2011,Pruitt.2006}).

Aus den in \ref{sub:NutzergruppenAnalyse} analysierten Nutzergruppen von \gls{ANALYSE} können mindestens fünf Personas erstellt werden.
Im Zuge der Entwicklung der \textsf{corpuls\color{corpulsred}{.web}}-Produkte wurden bereits Personas angelegt.
Dabei sind aus \ref{sub:NutzergruppenAnalyse} die Nutzergruppen 2 - 5 bereits abgedeckt:
\begin{description}
\item[Nutzergruppe 2: Jörn]
	\glqq Der Reanimations-Wissenschaftler\grqq 
\item[Nutzergruppe 3: Bernd]
	\glqq Der Sanitäter-Ausbilder\grqq 
\item[Nutzergruppe 4: Juliane]
	\glqq Miss Quality\grqq
\item[Nutzergruppe 5: Hermann]
	\glqq Der Land(Not)arzt\grqq
\end{description}

Folglich fehlt Nutzergruppe 1: Leiter Rettungsdienst.
Angelehnt an die zuvor erstellten Personas wird exemplarische eine neue Persona für die Nutzergruppe 1 erstellt:

\begin{table} [htb]
    \begin{tabular}{| p{4cm} | p{10cm} |}
    \hline
    \textbf{Persona}                      & \makecell[cl]{\tabitem Name: Andreas\\\tabitem Alter: 47\\\tabitem Hobbies: Schach, Golf\\\tabitem Beruf: Leiter Rettungsdienst}                                           \\ \hline
    \textbf{Berufsbeschreibung}           & \makecell[cl]{\tabitem Einteilen der Personen in Schichtgruppen und \\Schichtpläne erstellen\\\tabitem Interne Prozesse überwachen und verbessern}            \\ \hline
    \textbf{Motivation \& Ziele}        & \makecell[cl]{\tabitem Ich möchte meinen Rettungsdienst stetig verbessern\\ \tabitem Zufriedenes Personal und gesunde Patienten sind meine\\ höchste Priorität} \\ \hline
    \makecell[cl]{\textbf{Anforderungen }\\ \textbf{\& Erwartungen}} & \makecell[cl]{\tabitem Ich möchte schnell sehen, welche Einsätze in meinen\\ Wachen passieren\\\tabitem Zeit um große Tabellen durchzuschauen habe ich nicht} \\ \hline
    \textbf{Schmerzpunkte}                	& \makecell[cl]{\tabitem Veraltete Software die mir die Zeit raubt\\\tabitem Unübersichtliche Darstellung von Daten}                                          \\ \hline
    \end{tabular}
    \caption[Persona: Leiter Rettungsdienst]{Exemplarische Persona für Nutzergruppe 1: Leiter Rettungsdienst}
  \label{tbl:Persona}
\end{table}

Normalerweise wird einer Persona ein Avatar hinzugefügt, um dem Entwickler das Gefühl zu geben, für einen realen Menschen zu entwickeln. 
Bei der Persona in Tabelle \ref{tbl:Persona} wurde auf das übliche Avatar verzichtet. 


Es wurden die gängigen Beschreibungen wie bei den bereits vorhandenen Personas aufgeführt und entsprechend mit den Charakteristiken der Nutzergruppe \glqq Leiter Rettungsdienst\grqq{} versehen.
%\section{Aufgabenanalyse} ?

\section{Iterative Erhebung der zu beantwortenden Fragestellungen}
\label{sec:erhebung}
Im folgenden Abschnitt wird der Prozess der Erhebung von den Fragestellungen erläutert.
Hierbei wird die Methodik gemäß Requirements Engineering \cite{Pohl.2011} im iterativen Prozess angewandt.
Dieser Prozess geschieht zwar iterativ, zugunsten der Lesbarkeit dieser Arbeit werden die jeweiligen Schritte jedoch als ein Kapitel zusammengefasst.

\bild
{requireEngineering}
{12cm}
{Zyklus des Requirements Engineering. Bildquelle: \cite{Patig.}}
{Zyklus Requirements Engineering}

Demnach werden, wie in Abbildung \ref{fig:requireEngineering} zu sehen, im ersten Schritt die Fragestellungen ermittelt.
Wichtig hierbei ist die Berücksichtigung der verschiedenen Stakeholder des Produktes, welche in Tabelle \ref{tbl:Stakeholder-Liste} (S.\pageref{tbl:Stakeholder-Liste}) zusammengetragen wurden.

Anschließend werden die ermittelten Fragestellungen analysiert beziehungsweise geprüft.
Dabei sollen die vielen Daten aus dem vorherigen Schritt klassifiziert und strukturiert werden.
So lassen sich Redundanzen oder Diskordanzen vermeiden und aus der Häufigkeit der Vorkommnisse Prioritäten ableiten \cite[S.100f]{Sommerville.2012}.

Beim spezifizieren werden die geordneten Anforderungen üblicherweise in eine Standardform gebracht.
Für diese Arbeit ist jedoch die Spezifikation der benötigten Daten ebenso eine zielführende Handlung.
Dadurch wird die weitere Validierung, anschließende Umsetzung und Erarbeitung der Handlungshinweise vereinfacht und gewährleistet.

Im vierten Schritt aus Abbildung \ref{fig:requireEngineering} werden die spezifizierten Fragestellungen validiert.
Dies geschieht im fließenden Übergang zu Kapitel \ref{sec:evaluierung}, da hier geprüft wird, ob die spezifizierten Fragestellungen auch tatsächlich mit denen der Stakeholder übereinstimmen.
Damit soll garantiert werden, dass das entsprechende Endprodukt den Anforderungen der Stakeholder entspricht \cite{Patig.}.


\subsection{Ermittlung von Fragestellungen}
Um zu wissen, welche Fragestellungen beantwortet werden sollen, müssen diese an erster Stelle ermittelt werden.
Wie bei Anforderungen an eine Software werden hierfür die entsprechenden Stakeholder herangezogen.
Dabei werden primär adäquat qualifizierte Mitarbeiter der Firma \gls{GS} befragt, welche einer Stakeholder-Gruppe aus Tabelle \ref{tbl:Stakeholder-Liste} zugeordnet werden können.
Hierfür wurden mehrere geeignete Personen der Abteilungen \glqq Medizinische Forschung und Anwendung\grqq, Anwendungsspezialisten, Produktmanagement, Vertrieb und Kundendienst, sowie ein aktueller Kunde aus Dresden miteinbezogen. \todotext{China?}

Zur Ermittlung werden zu Beginn Einzelinterviews durchgeführt. 
Diese haben der Vorteil, dass sich die entsprechende Person vollends auf die Thematik fokussieren kann.
Außerdem ist dadurch die Beeinflussung durch Dritte ausgeschlossen, sodass die alleinige Perspektive der entsprechenden Nutzergruppe gewährleistet ist.
Die Einzelinterviews haben eine durchschnittliche Länge von 60-90 Minuten und werden in einem Raum mit Computer und Beamer durchgeführt.
Mittels der genannten Technik ist es möglich, dem Interviewpartner zu Beginn Eindrücke zu gewähren, inwiefern Fragestellungen beantwortet werden können.
Dabei wird der Person eine Beispiel-App von Qlik präsentiert und die zugrundeliegende Fragestellung zu einer Visualisierung dargelegt, damit jedes Interview mit einer gleichen Wissensbasis vollzogen werden kann.
So ist es möglich dem Befragten eine Vorstellung zu geben, um welche Art von Fragestellung es sich handelt und welche realisierbar sind.

Anschließend werden eingangs offene Fragen gestellt und versucht, dem Interviewpartner wenig Input zu liefern, sondern so viel persönliche und nicht beeinflusste Informationen und Meinungen wie möglich zu erhalten.
Im Verlaufe des Gesprächs werden, falls notwendig, verschiedene Bereiche, welche die jeweilige Person noch nicht berücksichtigt hatte, nur nennen.
Auch Ideen von anderen Interviews werden bei einem fortschreitend kontextlosem Gespräch eingebracht, um etwaige übersehene Bereiche abzudecken oder neue Ideen hierfür zu erlangen. 

Im Anschluss werden die Notizen beider Beteiligten in digitaler Form möglichst unkomprimiert dokumentiert.
Ein exemplarischer Auszug ist hier zu sehen: \todotext{Anhang alles?}
\bildbreit
{notes}
{Kurzer Auszug der digitalisierten, möglichst unkomprimierten Notizen aus den Gesprächen mit Personen der \glqq Medizinischen Forschung und Anwendung\grqq}
{Auszug Notizen der Ermittlung von Fragestellung}



\subsection{Analyse der Fragestellungen}
\label{sub:analyseFragen}
Abbildung \ref{fig:requireEngineering} ist zu entnehmen, dass nach der Ermittlung die erste Phase der Analyse oder Prüfung der gesammelten Anforderungen ansteht.
Hier ist bereits ein Pfeil zurück auf den ersten Schritt zu erkennen.
Dies bedeutet, dass nach Abschluss der Analyse ein erneutes ermitteln Anforderungen mit den gleichen oder anderen Personen erfolgen kann.
Dabei können die gewonnenen Erfahrungen aus dem ersten Durchlauf mit eingebracht werden und so schneller bessere Ergebnisse erzielt werden.
\bildrechts
{listeanalyse}
{7cm}
{Analysierte Informationen strukturiert in Kategorien}
{Auszug analysierte Liste der Fragestellungen}
Bei der Analyse werden die unstrukturierten gesammelten Daten und Informationen grob vorstrukturiert.
Hierbei ist beispielsweise eine Einordnung in Kategorien eine hilfreiche Handlung, welche die Lesbarkeit enorm verbessert und die Komplexität auf mehrere Ebenen verteilt.
Durch diesen Schritt entstehen Kategorien wie Reanimation, Einsätze, Planung, Bedienung und viele mehr.

Ein weiterer Prozess ist die Vermeidung oder das Auflösen von Redundanzen und Diskordanzen.
Dabei werden ähnliche Informationen zusammengefasst und innerhalb der Kategorie nach oben verschoben.
Dadurch entsteht eine erste Priorisierung basierend auf der Häufigkeit, wie viele Personen die gleiche Information oder Fragestellung hatten.
Auch Diskordanzen werden eliminiert, indem beispielsweise die Auswertung nach einzelnen Mitarbeitern herausgenommen wird, da in einem Gespräch herauskam, dass dies als Leistungs- oder Mitarbeiterüberwachung beziehungsweise aus Datenschutzgründen rechtlich nicht trivial ist.

So entsteht eine Liste mit einer ersten Priorisierung in den Kategorien und ohne offensichtliche Duplikate oder Konflikte.
Ein Ausschnitt dieser Liste ist in Abbildung \ref{fig:listeanalyse} zu sehen.
\clearpage
\subsection{Spezifikation der Fragen und deren benötigte Daten}
\label{sub:spezifikation}
Mithilfe der in \ref{sub:analyseFragen} analysierten und strukturierten Liste ist es nun möglich, den nächsten Schritt gemäß Requirements Engineering durchzuführen: Die Spezifikation der Fragestellungen \cite{Pohl.2011}.

Für diese Arbeit bedeutet das in dem Fall zwei große Schritte: 
\begin{enumerate}
\item Zum einen müssen die Daten nach \cite{Patig.} in eine Standardform überführt werden. 
Da sie bisher aus den Interview-Notizen digital festgehalten wurden, sind es heterogene Formen wie Stichpunkte, Schlagwörter oder Fragen.
Eine zielorientierte Standardform wäre hierbei die Formulierung in Fragestellungen aus Sicht des Benutzers.
Dieser Vorgang wird in \ref{subsub:standardform} näher beschrieben.
\item Zum anderen sollten die standardisierten Fragestellungen auf die benötigten Daten überprüft werden.
Dabei können unterschiedliche Kategorien entstehen, wenn die Daten beispielsweise bereits im richtigen Format vorliegen oder aber die benötigten Daten zum aktuellen Zeitpunkt schlicht nicht vorhanden oder exportierbar sind.
So können etwaige Fragestellungen bereits ausgeschlossen oder höher priorisiert werden, wenn deren Daten bereits optimal vorliegen.
Auch ist es für später folgende Prozesse leichter, wenn die benötigten Daten spezifiziert sind.
Diese Maßnahme wird in \ref{subsub:datenspez} weiter erläutert.
\end{enumerate}

\subsubsection{Überführen der Informationen in eine Standardform}
\label{subsub:standardform}
Das Übertragen von Anforderungen oder Informationen in eine Standardform ist eine essentielle und wichtige Handlung, welche die Vergleichbarkeit und Handhabung der Anforderungen beziehungsweise Fragestellungen garantiert.

Als Standardform für diese Arbeit wurde \glqq Fragestellung aus Sicht des Benutzers\grqq{} gewählt, da jede Nutzergruppe (siehe S.\pageref{sub:NutzergruppenAnalyse} \ref{sub:NutzergruppenAnalyse}) Fragen hat.
Hierdurch ist eine klare Form erkennbar und weitere bisher unentdeckte Duplikate können weiter dezimiert und in Prioritäten überführt werden.
Auch für die spätere Entwicklung der Dashboards ist diese Form hilfreich, da gut überprüft werden kann, ob ein Diagramm eine oder mehrere Fragestellung beantwortet.
Ein exemplarischer Auszug dieser Überführung ist in Abbildung \ref{fig:standardform} zu erkennen.
\bildbreit
{standardform}
{Übertragung von Informationen in die hier zielführende Standardform einer Fragestellung aus Sicht des Benutzers}
{Überführung von Informationen in Standardform}

\subsubsection{Spezifizieren der benötigten Daten}
\label{subsub:datenspez}
Ein für dieser Arbeit zusätzlicher Schritt bei der Spezifikation ist das Darlegen der benötigten Daten, um die entsprechenden Fragestellungen beantworten zu können.
Hierfür kann in drei Schritten vorgegangen werden:

\begin{enumerate}
\item \label{enum:grob} Grobe Einteilung der Fragestellungen mittels Farbgebung in vier Kategorien: % \todotext{4 oder 3 Kategorien sehr interessant?}
	\begin{enumerate}
	\item Rot - Nicht möglich: Diese Fragestellungen werden ausgeschlossen, da sie aufgrund von technischen oder rechtlichen Aspekten nicht beantwortbar sind.
	\item Gelb - Fraglich: Die Datengrundlage oder andere Aspekte sind zurzeit ungewiss und müssen geklärt werden
	\item Grün - Daten vorhanden: Hierfür sind die Daten garantiert bereits vorhanden und diese Fragen können beantwortet werden
	\item Blau - Sehr interessant (optional): Daten sind ebenfalls vorhanden und zusätzlich werden die Fragestellungen mit hoher Priorität hier eingeordnet
	\end{enumerate}
\item Visuelles kategorisieren der eingeteilten Fragestellungen in die oben genannten Kategorien. 
Zusätzlich kann eine erste Einschätzung der Komplexität beziehungsweise Ein- oder Mehrdimensionalität vorgenommen werden
\item Recherchieren der Datenquellen
\end{enumerate}

Analog zu Schritt \ref{enum:grob} wird nun zuerst eine grobe Kategorisierung der standardisierten Fragestellungen vorgenommen. 
Dabei werden unter anderem Entwickler oder Produktmanager befragt oder eigenes Wissen angewandt.
Vorteil hiervon ist die Aussortierung von nicht umsetzbaren Fragestellungen, was den späteren Aufwand reduziert.
Der Abbildung \ref{fig:einteilung2} kann ein Auszug dieser Einteilung entnommen werden.
Zusätzlich zu der Farbgebung werden rechts zugehörig zur Kategorie entsprechende Bemerkungen festgehalten, sofern ein oder mehrere Fragestellungen fraglich oder nicht möglich sind.
\bildbreit
{einteilung2}
{Auszug der farblichen Kategorisierung nach Umsetzbarkeit der Fragestellungen}
{Kategorisierung nach Umsetzbarkeit der Fragestellungen}

Der nächste Schritt gruppiert die zuvor kategorisierten Fragestellungen.
Dies geschieht in einer Tabelle und wird visuell durch die entsprechende Kategorie-Farbe unterstützt.
Des Weiteren wird eine erste Einschätzung der Dimensionalität vorgenommen.
Dies soll für die weitere Entwicklung eine Hilfestellung darbieten, damit schnell ersichtlich ist, wo Klärungsbedarf ist oder welche Fragestellungen bereits beantwortet werden können.
In Abbildung \ref{fig:viseinteilung} kann die große Tabelle grob erkannt werden.
Die Reihen sind die Einschätzungen der Umsetzbarkeit und die Spalten die Kategorien aus \ref{sub:analyseFragen}.
\bildbreit
{viseinteilung}
{Überblick über die Tabelle zur Gruppierung der Fragestellungen nach Umsetzbarkeit und Kategorie}
{Überblick über die Tabelle zur Gruppierung der Fragestellungen}

Im dritten, ergänzenden Akt werden bereits die benötigten Datenquellen recherchiert und spezifiziert.
Hierfür werden bereits vorhandene Daten und deren Quellen den entsprechenden Fragestellungen zugeordnet.
Bei den unklaren Fragen werden Ideen, Vorschläge und weitere Kommentare hinzugefügt.
Damit ist die Grundlage zur Erstellung eines Prototypen und später auch bei der Umsetzung gelegt.
Daten von Fragen, die beantwortet werden können, liegen dokumentiert vor, sowie die fraglichen Anforderungen sind mit ihren bisherigen Ideen und Informationen gelistet. \todotext{bild?}


%\bildrechts
%{komplexitaet}
%{6cm}
%{Auszug zur Einschätzung der Komplexität}
%{Überblick über die Tabelle zur Gruppierung der Fragestellungen}
%In Abbildung \ref{fig:komplexitaet} sds

\subsection{Validierung der spezifizierten Fragestellungen}
Die Validierung der ermittelten, analysierten und schließlich spezifizierten Fragestellungen ist der letzte Schritt im Requirements Engineering, was nicht bedeutet, dass es damit endet.
Dieser geschieht, wie auf S.\pageref{sec:erhebung} beschrieben, im fließenden Übergang mit der Evaluierung des Prototypen in \ref{sec:evaluierung}.
Hier wird geprüft, ob die in \ref{sub:spezifikation} spezifizierten Fragestellungen den Erwartungen der Nutzergruppen entsprechen.

Alle Schritte zuvor wurden wie in Abbildung \ref{fig:requireEngineering} zu sehen, iterativ und teilweise parallel durchgeführt.
Es können jederzeit neue Befragungen und somit Ermittlungen von Fragestellungen durchgeführt werden und mit den neu erhobenen Daten die zuvor beschriebenen Schritte erneut iterativ vollzogen werden.
Im Rahmen dieser Arbeit wurde ab einem bestimmten Zeitfenster die Anforderungsanalyse abgeschlossen, damit ein Prototyp (\ref{sec:erstellungPrototyp}) und anschließend eine präsentierfähige Version (\ref{kap:umsetzung}) auf Basis der bis dato spezifizierten Fragestellungen erstellt werden kann.