\title{ {\large  Exposé zur Bachelorthesis} \\
\vspace{1cm} 
{\huge \bfseries Big Data Analytics mit Medizintechnik im Rettungsdienst} \\
\vspace{1cm} {\small Betreut durch: } \\
{\large Prof. Jan-Torsten Milde} \\
{\large Christoph Graumann, M.Sc.} \\
\vspace{1cm} {\large Hochschule Fulda} \\
{\small Digitale Medien} \\
}

\author{ Joshua Hirsch }
\date{\today}

\documentclass[11pt]{article}
% \usepackage[latin1]{inputenc}
\usepackage{german}
\usepackage{breakcites}
\usepackage{microtype}
\usepackage{lipsum}
\begin{document}
\maketitle \clearpage

\tableofcontents \clearpage


\section{Einleitung}\label{intro}
\lipsum[1-14]
\subsection{Firmenvorstellung}
\subsubsection{Produkte}
%\subsection{corpuls.web ANALYSE}
\subsection{Problemstellung}
\subsubsection{Fragestellung}
\subsubsection{Zielsetzung}
\subsubsection{Abgrenzung}
\subsection{Vorgehensbeschreibung}


\section{Grundlagen}
\lipsum[1-21]
% Or here web.ANALYSE?
\subsection{Visualisierung von Daten}
\subsubsection{Gestaltungsgrundsätze}
\subsubsection{Statistische Auswertungen}
\subsubsection{Multidimensionale Daten darstellen}
\subsection{Daten und deren Verarbeitung in der Notfallmedizin} % Rettungsdienst  
\subsection{Begriffe zum Thema Data Analytics}
\subsubsection{Übersicht der Zusammenhänge}
\subsubsection{Big Data}
\subsubsection{Business Intelligence}
\subsubsection{Data Warehousing}
\subsubsection{Dashboards}
\subsubsection{Business Intelligence in der Notfallmedizin}
\subsection{Technologien}
\subsubsection{Qlik Sense}
\subsection{Requirements Engineering}
\subsubsection{Methodik}


\section{Stand der Technik} 
%\section{Analyse} Eventuell eigenes Kapitel
\lipsum[1-21]
\subsection{Ist-Zustand corpuls.web ANALYSE}
\subsubsection{Verwendung zur Auswertung}
\subsubsection{Vorliegende Daten der Geräte}
%\subsection{Anforderungsanalyse ?Eigenes Kapitel?}
%\subsubsection{Nutzergruppen}
%\subsubsection{Aufgabenanalyse}
%\subsubsection{Nutzungskontext}
%\subsubsection{Szenarien}
\subsection{Qlik Sense Server}
\subsubsection{Lizenzierungsmodell}
\subsubsection{?}


\section{Anforderungsanalyse}
%\section{Requirements Engineering}
\lipsum[1-26]
\subsection{Nutzergruppen}
\subsection{Aufgabenanalyse}
\subsection{Nutzungskontext}
\subsection{Iterative Erhebung der zu beantwortenden Fragestellungen}
\subsubsection{Ermittlung von Fragestellungen}
\subsubsection{Analyse der Anforderungen}
\subsubsection{Spezifikation der Fragen und benötigte Daten}
\subsubsection{Validierung der spezifizierten Fragestellungen}

\section{Konzeption?}
\lipsum[1-28]
\subsection{Vorgehen bei der Konzeption}
\subsection{Datenmodell}
\subsection{Erstellung eines Prototypen}
\subsection{Generierung von Dummy-Daten?}
\subsection{Evaluierung durch ?}
\subsection{Analyse der Ergebnisse}


\section{Umsetzung}
\lipsum[1-37]
\subsection{Erstellung der Qlik-Apps}
\subsubsection{ETL-Prozess}
\subsubsection{Dimensionen}
\subsubsection{Kennzahlen}
\subsubsection{Dashboards}
\subsection{Rechtliche Aspekte}
\subsubsection{Anonymisierung}
\subsection{Technische Aspekte}
\subsubsection{Schnittstelle ANALYSE und Qlik}
\subsubsection{JIRA-Stories}
\subsubsection{Lasttests?}
\subsubsection{Auslieferungsprozess?}


\section{Fazit}
\lipsum[1-21]
\subsection{Vorstellung der Ergebnisse}
\subsection{Rückschlüsse für Entwicklung}
\subsection{Aufgetretene Probleme}
\subsection{Erfüllte Anforderungen}
\subsubsection{Nutzeranforderungen}
\subsubsection{Theoriebasierte Anforderungen}
\subsection{Ausblick}
%\subsection{Mehrwert}





\bibliographystyle{apalike}
\bibliography{mybib}

\end{document}