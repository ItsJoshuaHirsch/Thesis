% PRÄAMBEL für Latexdokumente
% Diese Datei muss als erstes in einem neuen Latexdokument eingebunden werden (per \insert )
% und darf NICHT (!) in einen document-Block eingebettet werden!

%\documentclass[fontsize=11pt, paper=a4, headinclude, twoside=false, parskip=half+, pagesize=auto, numbers=noenddot, open=right, toc=listof, toc=bibliography]{scrreprt}

\documentclass[headsepline,footsepline,footinclude=false,parskip=half+, oneside,fontsize=11pt,paper=a4,listof=totoc,bibliography=totoc]{scrbook} % one-sided

% PDF-Kompression
\pdfminorversion=5
\pdfobjcompresslevel=1

% Allgemeines
%\usepackage[automark]{scrpage2} % Kopf- und Fußzeilen
%\usepackage{amsmath,marvosym} % Mathesachen
\usepackage[T1]{fontenc} % Ligaturen, richtige Umlaute im PDF
\usepackage[utf8]{inputenc}% UTF8-Kodierung für Umlaute usw
\usepackage{enumitem}

% Schriften
%\usepackage{fourier}
%\usepackage[light]{kpfonts}
%\usepackage{gfsdidot}
\usepackage{setspace} % Zeilenabstand

% Microtype
%\usepackage[activate={true,nocompatibility},final,tracking=true,kerning=true,spacing=true,factor=1100,stretch=10,shrink=10]{microtype}
\usepackage[tracking=true]{microtype}
\DeclareMicrotypeSet*[tracking]{my}% 
  { font = */*/*/sc/* }% 
\SetTracking{ encoding = *, shape = sc }{ 45 }% Hier wird festgelegt,
            % dass alle Passagen in Kapitälchen automatisch leicht
            % gesperrt werden.

% Schriften-Größen
\setkomafont{disposition}{\normalfont\bfseries} % use serif font for headings
%\linespread{1.05} % adjust line spread for mathpazo font


% Sprache: Deutsch
\usepackage[ngerman]{babel} % Silbentrennung
\usepackage[ngerman]{translator} % Deutsche Überschriften

% PDF
\usepackage[ngerman,pdfauthor={Joshua Hirsch},  pdfauthor={Joshua Hirsch}, pdftitle={Bachelorarbeit}, breaklinks=true,baseurl={http://www.corpuls.world}]{hyperref}
\usepackage{url}
\usepackage{pdflscape} % einzelne Seiten drehen können
\usepackage{pdfpages}

%  Bibliographie
\usepackage{bibgerm} % Umlaute in BibTeX
\usepackage{epigraph} % Hervorgehobene Zitate

% Glossar
\usepackage[acronym,nonumberlist,toc]{glossaries}
\renewcommand*{\glspostdescription}{} %Den Punkt am Ende jeder Beschreibung deaktivieren
\makeglossaries
\renewcommand{\glsnamefont}[1]{\textbf{#1}} % Abkürzungen Fett

% Tabellen
\usepackage{multirow} % Tabellen-Zellen über mehrere Zeilen
\usepackage{multicol} % mehre Spalten auf eine Seite
\usepackage{tabularx} % Für Tabellen mit vorgegeben Größen
\usepackage{longtable} % Tabellen über mehrere Seiten
\usepackage{array}
\usepackage{float}
\usepackage{rotating} % Tabellen im Querformat
\usepackage{makecell}
%\renewcommand{\cellalign/theadalign}{cl}

% Bilder
\usepackage{graphicx} % Bilder
\usepackage{color} % Farben
\usepackage{floatflt} % Textumfluss
\usepackage{caption} % Verbesserte Untertitel
\captionsetup{font={small},labelfont=bf}

\graphicspath{{images/}}
\DeclareGraphicsExtensions{.pdf,.png,.jpg} % bevorzuge pdf-Dateien
\usepackage{subfigure} % mehrere Abbildungen nebeneinander/übereinander
\newcommand{\subfigureautorefname}{\figurename} % um \autoref auch für subfigures benutzen
\usepackage[all]{hypcap} % Beim Klicken auf Links zum Bild und nicht zu Caption gehen
\usepackage[section]{placeins} % Bilder nur in zugehöriger Section unterbringen

% Bildunterschrift
\setcapindent{0em} % kein Einrücken der Caption von Figures und Tabellen
\setcapwidth{0.9\textwidth}
\setlength{\abovecaptionskip}{0.2cm} % Abstand der zwischen Bild- und Bildunterschrift

% Quellcode
\usepackage{listings} % für Formatierung in Quelltexten
\definecolor{grau}{gray}{0.25}
\lstset{
	extendedchars=true,
	basicstyle=\tiny\ttfamily,
	%basicstyle=\footnotesize\ttfamily,
	tabsize=2,
	keywordstyle=\textbf,
	commentstyle=\color{grau},
	stringstyle=\textit,
	numbers=left,
	numberstyle=\tiny,
	% für schönen Zeilenumbruch
	breakautoindent  = true,
	breakindent      = 2em,
	breaklines       = true,
	postbreak        = ,
	prebreak         = \raisebox{-.8ex}[0ex][0ex]{\Righttorque},
}
\usepackage{xcolor}
\usepackage{soul}
\newcommand{\ctext}[3][RGB]{%
  \begingroup
  \definecolor{hlcolor}{#1}{#2}\sethlcolor{hlcolor}%
  \hl{#3}%
  \endgroup
}



\newlength\myboxwidth
\setlength{\myboxwidth}{\dimexpr\textwidth-2\fboxsep}
\definecolor{light-gray}{gray}{0.95}
%\newcommand{\code}[1]{\colorbox{light-gray}{\texttt{#1}}}
\newcommand{\code}[1]{\ctext[RGB]{240,240,240}{\texttt{#1}}}


% linksbündige Fußnoten
%\deffootnote{1.5em}{1em}{\makebox[1.5em][l]{\thefootnotemark}}


% für autoref von Gleichungen in itemize-Umgebungen
%\makeatletter
%\newcommand{\saved@equation}{}
%\let\saved@equation\equation
%\def\equation{\@hyper@itemfalse\saved@equation}
%\makeatother 

% Einstellungen der Seitenränder
% Change it! Look at Master-Thesis CGU
\usepackage[left=2.5cm,right=3cm,top=2cm,bottom=2cm,includeheadfoot]{geometry}
%\typearea{14} % typearea am Schluss berechnen lassen, damit die Einstellungen oben 

% Stuff
\usepackage{lipsum}

% Mini tableofcontent at each chapter
\usepackage[nohints]{minitoc} % Table of content at each chapter
% Mini tables of content at a chapter
\dominitoc
% More toc depth in minitoc
\setcounter{minitocdepth}{5}
\setcounter{secnumdepth}{5}
%\setcounter{tocdepth}{5}
% Only subsections in main TOC (no subsub)
% \setcounter{tocdepth}{1}
% Deutsches Mini-Verzeichnis
\mtcselectlanguage{german}
\addto{\captionsngerman}{% Making babel aware of special titles
  \renewcommand{\mtctitle}{Inhalt}
}



% Eigene Befehle %%%%%%%%%%%%%%%%%%%%%%%%%%%%%%%%%%%%%%%%%%%%%%%%%5

% Bearbeitungshinweise im Text
\newcommand{\todo}[1]{
      {\colorbox{red}{ TODO: #1 }}
}
\newcommand{\todotext}[1]{
      {\color{red} TODO: #1} \normalfont
}
\newcommand{\info}[1]{
      {\colorbox{blue}{\color{white}(INFO: #1)}}
}

% Einfache Abkürzung
\newcommand{\abk}[2] {
	\newacronym{#1}{#1}{#2}
}

% bild mit defnierter Breite einfügen
\newcommand{\bild}[4]{
  \begin{figure}[!hbt]
    \centering
      \vspace{1ex}
      \includegraphics[width=#2]{img/#1}
      \caption[#4]{\label{fig:#1} #3}
    \vspace{1ex}
  \end{figure}
}
%bild volle Breite
\newcommand{\bildbreit}[3]{
  \begin{figure}[!hbt]
    \centering
      \vspace{1ex}
      \includegraphics[width=\linewidth]{img/#1}
      \caption[#3]{\label{fig:#1} #2}
    \vspace{1ex}
  \end{figure}
}
% bild mit eigener Breite
\newcommand{\bildfix}[3]{
  \begin{figure}[!hbt]
    \centering
      \vspace{1ex}
      \includegraphics{img/#1}
      \caption[#3]{\label{fig:#1} #2}
      \vspace{1ex}
  \end{figure}
}
% Bild todo
\newcommand{\bildtodo}[2]{
  \begin{figure}[!hbt]
    \begin{center}
      \vspace{2ex}
	      \includegraphics[width=6cm]{../common/todo}
      %\caption{\label{#1} \color{red}{ TODO: #2}}
      \caption{\label{fig:#1} \todotext{#2}}
      %{\caption{\label{#1} {\todo{#2}}}}
      \vspace{2ex}
    \end{center}
  \end{figure}
}
% Bild im Anhang
\newcommand{\bildanhang}[3]
{
  \begin{figure}[!hbt]
    \centering
    \vspace{1ex}
    \includegraphics[width=\linewidth]{img/anhang/#1}
    \caption[#3]{\label{fig:#1} #2}
    \vspace{1ex}
  \end{figure}
}
% Bild am rechten Rand
\newcommand{\bildrechts}[4]
{
	\begin{floatingfigure}[r]{#2}
		%\centering
		\includegraphics[width=#2]{img/#1}
		\captionsetup{width=#2}
		\caption[#4]{\label{fig:#1} #3}		
	\end{floatingfigure}
}


\definecolor{corpulsred}{HTML}{DD0B2F}

\newcommand{\tabitem}{~~\llap{\textbullet}~~}
\newcommand{\cweb}{\textsf{corpuls\color{corpulsred}{.web}}}

% Basic information for cover & title page
\newcommand*{\getUniversity}{Hochschule Fulda}
\newcommand*{\getFaculty}{Fachbereich Angewandte Informatik}
\newcommand*{\getTitle}{Big Data Analytics mit Medizintechnik im Rettungsdienst}
\newcommand*{\getAuthor}{Joshua Hirsch}
\newcommand*{\getDoctype}{Bachelor Thesis in \glqq Digitale Medien\grqq}
\newcommand*{\getSupervisor}{Prof.~Dr.~Jan-Torsten Milde}
\newcommand*{\getAdvisor}{Christoph Graumann,~M.Sc.}
\newcommand*{\getSubmissionDate}{08.04.2019}
\newcommand*{\getSubmissionLocation}{Fulda}