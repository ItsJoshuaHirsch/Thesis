\title{
{\large \bfseries Exposé zur Bachelorthesis} \\ \vspace{1cm}
{\huge Gestaltung von Dashboards für die Auswertung von Gerätedaten in der Notfallmedizin} \\ \vspace{1cm}
{\large Betreut durch: } \\
{\small Prof. Jan-Thorsten Milde} \\ 
{\small Christoph Graumann M.Sc.} \\ \vspace{1cm}
{\large Hochschule Fulda} \\
{\small Digitale Medien} \\ 
}

\author{ Joshua Hirsch }
\date{\today}

\documentclass[12pt]{article}
%\usepackage[latin1]{inputenc}
\begin{document}
\maketitle
\clearpage

\tableofcontents
\clearpage

\section{Problemstellung}\label{problem}
Ein Einsatz im Rettungsdienst ist jedes Mal ein ganz neuer Fall. Es kommen verschiedene Faktoren in unterschiedlicher Stärke hinzu und machen einen solchen Einsatz einzigartig. Dennoch lassen sich Korrelationen feststellen, welche in einigen Einsätzen zu gleichen oder ähnlichen Ereignissen führen. \\
Die Auswertung von Einsätzen im Rettungsdienst oder Patienten im Krankenhaus ist hierbei ein wichtiger Bestandteil zur Validierung verschiedener Qualitätsvorschriften, wie z.B. die Einhaltung von global-gesetzlichen  Richtlinien (z.B. ERC blabla Fußnote), lokale Vorgaben seitens der Qualitätsmanagement-Abteilung oder die Performanz von Mitarbeitern, bzw. Teams. \\
Die heutzutage eingesetzten Geräte in diesen Bereichen besitzen viel Technik und Möglichkeiten der Datensammlung und -haltung. Diese Daten gilt es persistent abzuspeichern, was teilweise bereits durchgeführt wird, damit sie für spätere Auswertungen sinnvoll sind und u.a. den genannten Stellen einen Mehrwert bieten.

% \subsection{Forschungsstand}\label{previous work}
% A much longer \LaTeXe{} example was written by Gil~\cite{Gil:02}.

\subsection{Erkenntnisinteresse}\label{Erkenntnisinteresse}
Mit der grafischen Auswertung von vielen Einsätzen über längere Zeiträume mit unterschiedlichen Geräten sollen bereits vermutete Fragestellungen bestätigt oder widerlegt werden können. 
So z.B. die Annahme, dass es nachts weniger Einsätze gibt als tagsüber. Oder weiterführend: Dass die Qualität einer Reanimation in der Nacht nicht gleichwertig zu jener tagsüber zur Mittagszeit ist. Auch sehr forschungsnahe Fragen, 
wie bspw. ob der Anstieg des Blutdrucks in Kombination mit der Senkung der Sauerstoffsättigung zu einer Apnoe führt, sind denkbar spannend.
Es gibt unzählige Fragen, welche sich die entsprechenden Leiter solcher Einrichtungen jährlich, monatlich oder gar täglich stellen, und zum jetzigen Zeitpunkt keine adäquate, schnelle und simple Möglichkeit hierfür zur Verfügung haben. \\
Es soll eine Überprüfung gewährleistet werden, ob entsprechende Richtlinien eingehalten werden, Stärken und Schwächen von Menschen und Geräten identifizieren können, Prognosen für die Zukunft 
möglich machen oder gar neue Forschungsfragen finden und im besten Fall gleich beantworten können.

\paragraph{Outline}
The remainder

\section{Fragestellung}\label{fragestellung}
Aus der oben genannten Problemstellung ergeben sich folgende Fragen:
\begin{description}
\item[Fragestellungen]~\par
\begin{itemize}
      \item Welche Fragen haben unsere Anwender, die wir mit unseren Daten beantworten können?    
      \item Wie müssen wir Dashboards gestalten, damit diese Fragen beantwortet werden?
      \begin{itemize}
        \item Wie müssen die Dashboards entworfen werden, damit sie medizinisch korrekt sind?  
      \end{itemize}
      \item Was müssen wir in unserem Datenmodell beachten, damit wir diese Dashboards erstellen können?
      \begin{itemize}
        \item Was muss bei der Schnittstelle beachtet werden?
      \end{itemize}
\end{itemize}
\end{description}

\subsection{Zielsetzung}\label{ziel}
Ziel ist es, so viele Fragestellungen wie möglich der entsprechenden Anwender, wie z.B. Rettungswachenleiter, Ärztlicher Leiter, Qualitätsmanagementbeauftragte  u.a. herauszufinden und zu konkretisieren.
(Anschließend sollen) die gefundenen/erhobenen Fragen analysiert werden bezüglich der benötigten Daten und deren Format. Daraufhin folgt die Konzipierung von Dashboards, welche so viele Fragestellungen wie möglich beantworten sollen. Erste Entwürfe und letztendlich präsentierfähige Dashboards sollen das visuelle Ergebnis dieser Arbeit werden. \\
Simultan werden geeignete Datenmodelle zur reibungslosen Darstellung sowie Anforderungen an die Schnittstelle erarbeitet.

\subsection{Theoriebezug}\label{previous work}
A much longer \LaTeXe{} example was written by Gil~\cite{Gil:02}.


\section{Abgrenzung}\label{previous work}
Im Rahmen der Bachelorarbeit ist es nicht notwendig, die entstandenen Dashboards in die derzeit laufende Software zu implementieren. Handlungsempfehlungen für die entsprechenden Entwickler sind hierbei ausreichend. \\


\section{Methodik}\label{previous work}
A much longer \LaTeXe{} example was written by Gil~\cite{Gil:02}.

\subsection{Technologien}\label{results}
In this section we describe the results.

\subsection{Material}\label{previous work}
A much longer \LaTeXe{} example was written by Gil~\cite{Gil:02}.

\section{Gliederung}\label{previous work}
A much longer \LaTeXe{} example was written by Gil~\cite{Gil:02}.

\subsection{Zeitplan}\label{conclusions}
We worked hard, and achieved very little.

\bibliographystyle{abbrv}
%\bibliography{main}

\end{document}