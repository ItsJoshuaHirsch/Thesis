\chapter{Einleitung}
\label{einleitung}
\minitoc\pagebreak


\section{Firmenvorstellung}
\label{sec:firma}
\bild
{../logos/gs}
{8cm}
{Logo der Marke \glqq corpuls\grqq{} von \acrlong*{GS}. Bildquelle: Intern}
{Logo der Marke \glqq corpuls\grqq{} von GS}

Das Unternehmen \gls{GS}, ist in Fachkreisen unter dem Markennamen \glqq corpuls\grqq{} wohl bekannt. 
Die Firma beschäftigt sich seit über 35 Jahren (1982) mit Medizintechnik\footnote{Die Informationen in diesem Kapitel \ref{sec:firma} und deren Unterkapitel stammen aus internen Quellen oder der offiziellen Website \cite{GSElektromedizinischeGerateG.StempleGmbH.}}.
Dabei beinhaltet das Produktportfolio vor allem Rettungsgeräte für den präklinischen Einsatz.
Darunter fallen Aufgaben wie Reanimation, Defibrillation sowie Monitoring von Patienten. 

Als Kellerfirma von Günter Stemple gegründet, hat sich das bayerische Unternehmen mittlerweile in den Mittelstand entwickelt.
Dabei hat es diverse Auszeichnungen vor allem für die innovativen Entwicklungen erhalten.

Von Beginn an überzeugten die ersten Geräte corpuls200 \& corpuls300 den Markt der präklinischen Notfallmedizin mit Beständigkeit, leichtem Gewicht und hoher Funktionalität. 

Ebenso Pioniertätigkeiten in der Telemedizin sind \gls{GS} zuzuschreiben, da 1984 bereits eine Übertragung eines 1-Kanal-\gls{EKG}s über den Globus nach Deutschland stattgefunden hat.
% \cite{GSElektromedizinischeGerateG.StempleGmbH.}.


\subsection{Produkte}
%Mit dem corpuls 08/16 (1992) und dem corpuls3 (2007) gelang es GS, sich entgültig im
%Weltmarkt zu etablieren.

\subsubsection{\acrlong*{C3}}
\bild
{c3}
{13cm}
{Der \acrlong{C3} dargestellt in seinen drei Modulen: Monitor, Patientenbox, Defibrillationseinheit. Bildquelle: Intern}
{Der \acrlong*{C3} dargestellt in seinen drei Modulen}

Der \gls{C3} war jenes Gerät, welche \gls{GS} ab dem Jahre 2007 endgültig ein festes Standbein in den weltweiten Markt ermöglicht haben.
Der primäre Zielmarkt für den \gls{C3} ist die professionelle, präklinische Notfallrettung.
Dabei war beim \gls{C3} eine Besonderheit, dass dieser in drei Modulen trennbar ist.
Somit können die Defibrillationseinheit, der Monitor und die Patientenbox voneinander getrennt werden (siehe Abbildung \ref{fig:c3}).
Dies war und ist bis heute eine Innovation, welche den Anwendern des \gls{C3}s den Alltag deutlich erleichtert, da diese Modularität in verschiedenen Szenarios einen deutlichen Mehrwert bietet.

Eine weitere, wichtige Besonderheit des Geräts ist die Funktionalität Stamm-, Vitaldaten, 12-Kanal-\gls{EKG}s sowie Nachrichten über die Schnittstelle für Telemetrie an \gls{LIVE} zu senden.
Dadurch ist beispielsweise eine zentrale Überwachung möglich oder die aufnehmende Klinik kann sich bereits im Vorfeld einen ersten Eindruck des Patienten verschaffen, bevor dieser in das Klinikum eingeliefert wird.
Diese Telemetrieverbindung kann entweder per LAN, W-LAN oder dem Mobilfunkstandard GSM erfolgen.



%In insgesamt drei Gerätegenerationen wurde der corpuls3 mit einem schmalen Defibrillationsmodul zum einen leichter, zum anderen über diverse Soft- und Hardwareupdates kontinuierlich verbessert.


\subsubsection{\acrlong*{C1}}
%Speziell für die Anwendergruppen in den Bereichen Feuerwehr, First-Responder, Krankentransport
%und Katastrophenschutz wurde 2011 der corpuls1 (Abbildung 2) vorgestellt.
%Vor Allem zeichnet der sich der corpuls1 durch seine Kompaktheit aus, während dennoch
%umfangreiche Funktionalitäten bereit gestellt werden.
\bild
{c1}
{9cm}
{Der \acrlong{C1} als kompaktes Gerät für First-Responder oder Feuerwehr. Bildquelle: Intern}
{Der \acrlong*{C1} als kompaktes Gerät für First-Responder oder Feuerwehr.}
%\FloatBarrier

Für die Anwendergruppen der vereinfachten oder schnellen Hilfeleistungen wurde der \gls{C1} entwickelt.
Dieser bietet beispielsweise der Feuerwehr oder anderen First-Respondern eine abgestimmte Ausgewogenheit aus umfangreichen Funktionen und einer effizienten Kompaktheit, welche in Abbildung \ref{fig:c1} zu sehen ist.



\subsubsection{\acrlong*{cCPR}}
%Einen neuen Schritt wagte GS 2015, als diese in den Bereich der Thoraxkompressionsger
%äte expandierte und den corpuls cpr (Abbildung 3) vorstellten.
%Besonderheit am corpuls cpr ist, dass dieser nur ein Arm auf einer Seite besitzt, was die Nutzerfreundlichkeit
%extrem erhöht.
%Durch die automatischen Thoraxkompressionsgeräte werden die
%Helfer entlastet und können die patientenkritische Zeit besser nutzen.
%Auch wird durch
%diese Geräte eine kontinuerliche Reanimationsqualität und dadurch eine bessere Patientenversorgung
%erreicht.
\bild
{ccpr}
{6cm}
{Der \acrlong{cCPR} als Thoraxkompressionsgerät der Firma mit der Besonderheit des einen Arms. Bildquelle: Intern}
{Der \acrlong*{cCPR} als Thoraxkompressionsgerät mit der Besonderheit des einen Arms}
\FloatBarrier

2015 wagte \gls{GS} einen neuen Schritt in den Bereich der Thoraxkompressionsgeräte. 
Jene Geräte sollen eine \glqq \gls{CPR}\grqq{} mechanisch durchführen und so eine gute Reanimationsqualität gewährleisten.
Auch werden die Helfer hierdurch entlastet, indem sie in dieser kritischen Zeit die Hände und den Kopf frei für andere wichtige Prozesse haben.
Die Besonderheit beim \gls{cCPR} ist unter anderem der Aufbau mit nur einem Arm (siehe Abbildung \ref{fig:ccpr}), welcher einen flexiblen und schnelleren Einsatz garantiert und dadurch die Nutzerakzeptanz deutlich steigert.


\subsubsection{\acrlong*{cAED}}
%Abgerundet wird die Versorgungskette in verschiedenen Kompetenzbereichen durch den
%2018 vorgestellten Automatisierter externer Defibrillator (AED) corpulsAED (Abbildung 2).
%Damit werden nun auch die Anwendergruppe der Laien abgedeckt, aber auch im professionellen
%Bereich ist der corpulsAED vielseitig einsetzbar.
%Durch klare, strukturierte
%Anweisungen führt der corpulsAED einen Laien kompetent durch ein Reanimationsgeschehen
\bild
{aed}
{6cm}
{Der \acrlong{cAED} als laientauglicher automatisierter externer Defibrillator. Bildquelle: Intern}
{Der \acrlong*{cAED} als laientauglicher automatisierter externer Defibrillator}

Vollumfänglich abgedeckt wird die Versorgungskette mit dem 2017 vorgestellten \gls{cAED} in Abbildung \ref{fig:aed}.
Damit ist auch ein Produkt für Laien von \gls{GS} auf dem Markt und kann aber auch im professionellen Bereich als \glqq Halbautomat\grqq{} eingesetzt werden.
Ein \glqq \gls{AED}\grqq{} soll einen Laien sicher und kompetent durch eine Reanimation führen und begleiten.


\subsubsection{\textsf{corpuls\color{corpulsred}{.web}}}
Das Portfolio an Geräten wird durch die zahlreichen \textsf{corpuls\color{corpulsred}{.web}}-Produkte abgerundet.
Hierbei steht die Telemetriesoftware \gls{LIVE} als Medizinprodukt an erster Stelle.
\bild
{live}
{7cm}
{Beispiel von \acrlong{LIVE} bei dem Empfang eines Einsatzes von einem \acrlong{C3}. Bildquelle: Intern}
{Beispiel von \acrlong*{LIVE} bei dem Empfang eines Einsatzes von einem \acrlong*{C3}}

Mithilfe von \gls{LIVE} hat das rettungsdienstliche Personal die Möglichkeit, vorab erhobene Patientendaten wie Vitalparameter, Stamm- oder Ruhe-EKG-Daten in Echtzeit an Fachpersonal, wie beispielsweise Ärzte, zu übertragen.
Somit kann sich die aufnehmende Klinik bereits im Vorfeld ein ersten Überblick des Krankheitsbildes des Patienten verschaffen und dadurch die Ankunft und vermutlich notwendige Vorkehrungen vorbereiten.


%Auch Funktionalitäten eines bestimmten Produkts kann durch \textsf{corpuls\color{corpulsred}{.web}} Software erweitert werden.
Kunden, welche viele \gls{cAED}s im Besitz haben, empfiehlt sich die Gerätemanagementsoftware \gls{MANAGER}.
Dabei können alle registrierten \gls{cAED}s über eine zentrale Instanz verwaltet werden, auch wenn diese eventuell viele Kilometer voneinander getrennt sind.
Die \gls{cAED}s führen regelmäßig Selbsttests durch und senden die Ergebnisse beispielsweise per W-LAN an den \gls{MANAGER}.
Dadurch ist eine stetige Überwachung der Einsatzbereitschaft gewährleistet.
Auch im Falle eines fehlgeschlagenen Tests kann durch das Aufspielen eines Softwareupdates aus der Ferne diese eingeschränkte Einsatzfähigkeit wiederhergestellt werden.

Des Weiteren wird jedem Kunden die Software \gls{REVIEW} unentgeltlich zum Download bereitgestellt.
Mit dieser Anwendung hat jeder Nutzer die Möglichkeit, Einsätze vom \gls{C3}, \gls{C1}, \gls{cAED} oder \gls{cCPR} nach dessen Abschluss nachzubereiten.
Zum Beispiel stehen dabei nach einer Reanimation Informationen zur Druckfrequenz, Drucktiefe, Reanimationspausen oder Defibrillationen zur Verfügung.
Für die Daten zur Druckfrequenz und -tiefe kann beim \gls{C3}, \gls{C1} und \gls{cAED} ein CPR-Feedbacksensor angeschlossen werden, welcher auch ad-hoc vor Ort Rückmeldung zur Reanimation gibt, wie die entsprechende Frequenz und Tiefe ist.

\gls{REVIEW} wird durch \gls{ANALYSE} um eine zentrale Datenverwaltungssoftware ergänzt.
Hier hat der Anwender die Möglichkeit, in einer zentralen Datenbank alle Einsatzdaten, auf Wunsch anonymisiert abzuspeichern und auf diese zu späteren Zeitpunkten mittels einfachen Filtern zuzugreifen.
Da \gls{ANALYSE} das Hauptprodukt dieser Arbeit ist, wird es in Kapitel \ref{sec:istAnalyse} näher erläutert.




%Diese Arbeit beschäftigt sich maßgeblich mit der Softwarelösung \gls{ANALYSE} und dem \gls{C3}, da das  in \gls{ANALYSE} integriert werden soll.
%Deshalb werden diese beiden Produkte nun nachfolgend detaillierter vorgestellt.

%\subsection{Relevante Produkte}
%\subsubsection{\acrlong*{C3}}
%\todo{}
%\todo{explain CPR-Feedbacksensor}
%Der primäre Zielmarkt für den corpuls3 ist die professionelle, präklinische Notfallrettung.
%Das auffälligste Merkmal ist die Modularität des corpuls3.
%Dieser kann in die Module Monitoreinheit, Patientenbox sowie Defibrillator/ Schrittmachereinheit
%aufgeteilt werden (Abbildung 4).
%
%Eine weitere, wichtige Eigenschaft des Geräts ist die Möglichkeit, Stammdaten, Vitaldaten,
%12-Kanal-EKGs sowie Nachrichten über die Telemetrieschnittstelle an corpuls.web
%LIVE zu senden.
%Dadurch kann der Patient zum Beispiel an zentraler Stelle überwacht
%werden oder die aufnehmende Klinik kann sich bereits vorab ein erstes Bild des Patienten
%schaffen, bevor dieser ins Klinikum eingeliefert wurde.
%Diese Telemetrieverbindung
%kann entweder per LAN, WLAN oder GSM erfolgen.
%\subsubsection{\acrlong*{ANALYSE}}
%\todo{}

\section{Problemstellung}
\label{problem}
Ein Einsatz im Rettungsdienst ist jedes Mal ein gänzlich neuer Fall. 
Es kommen verschiedene Faktoren in unterschiedlicher Gewichtung hinzu und machen einen solchen Einsatz einzigartig. 
Dennoch lassen sich Korrelationen feststellen, welche in einigen Einsätzen zu gleichen oder ähnlichen Ereignissen führen.

Die heutzutage eingesetzten Geräte in diesen Bereichen besitzen viel Technik und Möglichkeiten der Datensammlung und -haltung.
Diese Daten müssen, wegen rechtlicher Aspekte zumindest in Deutschland, permanent abgespeichert werden.
Jedoch liegen diese meistens im Anschluss der Persistierung ungenutzt auf einem Datenträger oder Server.

Dabei verbergen sich in diesen Daten neue Erkenntnisse und unzählige Antworten auf Fragen, welche sich die Leiter entsprechender Einrichtungen jährlich, monatlich oder gar täglich stellen, und zum jetzigen Zeitpunkt keine adäquate, schnelle und simple Möglichkeit zur Beantwortung jener zur Verfügung haben.

Mit dieser Arbeit sollen genau diese ungenutzten Daten so aufbereitet werden, damit sie für eine grafische Auswertung sinnvoll sind und Rettungswachen, Krankenhäusern oder anderen Einrichtungen einen deutlichen Mehrwert bieten.

\subsection{Erkenntnisinteresse}
\label{erkenntnis}
Mit der grafischen Auswertung von vielen Einsätzen über längere Zeiträume mit unterschiedlichen Geräten sollen bereits vermutete Fragestellungen bestätigt oder widerlegt werden können.
Fragen wie \glqq Wird tagsüber besser reanimiert als nachts? Wie steht dies im Verhältnis zur Einsatzdichte der jeweiligen Schichten?\grqq{} sind nur einige der vielen Fragen, die derzeit im Raum stehen und nur schwerlich mit einer Antwort ausgestattet werden können.
Auch sehr forschungsnahe Fragen, beispielsweise ob der Anstieg des Blutdrucks in Kombination mit der Senkung der Sauerstoffsättigung zu einer Apnoe\footnote{Aussetzen der Atmung, willentlich oder unbewusst \cite[S.168]{Adams.2016}; Atemstillstand \cite{Dudenredaktion.2015}} führt, sind denkbar spannend.

Die Daten sollen entsprechend vorliegen, damit die zu erwartenden Fragen in adäquater Form und Zeit beantwortet werden können.
Hierbei soll auch die Möglichkeit der Überprüfung gewährleistet werden, dass entsprechende gesetzliche Richtlinien \cite{Maconochie.2015} oder lokale Vorgaben eingehalten werden oder die Stärken und Schwächen von Menschen und Geräten identifiziert werden können.
Auch Prognosen für die Zukunft sollen möglich gemacht werden oder gar neue Forschungsfragen gefunden  und im besten Fall gleich beantwortet werden können.

Eine gute Möglichkeit um diese Daten übersichtlich zu visualisieren sind sogenannte \glqq \gls{Dashboard}s\grqq, ein englischer Begriff welcher \glqq Armaturenbrett\grqq{} bedeutet.
Dieser hat sich in der digitalen Welt als Schlagwort etabliert, eine Übersicht über viele verschiedene Informationen und Daten zu liefern, so wie es ein Armaturenbrett im Auto vollbringt (siehe \ref{sub:dashboards}).

\subsection{Fragestellungen}
\label{fragen}
Aus der oben genannten Problemstellung und Erkenntnisinteresse ergeben sich folgende Fragen:
\begin{description}
\item[Fragestellungen]~\par
\begin{itemize}
      \item Welche Fragen haben unsere Anwender, die wir mit unseren Daten beantworten können?    
      \item Wie müssen wir Dashboards gestalten, damit diese Fragen beantwortet werden?
      \begin{itemize}
        \item Wie müssen die Dashboards entworfen werden, damit sie medizinisch korrekt sind?  
      \end{itemize}
      \item Was müssen wir in unserem Datenmodell beachten, damit wir diese Dashboards erstellen 					können?
      \begin{itemize}
        \item Was muss bei der Schnittstelle beachtet werden?
      \end{itemize}
\end{itemize}
\end{description}

\subsection{Zielsetzung}
\label{ziel}
Ziel ist es, so viele Fragestellungen wie möglich der entsprechenden Anwender, wie zum Beispiel Rettungswachenleiter, Ärztlicher Leiter, Qualitätsmanagementbeauftragte u.a. herauszufinden und zu konkretisieren.
Anschließend sollen die erhobenen Fragen bezüglich der benötigten Daten und deren Format analysiert werden.
Daraufhin folgt die Konzipierung von Dashboards, welche so viele Fragestellungen wie möglich beantworten sollen.
Erste Entwürfe und letztendlich präsentierfähige Dashboards sollen das visuelle Ergebnis dieser
Arbeit werden.

Simultan werden geeignete Datenmodelle zur reibungslosen Darstellung sowie Anforderungen an die Schnittstelle erarbeitet.

\subsection{Abgrenzung}
\label{abgrenzung}
Im Rahmen der Bachelorarbeit ist es nicht notwendig, die entstandenen Dashboards in die derzeit laufende Software zu implementieren.
%Handlungsempfehlungen für die entsprechenden Entwickler sind hierbei ausreichend.
Des Weiteren ist keine produktive Anbindung an die entsprechende Schnittstelle vorgesehen.
Etwaige Rückschlüsse oder das Testen von Technologien sind hierbei zureichend.

\section{Vorgehensbeschreibung}
Die Erhebung der Fragestellungen wird vollumfänglich durch Interviews stattfinden.
Dies passiert in einem zyklisch-iterativen Verfahren, gemäß etablierter Prozesse des Requirements Engineering \cite{Pohl.2011} (siehe Kapitel \ref{sub:methodik}).
Dabei werden vorerst die internen Mitarbeiter gefragt, welche Fragestellungen sie für sinnvoll erachten.
Dabei wird darauf geachtet, dass die befragten Personen einen engen Bezug zum Rettungsdienst oder zu Kunden, beziehungsweise Mitarbeitern in dieser Branche haben.
Das Unternehmen liefert hierbei eine Menge in Frage kommender Stakeholder, da die Quote der ehemaligen und noch aktiven Rettungsdienstmitarbeiter überdurchschnittlich hoch ist (siehe Kapitel \ref{sec:nutzergruppen}). 

%Des Weiteren werden Interviews oder Gespräche mit Key-Opinion-Leaders dieser Thematik angestrebt, um Einschätzungen und Erkenntnisse aus professioneller, erster Hand zu gewinnen.

Im Anschluss der Erhebung werden die ermittelten Fragestellungen analysiert, hierbei wird untersucht:
\begin{itemize}
      \item Welche Daten für jene Frage benötigt werden
      \item Liegen die benötigten Daten vor
      \begin{itemize}
        \item Wenn ja, sind die Daten in dem richtigen Format
      \end{itemize}
      \item Wie relevant, beziehungsweise interessant die Frage allgemein ist
\end{itemize}
In Bezug darauf werden exemplarisch Dashboards für die Fragestellungen entworfen, bei welchen die benötigten Daten im richtigen Format vorliegen.
Danach werden diese vorgestellt und von entsprechenden Mitarbeitern validiert und evaluiert.
Parallel dazu werden für die Fragestellungen, welche die Daten nicht im richtigen Format vorliegen haben, entsprechende Datenmodelle oder Datenformate erarbeitet und Hinweise an die Softwareabteilung bezüglich der Schnittstelle kommuniziert.
Anschließend beginnt der gesamte Prozess mit den herausgefundenen Validierungsergebnissen erneut.


%Nach den ersten Zyklen, sobald die ersten Dashboards und Metriken intern validiert wurden, werden diese entsprechend interessierten Kunden vorgestellt.
%Die Erkenntnisse hierbei werden besonders in die Evaluierung und erneute Spezifizierung einfließen, da die Erfahrungen und der Wissensstand dieser Personen von enormer Wichtigkeit sind.

\subsection{Technologien}\label{tech}
Für das Erstellen der Dashboards wird ein Business-Intelligence-Werkzeug namens \glqq Qlik Sense\grqq{} eingesetzt.
Hierbei handelt es sich um eine leistungsstarke Software, welche auch zum Beispiel das Laden der Daten oder Umstrukturieren der geladenen Daten per Skript erlaubt.
Im Unternehmen gibt es Lizenzen für die Enterprise Variante, welche auch für den späteren produktiven Einsatz bei Kunden zum Tragen kommen wird (siehe Kapitel \ref{sub:qlik}).
Wie in Kapitel \ref{abgrenzung} beschrieben werden sich die Entwickler von \gls{GS} mit der Programmierung Schnittstelle beschäftigen, sodass die Daten vom entsprechenden Server im hier erarbeiteten Format vorliegen.
% Eine mögliche Evaluation meinerseits von Technologien, welche für die Schnittstelle hilfreich sein könnten, ist optional.

\subsection{Datenmodellierung}
Das Erarbeiten der benötigten Datenmodelle wird vorerst ohne produktive Daten erfolgen.
Hierbei ist das Ziel, eine grundlegende Struktur abzubilden, welche eine breite und nutzbare Basis von Daten schaffen soll.
Diese wird mithilfe von visuellen Datenmodellen ebenfalls in der im Kapitel \ref{tech} genannten
Software dargestellt.

Die Daten der Geräte liegen größtenteils im eigenen \glqq corpuls\grqq{}-Format vor. 
Hierbei finden sich auch sogenannte \glqq Events\grqq{}, welche bestimmte Ereignisse im Laufe einer Mission mit Parametern beschreiben, wie zum Beispiel eine Blutdruckmessung mit dem systolischen und diastolischen Druck als Parameter.
Somit ist ein einzelner Einsatz in der Auswertung als abgeschlossener atomarer Datensatz anzusehen, wobei es genau genommen eine definierte Zeitspanne mit \textit{n} Ereignissen und Messungen ist. 

Damit ein solches mehrdimensionales Datenpaket derzeit zur einfachen manuellen Auswertung geeignet ist, besteht die Notwendigkeit einer gewissen internen Vorverarbeitung dieser Daten.
Dies wird zurzeit unter anderem mit sogenannten \glqq \gls{MM} \grqq{} abgebildet.
Diese bilden Aggregationen von bestimmten Events, sodass beispielsweise die Anzahl an Blutdruckmessungen zusammengefasst wird. 
Mit diesen Aggregationen lassen sich bereits viele Auswertungen und Erkenntnisse finden.

Jedoch ist für eine tiefere Analyse eine komplexere Datenstruktur notwendig, sodass die eigentlich gebotene Mehrdimensionalität zur Verfügung stehen sollte. 
Die konfliktfreie und sinnvolle Modellierung von mehreren solcher mehrdimensionalen Daten ist eine der großen Herausforderungen dieser Bachelorarbeit (siehe Kapitel \ref{sub:erweiterung} \& \ref{sub:datenmodell}).
