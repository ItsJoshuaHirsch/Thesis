\newpage\thispagestyle{empty}
%\begin{samepage}
\section*{Zusammenfassung}

Die nachfolgende Arbeit beschäftigt sich mit der grafischen Auswertung von einer Vielzahl von Rettungsdiensteinsätzen, welche auf einem \acrlong{ANALYSE}-Server liegen.
Dies ist das Datenmanagement-Produkt der \textsf{corpuls\color{corpulsred}{.web}}-Produktfamilie, welche sich mit der Persistierung von Einsatzdaten auseinandersetzt.
Diese stammen von Geräten der Firma \acrlong{GS}, welche kabellos oder per Speicherkarte importiert wurden.

Da die Daten in Form einer großen, unübersichtlichen Tabelle vorliegen, ist die Notwendigkeit einer grafischen Darstellung essentiell, um die große Menge an Informationen in Wissen und Erkenntnisse beim Anwender zu konvertieren.

Hierfür werden gemäß etablierter Prozesse des Requirements Engineering entsprechende Anforderungen der Nutzergruppen erhoben, analysiert und spezifiziert.
Parallel erfolgt die Erstellung eines Prototypen, welche ebenfalls mit Stakeholdern evaluiert wird.
Im Zuge dessen werden geeignete Datenmodelle untersucht und daraus resultierend Entwicklungsanweisungen für einen Export der Daten erarbeitet.

Des Weiteren ist die Erweiterung des Exports um mehrdimensionale Daten ein Bestandteil dieser Arbeit, damit tiefgreifendere Auswertungen beispielsweise zu einer Herz-Lungen-Wiederbelebung möglich sind.

Schlussendlich gibt es eine Vielzahl an konzeptionierten und umgesetzten Dashboards mit Visualisierungen und Diagrammen, die den größten Teil der erhobenen Fragen der Anwender beantworten soll.

\section*{Abstract}
The following work deals with the graphical evaluation of a large number of
rescue operations which are located on a \acrlong{ANALYSE} server.
This is the data management product of the \textsf{corpuls\color{corpulsred}{.web}} product family, which deals with the persistence of rescue operation data.
These originate from devices of the company GS Elektromedizinische Geräte G. Stemple GmbH, which were imported wirelessly or by memory card.

Since the data is available in the form of a large, incomprehensible table, the necessity of a graphical representation is essential in order to convert the large amount of information into knowledge and insights for the user.

For this purpose, corresponding requirements of the user groups are collected, analyzed and specified according to established processes of requirements engineering.
At the same time, a prototype is created, which is also evaluated with stakeholders.
In this context, suitable data models are investigated and instructions for the development for an export of the data are elaborated.
\vbox{
Furthermore the extension of the export by multidimensional data is a component of this work, so that more thorough evaluations are possible for example to a cardiopulmonary resuscitation.

Finally, a variety of conceptualized and implemented dashboards with visualizations and diagrams were designed to answer most of the collected questions asked by users.}
%\end{samepage}
%\newpage