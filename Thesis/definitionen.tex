

\newglossaryentry{Zoomen}
{
     name=Zoomen,
     description={Vergrößerung ("`Zoom in"') oder Verkleinerung ("`Zoom out"') eines Bildschirminhaltes. Vom Englischen "`to zoom"', inzwischen aber auch in deutscher Umgangssprache gebräuchlich.}
}

\newglossaryentry{Key-Performance-Indicator}
{
     name=Key-Performance-Indicator,
     description={Zu Deutsch Kennzahl oder Leistungsindikator sind \glqq Zahlen, die zur Beurteilung der Leistung des betrachteten Objektes dienen. Es kann sich bei den Leistungsindikatoren entsprechend der verfolgten Ziele um Zeit-, Mengen- oder Wertgrößen handeln.\grqq{}  \cite[S. 342, S. 3f]{Friedl.2003, Maute.2009} }
}

\newglossaryentry{C.P.R}
{
     name=Cardiopulmonary resuscitation,
     description={Zu Deutsch kardiopulmonale Reanimation oder Herz-Lungen-Wiederbelebung ist die Sofortmaßnahme, wenn es zu einem Atem- und Kreislaufstillstand kommt. Dabei sollen etwa 100-120 Kompressionen des Brustkorbes mit einer optimalen Tiefe zwischen 5-6cm, sowie falls professionelles Personal vor Ort ist, Beatmungen durchgeführt werden. (vgl.\cite{Nolan.2010, Monsieurs.2015}) }
}

\newglossaryentry{CPR-Feedbacksensor}
{
     name=CPR-Feedbacksensor,
     description={Der CPR-Feedbacksensor kann beim \gls{C3}, \gls{C1} und \gls{cAED} angeschlossen werden. Dieser misst und berechnet die Daten für die Druckfrequenz und -tiefe einer Reanimation, und kann auch ad-hoc vor Ort Rückmeldung zur Reanimation geben, wie die entsprechende Frequenz und Tiefe ist.}
}

\newglossaryentry{C.C.F}
{
     name=Chest-Compression-Fraction,
     description={Zu Deutsch Brust-Kompressions-Anteil ist die \todo }
}

\newglossaryentry{A.E.D}
{
     name=Automatisierter externer Defibrillator,
     description={Ein Defibrillator, welcher aufgrund der Bauweise und häufiger Form von verschiedenen Anleitungen zur Reanimation besonders für Laien-Ersthelfer ausgelegt ist.}
}

\newglossaryentry{pacer}
{
     name=Pacer,
     description={Ist die Abkürzung für \glqq Pacemaker\grqq{}, was zu deutsch Herzschrittmacher bedeutet. }
}

%\newglossaryentry{filter}
%{
%     name=Filter?,
%     description={\todo }
%}

%\newglossaryentry{rest}
%{
%     name=REST?,
%     description={\todo }
%}

\newglossaryentry{cql}
{
     name=Corpuls Query Language,
     description={Die \glqq Corpuls Query Language\grqq ist eine \glqq Sprache\grqq{} angelehnt an SQL, welche die Abfrage von corpuls-spezifischen Daten, wie etwa \gls{MM} ermöglichen soll. Beispiel hierfür wäre eine Abfrage nach Einsätzen, welche im Jahr 2016 stattgefunden haben.}
}

\newglossaryentry{Drilldown}
{
     name=Drilldown,
     description={Bezeichnet die Navigation vom Generellen ins Spezifischere \cite{Walter.2008}}
}

%\newglossaryentry{Hovern}
%{
%     name=Hovern,
%     description={\todo }
%}

\newglossaryentry{Button}
{
     name=Button,
     description={Eine Fläche in einer Anwendung, welche interaktiv per Klick aktiviert werden kann. Die Namensgebung rührt durch die häufige Anlehnung an einen physikalischen Knopf.}
}

\newglossaryentry{Kapnografie}
{
     name=Kapnografie,
     description={\glqq Die Messung der CO2-Konzentration (etCO2)  der Ausatemluft nach der erfolgten endotrachealen Intubation gilt als ein sicheres Zeichen der korrekten Tubuslage \grqq{} \cite{Wnent.2013} }
}

\newglossaryentry{Binning}
{
     name=Binning,
     description={Das Einteilen von Daten in verschiedene Klassen oder Kategorien. Oftmals wird auch die Metapher der Töpfe oder Körbe verwendet, woher auch die Namensgebung kommt.}
}

\newglossaryentry{Feature}
{
     name=Feature,
     description={Eine (neue) Funktion eines Produktes}
}

\newglossaryentry{Dashboard}
{
     name=Dashboard,
     description={Sie sollen wie die Anzeigen im Flugzeug die wichtigsten Parameter respektive Daten auf einen kurzen Blick erkennbar machen und interpretierbar sein \cite[S.18]{Engels.2015}.}
}


\newglossaryentry{Archetyp}
{
     name=Archetyp,
     description={\glqq 2.a. (Psychologie) eins der ererbten, im kollektiven Unbewussten bereitliegenden urtümlichen Bilder, die Gestaltungen [vor]menschlicher Grunderfahrungen sind und zusammen die genetische Grundlage der Persönlichkeitsstruktur repräsentieren (nach C. G. Jung).
     2.b. (bildungssprachlich) Urform, Musterbild\grqq \cite{Dudenredaktion.2015}}
}

\newglossaryentry{uuid}
{
     name=Universally Unique Identifier,
     description={Ein \glqq Universally Unique Identifier\grqq- Nummer ist 128-Bit lang und kann Einzigartigkeit in Raum und Zeit garantieren \cite{Leach.2005}. }
}
\newglossaryentry{mm}
{
     name=Mission Marker,
     description={Dies ist die Aggregation von einem oder mehreren Events eines Einsatzes zu einem Marker. Beispiel hierfür ist das Vorkommen einer Defibrillation, woraus der Mission Marker \glqq hasShocks\grqq{} resultiert. }
}

\newglossaryentry{EKG}
{
     name=Elektrokardiogramm,
     description={Das Elektrokardiogramm, kurz EKG, ist eine nicht invasive Methode, um die elektrischen Aktivitäten, wie die Kontraktion des Herzens zu messen \cite{Gertsch.2008}.}
}


\abk{BRK}{Bayerisches Rotes Kreuz}
\abk{DRK}{Deutsches Rotes Kreuz}
\abk{GS}{GS Elektromedizinische Geräte G. Stemple GmbH}
\abk{KPI}{\gls{Key-Performance-Indicator}}
\abk{CPR}{\gls{C.P.R}}
\abk{CCF}{\gls{C.C.F}}
\abk{AED}{\gls{A.E.D}}
\abk{NIBD}{nicht-invasive Blutdruckmessung}
\abk{C3}{\textsf{corpuls\textsuperscript{\color{corpulsred}{3}}}}
\abk{C1}{\textsf{corpuls\textsuperscript{\color{corpulsred}{1}}}}
\abk{cCPR}{\textsf{corpuls \color{corpulsred}{cpr}}}
\abk{cAED}{\textsf{corpuls\textsuperscript{\color{corpulsred}{aed}}}}
\abk{LIVE}{\textsf{corpuls\color{corpulsred}{.web} \color{black}{LIVE}}}
\abk{ANALYSE}{\textsf{corpuls\color{corpulsred}{.web} \color{black}{ANALYSE}}}
\abk{MANAGER}{\textsf{corpuls\color{corpulsred}{.web} \color{black}{MANAGER}}}
\abk{REVIEW}{\textsf{corpuls\color{corpulsred}{.web} \color{black}{REVIEW}}}
\abk{REST}{Representational State Transfer}
\abk{CQL}{cql}
\abk{UUID}{\gls{uuid}}
\abk{MM}{\gls{mm}}
\abk{BI}{Business Intelligence}
\abk{DWH}{Data Warehouse}
\abk{ETL}{Extract-Transform-Load}
\abk{QMC}{Qlik Management Console}
\abk{BNF}{Backus-Naur-Formalismus}
\abk{OEM}{Original Equipment Manufacturer}
\abk{RE}{Requirements Engineering}
\abk{KIS}{Krankenhausinformationssystem}